%---------- Inleiding ---------------------------------------------------------

\section{Inleiding}%
\label{sec:inleiding}

Een ontwikkelingsteam heeft diverse verantwoordelijkheden. Naast hun primaire taak, het ontwikkelen van software, is het ook essentieel dat ze documentatie aanleveren. Dit kan variëren van API-documentatie voor andere teams tot een beginnersgids voor nieuwe medewerkers. Het is belangrijk dat deze documentatie steeds up-to-date blijft. Dit vormt echter niet de enige uitdaging; de documentatie moet ook overzichtelijk zijn en zo gestructureerd dat gebruikers eenvoudig de benodigde informatie kunnen vinden. Naarmate de tijd verstrijkt, wordt dit vaak steeds moeilijker te realiseren.

De beoogde doelgroep voor dit onderzoek bestaat uit IT-teams of organisaties die werken met een grote hoeveelheid documentatie, waarbij het soms een uitdaging is om snel specifieke informatie terug te vinden. Een virtuele assistent zou hier kunnen helpen door direct antwoorden te geven of gebruikers te verwijzen naar de relevante documentatie. Dit kan leiden tot duidelijkere inzichten in gedocumenteerde afspraken en daarmee een efficiëntere werking van het IT-team.

Het doel van dit onderzoek is om de mogelijkheden voor de ontwikkeling van een virtuele document-assistent te verkennen. Het onderzoek start met een literatuurstudie om verschillende opties en mogelijke beperkingen in kaart te brengen. Vervolgens wordt een Proof of Concept (POC) ontwikkeld, die als eerste stap dient richting de implementatie van een document-assistent.

%---------- Stand van zaken ---------------------------------------------------

\section{Stand van zaken}%
\label{sec:stand van zaken}

De opkomst van ChatGPT en andere Large Language Models (LLM) heeft de mogelijkheden voor generatieve informatieverzameling sterk vergroot. Dit roept de vraag op of dergelijke systemen ook op organisatieniveau effectief kunnen worden ingezet en welke factoren daarbij moeten worden overwogen.

Naast het bekende GPT-4 zijn er ook diverse open-source modellen, zoals Llama 2, StableBeluga2 en Mixtral 8x7B. Hoewel deze modellen over het algemeen iets minder krachtig zijn dan GPT-4, bieden ze belangrijke voordelen op het gebied van dataveiligheid en privacy. Bovendien kunnen open-source modellen eenvoudiger worden aangepast en gepersonaliseerd om beter aan te sluiten op de specifieke behoeften van de gebruiker \autocite{KernanFreire2024}.

Hoewel het ontwikkelen van een eigen gepersonaliseerde LLM wellicht als een luxeprobleem kan lijken, kan dit in sommige organisaties cruciaal zijn. Het is dan ook belangrijk om niet alleen te kijken naar de prestaties van een model, maar ook naar de mate waarin het kan worden aangepast aan de specifieke vereisten van de organisatie.

Uit het onderzoek van \textcite{Topsakal2023} blijkt dat er veel mogelijkheden zijn om zelf LLM-gebaseerde modellen op te zetten. In dit onderzoek wordt gebruikgemaakt van LangChain, een open-source framework dat specifiek is ontwikkeld om het bouwen van LLM-toepassingen te vergemakkelijken.

\textcite{Topsakal2023} onderzocht de diverse functionaliteiten van LangChain, waaronder de optie om vragen te beantwoorden op basis van documenten. In dit proces wordt de LLM getraind met de verstrekte documenten, zodat een gebruiker vragen kan stellen over deze documenten en zo snel extra kennis kan opdoen. LangChain ondersteunt een breed scala aan documenttypen, zoals CSV, PDF, HTML, JSON, Excel, GitHub, Google Drive, OneDrive en XML.

Hoewel LangChain als open-source framework veel potentie biedt, zal een verdere literatuurstudie moeten uitwijzen of er alternatieve frameworks beschikbaar zijn en wat de voor- en nadelen zijn van deze opties. Zo kan beter worden afgewogen welk framework het meest geschikt is voor specifieke use cases.

% Voor literatuurverwijzingen zijn er twee belangrijke commando's:
% \autocite{KEY} => (Auteur, jaartal) Gebruik dit als de naam van de auteur
%   geen onderdeel is van de zin.
% \textcite{KEY} => Auteur (jaartal)  Gebruik dit als de auteursnaam wel een
%   functie heeft in de zin (bv. ``Uit onderzoek door Doll & Hill (1954) bleek
%   ...'')

%---------- Methodologie ------------------------------------------------------
\section{Methodologie}%
\label{sec:methodologie}

Voor het opstarten van deze bachelorproef is een literatuurstudie essentieel om bestaande mogelijkheden op het gebied van virtuele assistenten te verkennen. De literatuurstudie richt zich op een vergelijkende analyse die inzicht moet bieden in de verschillende beschikbare opties voor het opzetten van een eigen virtuele assistent. Hierbij is het belangrijk om de voor- en nadelen van elke optie in kaart te brengen, zodat op basis van deze informatie een weloverwogen keuze kan worden gemaakt voor de uitwerking van een proof of concept. Deze studie zal tevens duidelijkheid moeten verschaffen over de benodigde hardware- en softwarevereisten voor het ontwikkelen van de proof of concept.

Een voorlopige literatuurstudie moet uitwijzen welke open-source software beschikbaar is voor het ontwikkelen van een eigen LLM-gebaseerde applicatie; LangChain is hierbij een mogelijke optie. Voor de realisatie van de proof of concept is daarnaast een frontend nodig, waarvoor een IDE zoals Visual Studio Code essentieel is om de frontend-ontwikkeling te faciliteren. Verder is een GitHub-repository noodzakelijk voor versiebeheer en samenwerking.

Voor mijn bachelorproef zijn zowel de vergelijkende studie als de proof of concept essentieel. Daarom zullen beide onderdelen elk de helft van de beschikbare tijd in beslag nemen, wat betekent dat de vergelijkende studie een periode van twee maanden krijgt, en ook voor de ontwikkeling van de proof of concept twee maanden zijn voorzien. Hoewel het mogelijk is dat tijdens de eerste twee maanden al ideeën worden verkend voor de proof of concept, of dat tijdens de ontwikkeling van de proof of concept nog aanpassingen aan de vergelijkende studie plaatsvinden, is het wel de bedoeling om eerst een gedegen vergelijkende studie op te stellen. Deze vergelijkende studie dient immers als basis voor de verdere uitwerking van de proof of concept.

%---------- Verwachte resultaten ----------------------------------------------
\section{Verwacht resultaat, conclusie}%
\label{sec:verwachte_resultaten}

Het verwachte resultaat van deze bachelorproef is tweeledig. Enerzijds wordt er een vergelijkende studie opgesteld, die voor bedrijven kan dienen als leidraad bij de implementatie van virtuele documentatie-assistenten en hen inzicht biedt in de verschillende mogelijkheden. Het is daarbij essentieel dat de vergelijkende studie niet alleen de technische aspecten behandelt, maar ook de financiële en juridische factoren in kaart brengt.

Anderzijds zal een proof of concept worden ontwikkeld als illustratie van een van de mogelijke opties binnen dit domein. Hiermee wordt de vergelijkende studie praktisch toegepast en vormt het een voorbeeld voor personen of organisaties die een soortgelijk project willen realiseren.

Bedrijven die te maken hebben met uitdagingen op het gebied van knowledge management kunnen deze vergelijkende studie benutten om te bepalen welke oplossing het beste bij hun behoeften past. Bovendien biedt de proof of concept, indien relevant en toepasbaar, de mogelijkheid om als basis te dienen voor een eigen implementatie, waarbij aanpassingen kunnen worden gemaakt om deze optimaal af te stemmen op de specifieke bedrijfscontext. Indien de proof of concept voldoende functioneel blijkt, kan deze een meerwaarde vormen door bestaande kennis en documentatie op een efficiënte en intuïtieve manier toegankelijk te maken voor medewerkers.

Het uiteindelijke onderzoeksresultaat zal een op ChatGPT lijkende interface opleveren, waarmee gebruikers gerichte antwoorden kunnen krijgen op vragen die specifiek betrekking hebben op hun organisatie. Deze oplossing richt zich dus op een bedrijfsspecifieke virtuele assistent, die getraind is op basis van de interne documentatie van een organisatie en binnen die context wordt ingezet.



