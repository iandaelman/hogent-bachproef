%==============================================================================
% Sjabloon onderzoeksvoorstel bachproef
%==============================================================================
% Gebaseerd op document class `hogent-article'
% zie <https://github.com/HoGentTIN/latex-hogent-article>

% Voor een voorstel in het Engels: voeg de documentclass-optie [english] toe.
% Let op: kan enkel na toestemming van de bachelorproefcoördinator!
\documentclass{hogent-article}

% Invoegen bibliografiebestand
\addbibresource{voorstel.bib}

% Informatie over de opleiding, het vak en soort opdracht
\studyprogramme{Professionele bachelor toegepaste informatica}
\course{Bachelorproef}
\assignmenttype{Onderzoeksvoorstel}
% Voor een voorstel in het Engels, haal de volgende 3 regels uit commentaar
% \studyprogramme{Bachelor of applied information technology}
% \course{Bachelor thesis}
% \assignmenttype{Research proposal}

\academicyear{2024-2025}

\title{Hulp voor de Helpdesk: Virtuele Assistenten als Ondersteuningstool}

% TODO: Studentnaam en emailadres invullen
\author{Ian Daelman}
\email{ian.daelman@student.hogent.be}

% TODO: Geef de co-promotor op
\supervisor[Co-promotoren]{Koen Mekers en Nathan Vlassembrouck}

%[Co-promotor]{S. Beekman (Synalco, \href{mailto:sigrid.beekman@synalco.be}{sigrid.beekman@synalco.be})}
% Binnen welke specialisatierichting uit 3TI situeert dit onderzoek zich?
% Kies uit deze lijst:
%
% - Mobile \& Enterprise development
% - AI \& Data Engineering
% - Functional \& Business Analysis
% - System \& Network Administrator
% - Mainframe Expert
% - Als het onderzoek niet past binnen een van deze domeinen specifieer je deze
%   zelf
%
\specialisation{AI \& Data Engineering}
\keywords{AGI, LLM, AI toepassing}

\begin{document}

\begin{abstract}
  
    Binnen de IT-sector is het verlenen van support aan klanten een essentieel onderdeel van de dienstverlening. Dit proces kan echter veel tijd in beslag nemen, vooral binnen grote organisaties. In dergelijke omgevingen is het niet altijd eenvoudig om snel en adequaat antwoorden te bieden op vragen van klanten. Hierdoor gaat vaak een aanzienlijke hoeveelheid tijd en middelen verloren aan het beantwoorden van deze vragen. Tegelijkertijd verwachten klanten zo snel mogelijk een antwoord op hun vraag. Het is met andere woorden in zowel het belang van de organisatie als de klant om deze vragen op een efficiënte manier te beantwoorden.
    
    Deze bachelorproef onderzoekt de mogelijkheden voor de ontwikkeling van een virtuele assistent die het zoeken naar relevante antwoorden door supportmedewerkers kan vergemakkelijken. Door middel van een grondige literatuurstudie worden verschillende opties voor een dergelijke assistent in kaart gebracht, met als doel een proof of concept te ontwikkelen. Deze virtuele assistent moet bijdragen aan het efficiënt verwerken van klantvragen en het ondersteunen van medewerkers bij het oplossen van problemen.
  
 
\end{abstract}

\tableofcontents

% De hoofdtekst van het voorstel zit in een apart bestand, zodat het makkelijk
% kan opgenomen worden in de bijlagen van de bachelorproef zelf.
%---------- Inleiding ---------------------------------------------------------

% TODO: Is dit voorstel gebaseerd op een paper van Research Methods die je
% vorig jaar hebt ingediend? Heb je daarbij eventueel samengewerkt met een
% andere student?
% Zo ja, haal dan de tekst hieronder uit commentaar en pas aan.

%\paragraph{Opmerking}

% Dit voorstel is gebaseerd op het onderzoeksvoorstel dat werd geschreven in het
% kader van het vak Research Methods dat ik (vorig/dit) academiejaar heb
% uitgewerkt (met medesturent VOORNAAM NAAM als mede-auteur).
% 

\section{Inleiding}%
\label{sec:inleiding}

Een ontwikkelingsteam heeft diverse verantwoordelijkheden. Naast hun primaire taak, het ontwikkelen van software, is het ook essentieel dat ze documentatie aanleveren. Dit kan variëren van API-documentatie voor andere teams tot een beginnersgids voor nieuwe medewerkers. Het is belangrijk dat deze documentatie steeds up-to-date blijft. Dit vormt echter niet de enige uitdaging; de documentatie moet ook overzichtelijk zijn en zo gestructureerd dat gebruikers eenvoudig de benodigde informatie kunnen vinden. Naarmate de tijd verstrijkt, wordt dit vaak steeds moeilijker te realiseren.

De beoogde doelgroep voor dit onderzoek bestaat uit IT-teams of organisaties die werken met een grote hoeveelheid documentatie, waarbij het soms een uitdaging is om snel specifieke informatie terug te vinden. Een virtuele assistent zou hier kunnen helpen door direct antwoorden te geven of gebruikers te verwijzen naar de relevante documentatie. Dit kan leiden tot duidelijkere inzichten in gedocumenteerde afspraken en daarmee een efficiëntere werking van het IT-team.

Het doel van dit onderzoek is om de mogelijkheden voor de ontwikkeling van een virtuele document-assistent te verkennen. Het onderzoek start met een literatuurstudie om verschillende opties en mogelijke beperkingen in kaart te brengen. Vervolgens wordt een Proof of Concept (POC) ontwikkeld, die als eerste stap dient richting de implementatie van een document-assistent.

%---------- Stand van zaken ---------------------------------------------------

\section{Stand van zaken}%
\label{sec:stand van zaken}

Hier beschrijf je de \emph{state-of-the-art} rondom je gekozen onderzoeksdomein, d.w.z.\ een inleidende, doorlopende tekst over het onderzoeksdomein van je bachelorproef. Je steunt daarbij heel sterk op de professionele \emph{vakliteratuur}, en niet zozeer op populariserende teksten voor een breed publiek. Wat is de huidige stand van zaken in dit domein, en wat zijn nog eventuele open vragen (die misschien de aanleiding waren tot je onderzoeksvraag!)?

Je mag de titel van deze sectie ook aanpassen (literatuurstudie, stand van zaken, enz.). Zijn er al gelijkaardige onderzoeken gevoerd? Wat concluderen ze? Wat is het verschil met jouw onderzoek?

Verwijs bij elke introductie van een term of bewering over het domein naar de vakliteratuur, bijvoorbeeld~\autocite{Hykes2015}! Denk zeker goed na welke werken je refereert en waarom.

Draag zorg voor correcte literatuurverwijzingen! Een bronvermelding hoort thuis \emph{binnen} de zin waar je je op die bron baseert, dus niet er buiten! Maak meteen een verwijzing als je gebruik maakt van een bron. Doe dit dus \emph{niet} aan het einde van een lange paragraaf. Baseer nooit teveel aansluitende tekst op eenzelfde bron.

Als je informatie over bronnen verzamelt in JabRef, zorg er dan voor dat alle nodige info aanwezig is om de bron terug te vinden (zoals uitvoerig besproken in de lessen Research Methods).

% Voor literatuurverwijzingen zijn er twee belangrijke commando's:
% \autocite{KEY} => (Auteur, jaartal) Gebruik dit als de naam van de auteur
%   geen onderdeel is van de zin.
% \textcite{KEY} => Auteur (jaartal)  Gebruik dit als de auteursnaam wel een
%   functie heeft in de zin (bv. ``Uit onderzoek door Doll & Hill (1954) bleek
%   ...'')

Je mag deze sectie nog verder onderverdelen in subsecties als dit de structuur van de tekst kan verduidelijken.

%---------- Methodologie ------------------------------------------------------
\section{Methodologie}%
\label{sec:methodologie}


Voor het opstarten van deze bachelorproef is een literatuurstudie essentieel om bestaande mogelijkheden op het gebied van virtuele assistenten te verkennen. De literatuurstudie richt zich op een vergelijkende analyse die inzicht moet bieden in de verschillende beschikbare opties voor het opzetten van een eigen virtuele assistent. Hierbij is het belangrijk om de voor- en nadelen van elke optie in kaart te brengen, zodat op basis van deze informatie een weloverwogen keuze kan worden gemaakt voor de uitwerking van een proof of concept. Deze studie zal tevens duidelijkheid moeten verschaffen over de benodigde hardware- en softwarevereisten voor het ontwikkelen van de proof of concept.

Voor de realisatie van de proof of concept is een frontend nodig, waarvoor een IDE zoals Visual Studio Code essentieel is om de frontend-ontwikkeling te ondersteunen. Daarnaast is een GitHub-repository nodig voor versiebeheer en samenwerking.

Voor mijn bachelorproef zijn zowel de vergelijkende studie als de proof of concept essentieel. Daarom zullen beide onderdelen elk de helft van de beschikbare tijd in beslag nemen, wat betekent dat de vergelijkende studie een periode van twee maanden krijgt, en ook voor de ontwikkeling van de proof of concept twee maanden zijn voorzien. Hoewel het mogelijk is dat tijdens de eerste twee maanden al ideeën worden verkend voor de proof of concept, of dat er tijdens de ontwikkeling van de proof of concept nog aanpassingen aan de vergelijkende studie plaatsvinden, is het wel de bedoeling om eerst een gedegen vergelijkende studie op te stellen. Deze vergelijkende studie dient als basis voor de verdere uitwerking van de proof of concept.

%---------- Verwachte resultaten ----------------------------------------------
\section{Verwacht resultaat, conclusie}%
\label{sec:verwachte_resultaten}

Het verwachte resultaat van deze bachelorproef is tweeledig. Enerzijds wordt er een vergelijkende studie opgesteld, die voor bedrijven kan dienen als leidraad bij de implementatie van virtuele documentatie-assistenten en hen inzicht biedt in de verschillende mogelijkheden. Het is daarbij essentieel dat de vergelijkende studie niet alleen de technische aspecten behandelt, maar ook de financiële en juridische factoren in kaart brengt.

Anderzijds zal een proof of concept worden ontwikkeld als illustratie van een van de mogelijke opties binnen dit domein. Hiermee wordt de vergelijkende studie praktisch toegepast en vormt het een voorbeeld voor personen of organisaties die een soortgelijk project willen realiseren.

Bedrijven die te maken hebben met uitdagingen op het gebied van knowledge management kunnen deze vergelijkende studie benutten om te bepalen welke oplossing het beste bij hun behoeften past. Bovendien biedt de proof of concept, indien relevant en toepasbaar, de mogelijkheid om als basis te dienen voor een eigen implementatie, waarbij aanpassingen kunnen worden gemaakt om deze optimaal af te stemmen op de specifieke bedrijfscontext. Indien de proof of concept voldoende functioneel blijkt, kan deze een meerwaarde vormen door bestaande kennis en documentatie op een efficiënte en intuïtieve manier toegankelijk te maken voor medewerkers.

Het uiteindelijke onderzoeksresultaat zal een op ChatGPT lijkende interface opleveren, waarmee gebruikers gerichte antwoorden kunnen krijgen op vragen die specifiek betrekking hebben op hun organisatie. Deze oplossing richt zich dus op een bedrijfsspecifieke virtuele assistent, die getraind is op basis van de interne documentatie van een organisatie en binnen die context wordt ingezet.





\printbibliography[heading=bibintoc]

\end{document}