%%=============================================================================
%% Inleiding
%%=============================================================================

\chapter{\IfLanguageName{dutch}{Inleiding}{Introduction}}%
\label{ch:inleiding}

Met de opkomst van artificiële intelligentie (AI) en Large Language Models (LLM’s) ontstaan er steeds meer mogelijkheden om deze technologieën in te zetten voor intelligente IT-support chatbots. Traditionele IT-supportsystemen vereisen vaak veel menselijke tussenkomst, wat kan leiden tot langere wachttijden en hogere operationele kosten. Een LLM-gebaseerde chatbot kan hier een oplossing bieden door vragen snel en accuraat te beantwoorden, waardoor de efficiëntie van IT-supportteams toeneemt.

\section{\IfLanguageName{dutch}{Context en Probleemstelling}{Problem Statement}}%
\label{sec:probleemstelling}

Deze bachelorproef richt zich specifiek op de IT-support binnen de Federale Overheidsdienst (FOD) Financiën. Deze organisatie biedt een uitgebreide en complexe vorm van support aan zowel interne medewerkers als externe klanten over een breed scala aan onderwerpen, zoals belastingaangiftes, fiscale wetgeving en technische ondersteuning voor digitale overheidsdiensten. Door de omvang en diversiteit van de vragen waarmee de FOD Financiën wordt geconfronteerd, is er een duidelijke behoefte aan een efficiëntere ondersteuning.

Om te onderzoeken hoe een LLM-gebaseerde chatbot hieraan kan bijdragen, worden twee belangrijke benaderingen vergeleken: Retrieval Augmented Generation (RAG) en Cache-Augmented Generation (CAG). De keuze tussen deze methoden heeft een aanzienlijke impact op de kwaliteit en relevantie van de gegenereerde antwoorden. Daarom zal in deze bachelorproef een Proof of Concept (PoC) worden ontwikkeld en getest binnen de context van de FOD Financiën, met als doel de meest geschikte aanpak te identificeren. Hoewel deze studie zich richt op deze specifieke usecase, kunnen de bevindingen en methodieken ook relevant zijn voor andere organisaties met complexe IT-supportvragen.

\section{\IfLanguageName{dutch}{Onderzoeksvraag}{Research question}}%
\label{sec:onderzoeksvraag}

De centrale onderzoeksvraag van deze bachelorproef luidt:

“\textbf{Welke methode, Retrieval Augmented Generation (RAG) of Cache-Augmented Generation (CAG), is het meest geschikt voor een LLM-gebaseerde IT-support chatbot binnen de context van FOD Financiën, en hoe presteren verschillende implementaties van de gekozen methode?}”

Om deze onderzoeksvraag te beantwoorden, worden de volgende deelvragen geformuleerd:

\begin{itemize}
    \item Wat zijn de belangrijkste verschillen tussen RAG en CAG volgens de bestaande literatuur?
    \item Welke LLM-modellen zijn het meest geschikt binnen deze implementatie volgens de bestaande literatuur?
    \item Welke methode is het meest geschikt voor een IT-support chatbot binnen de FOD Financiën op basis van relevante criteria (bv. nauwkeurigheid, snelheid, gebruiksvriendelijkheid, kosten)?
    \item Hoe presteren verschillende implementaties van de gekozen methode in een Proof of Concept (PoC)?
    \item Hoe ervaren gebruikers de chatbot en welke implementatie levert de meest accurate en efficiënte antwoorden?
\end{itemize}

\section{\IfLanguageName{dutch}{Onderzoeksdoelstelling}{Research objective}}%
\label{sec:onderzoeksdoelstelling}

Het doel van deze bachelorproef is tweeledig. Enerzijds wordt een Proof of Concept (PoC) ontwikkeld die actief gebruikt kan worden om de IT-support binnen de FOD Financiën efficiënter te organiseren. Hiervoor worden verschillende versies van de PoC geïmplementeerd en met elkaar vergeleken om de meest geschikte aanpak te bepalen.

Anderzijds biedt deze studie een theoretisch overzicht van de verschillende methoden die beschikbaar zijn voor het ontwikkelen van een LLM-gebaseerde IT-support chatbot. Dit omvat een vergelijking tussen RAG en CAG, waarbij hun respectieve voor- en nadelen worden geanalyseerd op basis van bestaande literatuur.

De combinatie van deze praktische en theoretische analyse moet leiden tot een onderbouwde aanbeveling voor de implementatie van een IT-support chatbot binnen complexe organisaties zoals de FOD Financiën.

\section{\IfLanguageName{dutch}{Opzet van deze bachelorproef}{Structure of this bachelor thesis}}%
\label{sec:opzet-bachelorproef}

De rest van deze bachelorproef is als volgt opgebouwd:

In Hoofdstuk~\ref{ch:stand-van-zaken} wordt een overzicht gegeven van de stand van zaken binnen het onderzoeksdomein, op basis van een literatuurstudie.

In Hoofdstuk~\ref{ch:methodologie} wordt de methodologie toegelicht en worden de gebruikte onderzoekstechnieken besproken om een antwoord te kunnen formuleren op de onderzoeksvragen.

In Hoofdstuk~\ref{ch:resultaten} worden de bevindingen van het onderzoek gepresenteerd. De verzamelde gegevens worden geanalyseerd en overzichtelijk weergegeven.

In Hoofdstuk~\ref{ch:discussie} worden de resultaten geïnterpreteerd. Daarnaast komen de sterktes en beperkingen van het onderzoek aan bod en wordt een vergelijking gemaakt met bestaande literatuur.

In Hoofdstuk~\ref{ch:aanbevelingen} worden mogelijke implicaties van de onderzoeksresultaten besproken en worden aanbevelingen geformuleerd voor de praktijk en/of toekomstig onderzoek.

In Hoofdstuk~\ref{ch:conclusie}, tenslotte, wordt de conclusie gegeven en een antwoord geformuleerd op de onderzoeksvragen.