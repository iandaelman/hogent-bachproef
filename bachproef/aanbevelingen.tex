%%=============================================================================
%% Aanbevelingen
%%=============================================================================

\chapter{Aanbevelingen}
\label{ch:aanbevelingen}

Op basis van de resultaten van dit onderzoek kunnen diverse aanbevelingen worden geformuleerd die bijdragen aan de verdere ontwikkeling van de bestaande PoC.
Allereerst wordt aanbevolen om de huidige flow te testen met de modellen die beschikbaar zijn binnen de FOD Financiën. Dit maakt het mogelijk om de prestaties van de PoC te evalueren en te bepalen in hoeverre krachtigere modellen leiden tot betere resultaten.
\\[1em]
Daarnaast is het raadzaam om een vorm van geheugen aan de PoC toe te voegen. In eerste instantie op het niveau van een sessie. Hierdoor kan de toepassing eerdere interacties tijdens een gesprek met de gebruiker onthouden en daarop inspelen, wat resulteert in een natuurlijker en effectiever gesprek.
\\[1em]
Verder moet de bestaande support documentatie geherevalueerd worden. Functionaliteiten en processen die momenteel niet of onvoldoende zijn beschreven, dienen te worden aangevuld. Aangezien de PoC gebaseerd is op een RAG flow, is het essentieel dat alle relevante informatie volledig en accuraat in de documentatie aanwezig is. Ontbrekende informatie kan immers niet door het systeem worden opgehaald, wat de kwaliteit van de antwoorden negatief beïnvloedt.
\\[1em]
Ook wordt aanbevolen te onderzoeken in hoeverre de Confluence API kan worden ingezet voor het automatisch ophalen van documentatie. Hierdoor kan de meest recente informatie direct in de vectordatabase worden opgenomen, wat het onderhoud van de toepassing vereenvoudigt.
\\[1em]
Op de langere termijn kan deze oplossing bovendien verder worden ontwikkeld tot een ambient agent die, met toegang tot de mailbox, proactief relevante informatie opzoekt over binnenkomende vragen en deze direct aanbiedt aan de medewerker die op dat moment verantwoordelijk is voor de support.