\chapter{\IfLanguageName{dutch}{Stand van zaken}{State of the art}}%
\label{ch:stand-van-zaken}

% Tip: Begin elk hoofdstuk met een paragraaf inleiding die beschrijft hoe
% dit hoofdstuk past binnen het geheel van de bachelorproef. Geef in het
% bijzonder aan wat de link is met het vorige en volgende hoofdstuk.

% Pas na deze inleidende paragraaf komt de eerste sectiehoofding.

Dit hoofdstuk bevat je literatuurstudie. De inhoud gaat verder op de inleiding, maar zal het onderwerp van de bachelorproef *diepgaand* uitspitten. De bedoeling is dat de lezer na lezing van dit hoofdstuk helemaal op de hoogte is van de huidige stand van zaken (state-of-the-art) in het onderzoeksdomein. Iemand die niet vertrouwd is met het onderwerp, weet nu voldoende om de rest van het verhaal te kunnen volgen, zonder dat die er nog andere informatie moet over opzoeken \autocite{Pollefliet2011}.

Je verwijst bij elke bewering die je doet, vakterm die je introduceert, enz.\ naar je bronnen. In \LaTeX{} kan dat met het commando \texttt{$\backslash${textcite\{\}}} of \texttt{$\backslash${autocite\{\}}}. Als argument van het commando geef je de ``sleutel'' van een ``record'' in een bibliografische databank in het Bib\LaTeX{}-formaat (een tekstbestand). Als je expliciet naar de auteur verwijst in de zin (narratieve referentie), gebruik je \texttt{$\backslash${}textcite\{\}}. Soms is de auteursnaam niet expliciet een onderdeel van de zin, dan gebruik je \texttt{$\backslash${}autocite\{\}} (referentie tussen haakjes). Dit gebruik je bv.~bij een citaat, of om in het bijschrift van een overgenomen afbeelding, broncode, tabel, enz. te verwijzen naar de bron. In de volgende paragraaf een voorbeeld van elk.

\textcite{Knuth1998} schreef een van de standaardwerken over sorteer- en zoekalgoritmen. Experten zijn het erover eens dat cloud computing een interessante opportuniteit vormen, zowel voor gebruikers als voor dienstverleners op vlak van informatietechnologie~\autocite{Creeger2009}.

Let er ook op: het \texttt{cite}-commando voor de punt, dus binnen de zin. Je verwijst meteen naar een bron in de eerste zin die erop gebaseerd is, dus niet pas op het einde van een paragraaf.

Vragen te beantwoorden:

\section{Inleiding}
In dit hoofdstuk wordt de werking van een Retrieval-Augmented Generation (RAG) model besproken. We behandelen de belangrijkste concepten en recente ontwikkelingen binnen zowel RAG als Large Language Models (LLM). Dit overzicht biedt inzicht in de huidige stand van zaken en vormt de basis voor een weloverwogen keuze bij de ontwikkeling van een Proof of Concept (PoC).

Concreet komen de volgende onderwerpen aan bod:
\begin{itemize}
    \item \textbf{LLM-modellen voor Retrieval-Augmented Generation (RAG)} – een diepgaande analyse van beschikbare modellen en hun geschiktheid voor RAG.
    \item \textbf{De AI Act en de belangrijkste richtlijnen} – een overzicht van de relevante wet- en regelgeving en de impact hiervan op RAG-toepassingen.
    \item \textbf{Best practices voor IT-supportprocessen} – inzichten in hoe RAG kan worden toegepast binnen IT-support en welke methodologieën daarbij best worden gevolgd.
\end{itemize}

De nadruk zal voornamelijk liggen op het eerste luik, \textit{LLM-modellen voor Retrieval-Augmented Generation (RAG)}, aangezien dit een cruciale factor is voor het maken van een doordachte keuze bij de verdere uitwerking van de PoC. Indien nodig zullen de overige onderwerpen verder worden uitgediept.

\section{LLM-modellen voor Retrieval-Augmented Generation (RAG)}
Welke bestaande LLM-modellen kunnen worden gebruikt voor het ontwikkelen van een Retrieval-Augmented Generation (RAG)?

        \subsubsection{Introductie}
    
    \subsection{Voor- en nadelen van bestaande LLM-modellen}
    Wat zijn de belangrijkste voor- en nadelen van deze modellen?
    
    \subsection{Tools en frameworks voor RAG}
    Welke bestaande tools en frameworks kunnen bijdragen aan het opzetten van een RAG?

\section{De AI Act en de belangrijkste richtlijnen}
Wat houdt de AI Act in, en wat zijn de belangrijkste richtlijnen die hierin moeten worden gevolgd?

\section{Best practices voor IT-supportprocessen}
Wat zijn de bestaande best practices voor de organisatie van IT-supportprocessen?
