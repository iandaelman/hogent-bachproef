%%=============================================================================
%% Conclusie
%%=============================================================================

\chapter{Conclusie}%
\label{ch:conclusie}

% Trek een duidelijke conclusie, in de vorm van een antwoord op de
% onderzoeksvra(a)g(en). Wat was jouw bijdrage aan het onderzoeksdomein en
% hoe biedt dit meerwaarde aan het vakgebied/doelgroep? 
% Reflecteer kritisch over het resultaat. In Engelse teksten wordt deze sectie
% ``Discussion'' genoemd. Had je deze uitkomst verwacht? Zijn er zaken die nog
% niet duidelijk zijn?
% Heeft het onderzoek geleid tot nieuwe vragen die uitnodigen tot verder 
%onderzoek?


% onderzoeksvra(a)g(en). Wat was jouw bijdrage aan het onderzoeksdomein en
% hoe biedt dit meerwaarde aan het vakgebied/doelgroep? 
% Reflecteer kritisch over het resultaat. In Engelse teksten wordt deze sectie

%TODO Dit is een goeie basis! Check wel eens of je op alle onderzoeksvragen een antwoord geeft in de conclusie.
Dit onderzoek biedt een helder overzicht van de verschillende mogelijkheden om een virtuele assistent te ontwikkelen. De uitgevoerde literatuurstudie maakt het mogelijk om, afhankelijk van de specifieke use case, een weloverwogen keuze te maken tussen de verschillende benaderingen. Zowel fine-tuning, RAG als CAG kunnen bijdragen aan het verminderen van de tekortkomingen van huidige LLM's, bijvoorbeeld door domeinspecifieke kennis toe te voegen of door hallucinaties via gerichte prompting te beperken.
\\[1em]
Door de beperkte scope van de PoC was fine-tuning niet mogelijk. Om de schaalbaarheid te behouden en de kosten per query zo laag als mogelijk te houden, kon CAG eveneens niet worden gekozen. Bijgevolg werd de keuze voor RAG gemaakt. Deze methode paste binnen de scope en biedt voordelen op het gebied van schaalbaarheid en kosten die CAG niet biedt.
\\[1em]
De uitgewerkte PoC toont aan hoe RAG in de praktijk kan worden toegepast. Niet alleen de PoC, maar ook de problemen die zich tijdens de ontwikkeling voordeden, bieden waardevolle lessen voor anderen die een vergelijkbare PoC wil realiseren.
\\[1em]
De resultaten tonen aan dat de keuze van het LLM-model een cruciale rol speelt. Niet elk model is geschikt voor elke taak, zoals duidelijk werd bij functionaliteiten zoals tool calling. Binnen deze studie konden echter enkel kleinere modellen worden getest,waardoor het niet uitgesloten is dat krachtigere varianten van dezelfde modellen beter presteren dan de geselecteerde modellen.
\\[1em]
Aan de hand van verschillende testsituaties konden de prestaties van de modellen worden geëvalueerd. Hieruit bleek dat het model Qwen3 de beste prestaties leverde van de geteste modellen.
\\[1em]
Heel wat aspecten binnen dit onderzoek werden proefondervindelijk getest. De keuzes die binnen deze PoC gemaakt werden, zijn echter niet per definitie de beste voor elke RAG use case. Elke toepassing vraagt om een specifieke afweging, afhankelijk van context, noden en technische randvoorwaarden.
\\[1em]
Het zou daarom waardevol zijn om in vervolgonderzoek niet enkel modellen onderling te vergelijken, maar ook andere aspecten van een RAG-architectuur nader te analyseren. Een eerste voorbeeld is het effect van verschillende parsers op ongestructureerde bestanden. Het is interessant om na te gaan welke parser het best omgaat met complexe PDF-bestanden die niet uitsluitend uit tekst bestaan, zoals gescande documenten of formulieren met tabellen en grafieken.
\\[1em]
Een tweede mogelijk onderzoekspiste betreft de impact van verschillende retrieval-methodes. Momenteel zijn er vier methodes beschikbaar, elk met een eigen aanpak voor het ophalen van documenten uit de vectordatabase. Een vergelijking van deze methodes op vlak van nauwkeurigheid, snelheid en robuustheid kan waardevolle inzichten opleveren.
\\[1em]
Een derde mogelijkheid is het analyseren van het effect van modelgrootte. In dit onderzoek werden voornamelijk modellen gebruikt met een vergelijkbaar aantal parameters. Het zou echter interessant zijn om na te gaan hoe verschillende versies van eenzelfde model met meer of minder parameters presteren in een RAG-context. Dit kan duidelijkheid scheppen over de verhouding tussen prestaties en efficiëntie.
\\[1em]
Tot slot kan het ook waardevol zijn om alternatieven voor RAG te verkennen. Binnen dit onderzoek werd bewust gekozen voor een RAG aanpak, maar gezien de snelle evolutie van LLM's en de steeds grotere context die deze modellen ter beschikking hebben kan het relevant zijn om te onderzoeken hoe een CAG systeem functioneert en of dit een valabel alternatief kan vormen.
\\[1em]
Samenvattend toont deze bachelorproef aan dat RAG een haalbare en veelbelovende aanpak is voor de ontwikkeling van een LLM-gebaseerde IT-supportbot in de context van MyMinfin. De opgedane inzichten vormen een stevige basis voor verdere optimalisatie en uitbreiding van dergelijke toepassing in de toekomst.