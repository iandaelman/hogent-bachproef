%%=============================================================================
%% Conclusie
%%=============================================================================

\chapter{Conclusie}%
\label{ch:conclusie}

% TODO: Trek een duidelijke conclusie, in de vorm van een antwoord op de
% onderzoeksvra(a)g(en). Wat was jouw bijdrage aan het onderzoeksdomein en
% hoe biedt dit meerwaarde aan het vakgebied/doelgroep? 
% Reflecteer kritisch over het resultaat. In Engelse teksten wordt deze sectie
% ``Discussion'' genoemd. Had je deze uitkomst verwacht? Zijn er zaken die nog
% niet duidelijk zijn?
% Heeft het onderzoek geleid tot nieuwe vragen die uitnodigen tot verder 
%onderzoek?


VRAGEN DIE TOT VERDER ONDERZOEK LEIDEN:
\\[1em]
%TODO laten corrigeren op taal door ChatGPT

Heel wat zaken binnen dit onderzoek werden proefondervindelijk getest maar de keuzes die binnen deze PoC werden gemaakt zijn niet per definitie steeds de beste keuzes voor iedere RAG usecase. 
\\[1em]
Het zou dus met andere woorden interessant zijn om naast een vergelijking van verschillende modellen andere aspecten van een RAG architectuur onder de loep te nemen.
Bijvoorbeeld wat de impact is van verschillende parsers op ongestructureerde bestanden. Wat zijn de verschillende opties en welke kan het best omgaan met PDF bestanden die niet louter uit tekst bestaan. 
\\[1em]
Een tweede mogelijk onderzoek is nagaan wat de impact is van de verschillende beschikbare retrieval methodes. Tot op heden zijn er vier methodes beschikbaar en elk hebben ze een andere manier om documenten uit de vector database te halen.
\\[1em]
Een andere mogelijkheid is om na te gaan wat de impact is van modellen met een groter of kleiner aantal parameters. Voor dit onderzoek werden verschillende modellen onderzocht met gelijkaardige aantal parameters maar het zou interessant zijn om na te gaan hoe verschillende versies van een model presteren tegenover elkaar. 
\\[1em]
Een laatste mogelijk onderzoek is kiezen voor een andere methode. Binnen dit onderzoek werd gekozen voor RAG maar met de evolutie van de modellen en de grotere context die deze bieden is het zeker mogelijk om na te gaan hoe een CAG systeem functioneert en kan werken.

