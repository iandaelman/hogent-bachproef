%%=============================================================================
%% Conclusie
%%=============================================================================

\chapter{Conclusie}%
\label{ch:conclusie}

% TODO: Trek een duidelijke conclusie, in de vorm van een antwoord op de
% onderzoeksvra(a)g(en). Wat was jouw bijdrage aan het onderzoeksdomein en
% hoe biedt dit meerwaarde aan het vakgebied/doelgroep? 
% Reflecteer kritisch over het resultaat. In Engelse teksten wordt deze sectie
% ``Discussion'' genoemd. Had je deze uitkomst verwacht? Zijn er zaken die nog
% niet duidelijk zijn?
% Heeft het onderzoek geleid tot nieuwe vragen die uitnodigen tot verder 
%onderzoek?


VRAGEN DIE TOT VERDER ONDERZOEK LEIDEN:
\\[1em]
%TODO laten corrigeren op taal door ChatGPT

Heel wat elementen binnen dit onderzoek werden proefondervindelijk getest. De keuzes die binnen deze PoC gemaakt werden, zijn echter niet per definitie de beste voor elke RAG usecase. Elke toepassing vraagt om een specifieke afweging, afhankelijk van context, noden en technische randvoorwaarden.
\\[1em]
Het zou daarom waardevol zijn om in vervolgonderzoek niet enkel modellen onderling te vergelijken, maar ook andere aspecten van een RAG-architectuur nader te analyseren. Een eerste voorbeeld is het effect van verschillende parsers op ongestructureerde bestanden. Het is interessant om na te gaan welke parser het best omgaat met complexe PDF-bestanden die niet uitsluitend uit tekst bestaan, zoals gescande documenten of formulieren met tabellen en grafieken.
\\[1em]
Een tweede mogelijk onderzoekspiste betreft de impact van verschillende retrieval-methodes. Momenteel zijn er vier methodes beschikbaar, elk met een eigen aanpak voor het ophalen van documenten uit de vectordatabase. Een vergelijking van deze methodes op vlak van nauwkeurigheid, snelheid en robuustheid kan waardevolle inzichten opleveren.
\\[1em]
Een derde mogelijkheid is het analyseren van het effect van modelgrootte. In dit onderzoek werden voornamelijk modellen gebruikt met een vergelijkbaar aantal parameters. Het zou echter interessant zijn om na te gaan hoe verschillende versies van eenzelfde model met meer of minder parameters presteren in een RAG-context. Dit kan duidelijkheid scheppen over de verhouding tussen prestaties en efficiëntie.
\\[1em]
Tot slot kan het ook waardevol zijn om alternatieven voor RAG te verkennen. Binnen dit onderzoek werd bewust gekozen voor een RAG aanpak, maar gezien de snelle evolutie van LLM's en de steeds grotere context die deze modellen ter beschikking hebben kan het relevant zijn om te onderzoeken hoe een Cache-Augmented Generation (CAG) systeem functioneert en of dit een valabel alternatief kan vormen.
