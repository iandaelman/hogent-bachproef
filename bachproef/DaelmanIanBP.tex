%===============================================================================
% LaTeX sjabloon voor de bachelorproef toegepaste informatica aan HOGENT
% Meer info op https://github.com/HoGentTIN/latex-hogent-report
%===============================================================================

\documentclass[dutch,dit,thesis]{hogentreport}

\usepackage{lipsum}  % For blind text, can be removed after adding actual content
\usepackage{subcaption}
\usepackage{listings}
\usepackage{subcaption}
\usepackage{pgfplots}
\pgfplotsset{compat=1.18}

\usepackage{xcolor}

\definecolor{codegreen}{rgb}{0,0.6,0}
\definecolor{codegray}{rgb}{0.5,0.5,0.5}
\definecolor{codepurple}{rgb}{0.58,0,0.82}
\definecolor{backcolour}{rgb}{0.95,0.95,0.92}

\lstdefinestyle{mystyle}{
    backgroundcolor=\color{backcolour},   
    commentstyle=\color{codegreen},
    keywordstyle=\color{magenta},
    numberstyle=\tiny\color{codegray},
    stringstyle=\color{codepurple},
    basicstyle=\ttfamily\footnotesize,
    breakatwhitespace=false,         
    breaklines=true,                 
    captionpos=b,                    
    keepspaces=true,                 
    numbers=left,                    
    numbersep=5pt,                  
    showspaces=false,                
    showstringspaces=false,
    showtabs=false,                  
    tabsize=2
}

\lstset{style=mystyle}

%% Pictures to include in the text can be put in the graphics/ folder
\graphicspath{{../graphics/}}

%% For source code highlighting, requires pygments to be installed
%% Compile with the -shell-escape flag!
%% \usepackage[chapter]{minted}
%% If you compile with the make_thesis.{bat,sh} script, use the following
%% import instead:
\usepackage[chapter]{minted}
\usemintedstyle{solarized-light}

%% Formatting for minted environments.
\setminted{%
    autogobble,
    frame=lines,
    breaklines,
    linenos,
    tabsize=4
}

%% Ensure the list of listings is in the table of contents
\renewcommand\listoflistingscaption{%
    \IfLanguageName{dutch}{Lijst van codefragmenten}{List of listings}
}
\renewcommand\listingscaption{%
    \IfLanguageName{dutch}{Codefragment}{Listing}
}
\renewcommand*\listoflistings{%
    \cleardoublepage\phantomsection\addcontentsline{toc}{chapter}{\listoflistingscaption}%
    \listof{listing}{\listoflistingscaption}%
}

% Other packages not already included can be imported here

%%---------- Document metadata -------------------------------------------------
\author{Ian Daelman}
\supervisor{Dhr. T. Aelbrecht}
\cosupervisor{Dhr. K. Mekers}
\title{Hulp voor de Helpdesk: Virtuele Assistenten als Ondersteuningstool.}
\academicyear{\advance\year by -1 \the\year--\advance\year by 1 \the\year}
\examperiod{1}
\degreesought{\IfLanguageName{dutch}{Professionele bachelor in de toegepaste informatica}{Bachelor of applied computer science}}
\partialthesis{false} %% To display 'in partial fulfilment'
%\institution{Internshipcompany BVBA.}

%% Add global exceptions to the hyphenation here
\hyphenation{back-slash}

%% The bibliography (style and settings are  found in hogentthesis.cls)
\addbibresource{bachproef.bib}            %% Bibliography file
\addbibresource{../voorstel/voorstel.bib} %% Bibliography research proposal
\defbibheading{bibempty}{}

%% Prevent empty pages for right-handed chapter starts in twoside mode
\renewcommand{\cleardoublepage}{\clearpage}

\renewcommand{\arraystretch}{1.2}

%% Content starts here.
\begin{document}

%---------- Front matter -------------------------------------------------------

\frontmatter

\hypersetup{pageanchor=false} %% Disable page numbering references
%% Render a Dutch outer title page if the main language is English
\IfLanguageName{english}{%
    %% If necessary, information can be changed here
    \degreesought{Professionele Bachelor toegepaste informatica}%
    \begin{otherlanguage}{dutch}%
       \maketitle%
    \end{otherlanguage}%
}{}

%% Generates title page content
\maketitle
\hypersetup{pageanchor=true}

%%=============================================================================
%% Voorwoord
%%=============================================================================

\chapter{Woord vooraf}
\label{ch:voorwoord}
%% TODO:
%% Het voorwoord is het enige deel van de bachelorproef waar je vanuit je
%% eigen standpunt (``ik-vorm'') mag schrijven. Je kan hier bv. motiveren
%% waarom jij het onderwerp wil bespreken.
%% Vergeet ook niet te bedanken wie je geholpen/gesteund/... heeft

Deze bachelorproef is het slotstuk van een opleiding die ik bijna vijf jaar geleden begon als werkstudent. Het is niet enkel het einde van mijn studies, maar ook van een periode die veel heeft veranderd in mijn leven. Zowel op academisch als professioneel vlak heeft deze opleiding mij enorm veel bijgebracht en nieuwe inzichten gegeven. Het onderwerp van dit eindwerk sluit daar mooi op aan, omdat het te maken heeft met een probleem die rechtstreeks uit mijn werkomgeving komt.
\\[1em]
Tijdens mijn dagelijkse werkzaamheden merkte ik samen met mijn collega’s dat het supportproces efficiënter kon worden georganiseerd. Deze behoefte vormde het startpunt voor dit onderzoek. Met de snelle opkomst van LLM-toepassingen zag ik een mooie kans om deze technologie te verkennen en te kijken welke mogelijkheden er zijn om een IT-supportbot te ontwikkelen die hierin een hulp kan zijn.
\\[1em]
Deze bachelorproef was er nooit gekomen zonder het advies van mijn promotor, Thomas Aelbrecht en mijn co-promotor Koen Mekers. Hun hulp, advies en bereidheid om mee te denken hebben voor mij een wereld van verschil gemaakt. Daarom wil ik hen op deze plek nog eens heel oprecht bedanken.
\\[1em]
Met dit werk sluit ik een bijzonder hoofdstuk af en kijk ik met veel enthousiasme uit naar wat er nog komt.





%%=============================================================================
%% Samenvatting
%%=============================================================================

% TODO: De "abstract" of samenvatting is een kernachtige (~ 1 blz. voor een
% thesis) synthese van het document.
%
% Een goede abstract biedt een kernachtig antwoord op volgende vragen:
%
% 1. Waarover gaat de bachelorproef?
% 2. Waarom heb je er over geschreven?
% 3. Hoe heb je het onderzoek uitgevoerd?
% 4. Wat waren de resultaten? Wat blijkt uit je onderzoek?
% 5. Wat betekenen je resultaten? Wat is de relevantie voor het werkveld?
%
% Daarom bestaat een abstract uit volgende componenten:
%
% - inleiding + kaderen thema
% - probleemstelling
% - (centrale) onderzoeksvraag
% - onderzoeksdoelstelling
% - methodologie
% - resultaten (beperk tot de belangrijkste, relevant voor de onderzoeksvraag)
% - conclusies, aanbevelingen, beperkingen
%
% LET OP! Een samenvatting is GEEN voorwoord!

%%---------- Samenvatting -----------------------------------------------------
% De samenvatting in de hoofdtaal van het document

\chapter {Samenvatting}
\label{ch:samenvatting}

Deze bachelorproef onderzoekt de mogelijkheden om een LLM-gebaseerde chatbot te ontwikkelen die IT-supportmedewerkers kan bijstaan bij het behandelen van technische IT vragen. Gezien het belang van een goed functionerend supportproces, zowel voor de eindgebruiker als voor het team dat de IT-vragen afhandelt, is het noodzakelijk om dit proces zo efficiënt mogelijk te organiseren. In deze specifieke use case betekent dit dat ontwikkelaars van het MyMinfin-team potentieel minder tijd hoeven te besteden aan support en meer tijd kunnen vrijmaken voor de ontwikkeling van de website.
\\[1em]
De centrale onderzoeksvraag luidt: “Welke methode, Retrieval Augmented Generation (RAG), Finetuning of Cache-Augmented Generation (CAG), is het meest geschikt voor een LLM-gebaseerde IT-support chatbot binnen de context van de FOD Financiën?” Om deze vraag te beantwoorden werd eerst een literatuurstudie uitgevoerd waarin de voor- en nadelen van de drie benaderingen werden geanalyseerd. Uit deze analyse bleek RAG de meest geschikte methode voor deze specifieke usecase.
\\[1em]
Op basis van deze keuze werd een Proof of Concept (PoC) ontwikkeld die RAG in de praktijk implementeert. Aangezien de PoC gebruik maakt van een LLM-model werd aan deze PoC een vergelijkende studie gekoppeld. Op die manier kon worden bepaald welk LLM-model het best presteert voor deze specifieke toepassing. Gezien de scope van dit onderzoek en de functionele vereisten waaraan de PoC moest voldoen, konden niet alle modellen worden opgenomen in de vergelijkende studie. Rekening houdend met deze beperking zijn in de vergelijkende studie de modellen Llama3.1, Llama3.2, Qwen2.5 en Qwen3 getest.
\\[1em]
De resultaten van het onderzoek tonen aan dat het Qwen3-model het beste heeft gepresteerd. In alle drie de uitgevoerde testfases behoorde dit model telkens bij de best presterende modellen. De ontwikkelde PoC biedt het MyMinfin-team een concrete basis om LLM-gebaseerde ondersteuning in het IT-supportproces te verkennen en verder te optimaliseren. Hiermee kan het team ook andere, mogelijk meer performante modellen evalueren en bepalen welke modellen het meest geschikt zijn voor het bieden van deze ondersteuning. Op deze manier draagt deze bachelorproef bij aan een efficiënter en meer betrouwbaar IT-supportproces binnen het MyMinfin-team.

%---------- Inhoud, lijst figuren, ... -----------------------------------------

\tableofcontents

% In a list of figures, the complete caption will be included. To prevent this,
% ALWAYS add a short description in the caption!
%
%  \caption[short description]{elaborate description}
%
% If you do, only the short description will be used in the list of figures

\listoffigures

% If you included tables and/or source code listings, uncomment the appropriate
% lines.
\listoftables

\listoflistings

% Als je een lijst van afkortingen of termen wil toevoegen, dan hoort die
% hier thuis. Gebruik bijvoorbeeld de ``glossaries'' package.
% https://www.overleaf.com/learn/latex/Glossaries

%---------- Kern ---------------------------------------------------------------

\mainmatter{}

% De eerste hoofdstukken van een bachelorproef zijn meestal een inleiding op
% het onderwerp, literatuurstudie en verantwoording methodologie.
% Aarzel niet om een meer beschrijvende titel aan deze hoofdstukken te geven of
% om bijvoorbeeld de inleiding en/of stand van zaken over meerdere hoofdstukken
% te verspreiden!

%%=============================================================================
%% Inleiding
%%=============================================================================

\chapter{\IfLanguageName{dutch}{Inleiding}{Introduction}}%
\label{ch:inleiding}

Met de opkomst van artificiële intelligentie (AI) en Large Language Models (LLM’s) ontstaan er steeds meer mogelijkheden om deze technologieën in te zetten voor intelligente IT-support chatbots. Traditionele IT-supportsystemen vereisen vaak veel menselijke tussenkomsten, wat kan leiden tot langere wachttijden en hogere kosten. Een LLM-gebaseerde chatbot kan hier een oplossing bieden door het supportteam bij te staan om vragen snel en accuraat te beantwoorden, waardoor de efficiëntie van IT-supportteams toeneemt.

\section{\IfLanguageName{dutch}{Probleemstelling}{Problem Statement}}%
\label{sec:probleemstelling}

Deze bachelorproef richt zich specifiek op de IT-support binnen de FOD Financiën. Deze organisatie biedt een uitgebreide en complexe vorm van support aan zowel interne medewerkers als externe klanten over een breed scala aan onderwerpen. Een deel hiervan is technische ondersteuning voor digitale diensten die de organisatie aanbiedt. Door de omvang en diversiteit van de vragen waarmee de FOD Financiën wordt geconfronteerd, is er een duidelijke behoefte aan een efficiëntere ondersteuning. Bijvoorbeeld is iedere service verantwoordelijk om technische support te bieden, het is daarbij echter soms moeilijk om te achterhalen wie wanneer en voor wat exact verantwoordelijk is. 

Momenteel is de IT-support versnipperd, wat deels onvermijdelijk is omdat niet iedereen over de nodige technische kennis beschikt. De gespecialiseerde teams zijn verantwoordelijk voor de ondersteuning binnen hun eigen domeinen, maar dit kan leiden tot onduidelijkheid over wie wanneer en waarvoor exact verantwoordelijk is. Het is niet altijd meteen duidelijk of een probleem onder de verantwoordelijkheid van het eigen team valt of dat het door een ander technisch team moet worden opgepakt. Zelfs wanneer vaststaat dat een ander team het probleem moet oplossen, is het vaak onduidelijk hoe zij gecontacteerd kunnen worden of welk e-mailadres hiervoor gebruikt moet worden.

Bovendien bestaat er wel documentatie over deze processen, maar deze is verspreid over verschillende Confluence pagina’s, waardoor medewerkers moeite hebben om snel de juiste informatie te vinden. Dit gebrek aan centralisatie zorgt ervoor dat IT-medewerkers soms werken met zeer beperkte informatie, wat het oplossen van problemen bemoeilijkt. In de praktijk leidt dit vaak tot een inefficiënte heen-en-weerstroom van e-mails tussen teams, zonder dat dit altijd bijdraagt aan een daadwerkelijke oplossing van het probleem.

Om deze problemen te verhelpen wordt onderzocht hoe een LLM-gebaseerde chatbot dit proces kan ondersteunen. Hiervoor worden twee belangrijke benaderingen vergeleken: Retrieval Augmented Generation (RAG) en Cache-Augmented Generation (CAG). De keuze tussen deze methoden heeft een aanzienlijke impact op de kwaliteit en relevantie van de gegenereerde antwoorden. Daarom zal in deze bachelorproef een Proof of Concept (PoC) worden ontwikkeld en getest binnen de context van de FOD Financiën, met als doel de meest geschikte aanpak te identificeren. Hoewel deze studie zich richt op deze specifieke usecase, kunnen de bevindingen en methodieken ook relevant zijn voor andere organisaties die hun IT-support op een efficiëntere manier willen organiseren.

\section{\IfLanguageName{dutch}{Onderzoeksvraag}{Research question}}%
\label{sec:onderzoeksvraag}

De centrale onderzoeksvraag van deze bachelorproef luidt:

“{Welke methode, Retrieval Augmented Generation (RAG) of Cache-Augmented Generation (CAG), is het meest geschikt voor een LLM-gebaseerde IT-support chatbot binnen de context van de FOD Financiën?}”

Om deze onderzoeksvraag te beantwoorden, worden de volgende deelvragen geformuleerd:
%TODO Denk wat meer na over de verschillende deelvragen nu zijn ze nog te breed maak ze concreter

\begin{itemize}
    \item Wat is RAG en CAG?
    \item Wat zijn de belangrijkste verschillen tussen RAG en CAG?
    \item Welke LLM-modellen zijn het meest geschikt om te gebruiken voor een implementatie?
    \item Welke methode is het meest geschikt voor een IT-support chatbot binnen de FOD Financiën op basis van relevante criteria (bv. nauwkeurigheid, snelheid, gebruiksvriendelijkheid, kosten)?
    \item Hoe presteren verschillende implementaties van de gekozen methode in een Proof of Concept (PoC)?
\end{itemize}

\section{\IfLanguageName{dutch}{Onderzoeksdoelstelling}{Research objective}}%
\label{sec:onderzoeksdoelstelling}

Het doel van deze bachelorproef is tweeledig. Enerzijds wordt een Proof of Concept (PoC) ontwikkeld die actief gebruikt kan worden om de IT-support binnen de FOD Financiën efficiënter te organiseren. Hiervoor worden verschillende versies van de PoC geïmplementeerd en met elkaar vergeleken om de meest geschikte aanpak te bepalen.

Anderzijds biedt deze studie een theoretisch overzicht van de verschillende methoden die beschikbaar zijn voor het ontwikkelen van een LLM-gebaseerde IT-support chatbot. Dit omvat een vergelijking tussen RAG en CAG, waarbij hun respectievelijke voor- en nadelen worden geanalyseerd.

De combinatie van deze praktische en theoretische analyse moet leiden tot een onderbouwde aanbeveling voor de implementatie van een IT-support chatbot binnen grote organisaties zoals de FOD Financiën.

\section{\IfLanguageName{dutch}{Opzet van deze bachelorproef}{Structure of this bachelor thesis}}%
\label{sec:opzet-bachelorproef}

De rest van deze bachelorproef is als volgt opgebouwd:

In Hoofdstuk~\ref{ch:stand-van-zaken} wordt een overzicht gegeven van de stand van zaken binnen het onderzoeksdomein, op basis van een literatuurstudie.

In Hoofdstuk~\ref{ch:methodologie} wordt de methodologie toegelicht en worden de gebruikte onderzoekstechnieken besproken om een antwoord te kunnen formuleren op de onderzoeksvragen.

In Hoofdstuk~\ref{ch:resultaten} worden de bevindingen van het onderzoek gepresenteerd. De verzamelde gegevens worden geanalyseerd en overzichtelijk weergegeven.

In Hoofdstuk~\ref{ch:discussie} worden de resultaten geïnterpreteerd. Daarnaast komen de sterktes en beperkingen van het onderzoek aan bod en wordt een vergelijking gemaakt met bestaande literatuur.

In Hoofdstuk~\ref{ch:aanbevelingen} worden mogelijke implicaties van de onderzoeksresultaten besproken en worden aanbevelingen geformuleerd voor de praktijk en/of toekomstig onderzoek.

In Hoofdstuk~\ref{ch:conclusie}, tenslotte, wordt de conclusie gegeven en een antwoord geformuleerd op de onderzoeksvragen.
\chapter{\IfLanguageName{dutch}{Stand van zaken}{State of the art}}
\label{ch:stand-van-zaken}

% Tip: Begin elk hoofdstuk met een paragraaf inleiding die beschrijft hoe
% dit hoofdstuk past binnen het geheel van de bachelorproef. Geef in het
% bijzonder aan wat de link is met het vorige en volgende hoofdstuk.

% Pas na deze inleidende paragraaf komt de eerste sectiehoofding.
Dit overzicht biedt inzicht in de huidige stand van zaken en vormt de basis voor een weloverwogen keuze bij de ontwikkeling van een Proof of Concept (PoC). Dit betreft zowel de technische als de juridische aspecten. Daarnaast wordt in dit deel ook stilgestaan bij het belang van een vlot supportproces binnen IT.

Concreet komen de volgende onderwerpen aan bod:
\begin{itemize}
    \item \textbf{IT-support chatbot} - Wat zijn de mogelijkheden van om een IT-support chatbot te maken aan de hand van een {Large Language Models} (LLM) en welke factoren moeten in overweging worden genomen?
    \item \textbf{De AI Act en de belangrijkste richtlijnen} - Een overzicht van de relevante wet- en regelgeving en de impact hiervan op RAG-toepassingen.
    \item \textbf{Best practices voor IT-supportprocessen} - Inzichten in hoe RAG kan worden toegepast binnen IT-support en welke methodologieën het best kunnen worden gevolgd.
\end{itemize}

\section{IT-support chatbot}
    
    Om een IT-support chatbot te ontwikkelen, moeten verschillende aspecten in overweging worden genomen. Bij het gebruik van een LLM-model zijn er twee belangrijke opties: Retrieval-Augmented Generation (RAG) en Cache-Augmented Generation (CAG). Beide methoden maken gebruik van een LLM en verrijken de kennis van dit model met behulp van eigen documenten. Echter, beide benaderingen hebben specifieke voor- en nadelen. Het is daarom essentieel om deze zorgvuldig af te wegen om een weloverwogen keuze te maken voor een Proof of Concept (PoC).
    
    \subsection{Large Language Model}
    
    %TODO hier moet je veel uitgebreider over praten behandel ook onderwerpen zoals NLP en LM om dan naar dit punt te komen. Zal vollediger zijn en je kan op die manier ook veel meer tekst hebben
    
    WORK IN PROGRESS
    
     Een LLM is een geavanceerd neuraal netwerk dat getraind is op grote hoeveelheden tekstdata om menselijke taal te kunnen begrijpen, genereren en manipuleren. LLM's gebruiken de transformer architectuur, zoals geïntroduceerd door \textcite{Vaswani2017}, die uitblinkt in het modelleren van sequentiële tekst en contextuele afhankelijkheden.
      
     Deze modellen worden "large" genoemd vanwege hun schaal: ze bevatten vaak miljarden parameters. Getraind door op gigantische bronnen aan data, kunnen LLM's tekst genereren die coherente, contextuele en soms zelfs creatieve eigenschappen vertoont \cite{Gupta2025}.
      
     Toepassingen van LLM's omvatten onder andere chatbots, automatische vertaling, samenvatten van documenten, codegeneratie, en vragen-beantwoording. Hoewel hun prestaties indrukwekkend zijn, bestaan er ook zorgen over bias, hallucinatie van onjuiste informatie, en het gebruik van niet-gecontroleerde trainingsdata \cite{Gupta2025}.
      
     De ontwikkeling van LLM’s markeert een belangrijke stap richting algemene taalintelligentie, maar roept tegelijkertijd vragen op over ethiek, transparantie, en betrouwbaarheid in gebruik \cite{Gupta2025}.
     
    \subsection{Retrieval Augmented Generation}
    
    LLM's hebben de afgelopen jaren een enorme opmars gemaakt en vandaag de dag hebben deze modellen een brede impact op verschillende domeinen in de samenleving. Ondanks hun indrukwekkende mogelijkheden brengen LLM’s ook enkele nadelen met zich mee. Zo kunnen ze hallucineren, beschikken ze niet altijd over de meest actuele informatie, en missen ze vaak domein- en bedrijfsspecifieke kennis.  
    
    Een mogelijke oplossing voor deze beperkingen is \textit{Retrieval-Augmented Generation} (RAG). Deze techniek combineert de kracht van LLM’s met externe databronnen om nauwkeurigere en beter onderbouwde antwoorden te genereren. In deze sectie wordt toegelicht wat RAG is, hoe het werkt en op welke manier het kan bijdragen aan de ontwikkeling van een effectieve support chatbot.
    
    \subsubsection{Wat is RAG}
    RAG, oftewel Retrieval-Augmented Generation, is een techniek die, zoals eerder vermeld, een oplossing biedt voor de tekortkomingen van klassieke LLM’s. Door externe databronnen te gebruiken, kan een traditioneel LLM-model betere resultaten behalen. Deze methode maakt het mogelijk om domeinspecifieke data te integreren en modellen bij te werken met actuele informatie. Zo kunnen klassieke LLM’s worden verrijkt met nieuwe, up-to-date gegevens die voldoen aan specifieke behoeften \autocite{Wu2024}.
    
    Om RAG in de praktijk toe te passen, moeten een aantal stappen worden doorlopen. Deze worden in het volgende deel nader toegelicht, maar samengevat bestaat het proces uit de volgende fasen:
    
    \begin{enumerate}
        \item {Ophalen}
        \item {Verrijking}
        \item {Generatie}
    \end{enumerate}
    
    \subsubsection{Hoe werkt RAG}
    
    Het doel van RAG is om een LLM te verrijken met specifieke kennis, zodat de gebruiker beter ondersteund wordt bij het beantwoorden van gerichte vragen. Dit proces bestaat uit drie hoofdfasen. Eerst wordt relevante informatie opgehaald (Retrieval). Vervolgens wordt het antwoord verrijkt op basis van de beschikbare documentatie. Tot slot genereert de LLM een antwoord op de gestelde vraag.
    
    Op Figuur \ref{fig:Rag process} is te zien die dit proces illustreert.
    
    \begin{figure}[H]
        \centering
        \includegraphics[width=\textwidth]{RagGao2023.png}
        \caption{Een generieke RAG architectuur \cite{Gao2023}}
        \label{fig:Rag process}
    \end{figure}
    
    \paragraph{Ophalen}
     
    Voordat een LLM effectief ondervraagd kan worden met informatie die via documentverrijking is toegevoegd, moet eerst een grondige voorbereidende verwerking plaatsvinden. Dit proces omvat het selecteren van relevante documenten en het vervolgens combineren van deze informatie met de gestelde vraag. Deze vraag en de geselecteerde documentfragmenten, die worden opgehaald vanuit de vector database, worden toegevoegd aan de context van de de LLM die vervolgens in staat is om een contextueel passend antwoord te genereren.
    
    Een eerste essentiële stap in dit proces is het identificeren en selecteren van de documenten die inhoudelijk relevant zijn voor de vraagstelling. Deze documenten worden niet in hun geheel, maar in kleinere segmenten verwerkt, zogenaamde "chunks". Het opdelen in chunks is noodzakelijk omdat volledige documenten doorgaans te omvangrijk zijn om efficiënt te verwerken binnen de contextlimieten van een LLM. Bovendien maakt deze fragmentatie het mogelijk om op een meer gerichte manier informatie te extraheren uit de vector database \autocite{Wu2024}.
    
    Voor het opdelen van documenten bestaan verschillende methodologische benaderingen. Een veelgebruikte techniek is het verdelen van tekst op basis van een vooraf bepaald aantal tokens of karakters, zodat elke chunk ongeveer dezelfde lengte heeft. Een alternatieve strategie, die vooral geschikt is voor natuurlijke taalteksten, is het opdelen op basis van zinnen of alinea’s. Deze aanpak draagt doorgaans bij aan het behoud van de semantische samenhang binnen een chunk, wat de kwaliteit van de informatieophaling ten goede komt \autocite{Wang2024}.
    
    Na het opdelen worden deze tekstsegmenten omgezet in zogeheten embeddings. Dit zijn vector representaties die de semantische inhoud van de chunks op een wiskundige manier vastleggen. Deze embeddings worden vervolgens opgeslagen in een vector database, een specifiek type gegevensopslag dat geoptimaliseerd is voor het bewaren en efficiënt ophalen van vectorrepresentaties op basis van semantische gelijkenis \autocite{Wu2024}. Dit gehele proces vormt de basis waarop de LLM tijdens het beantwoorden van vragen relevante context uit de vector database kan ophalen en verwerken.
   
     \begin{figure}[H]
        \centering
        \includegraphics[width=\textwidth]{retrieverWu2024.png}
        \caption{Documentverwerking en opslag in een vector database via embeddings \cite{Wu2024}}
        \label{fig:RAG opmaken vector database}
    \end{figure}
    
    Zodra de documenten via embeddings zijn toegevoegd aan de vector database, zal een gebruiker in de context van RAG (Retrieval-Augmented Generation) ook vragen stellen. Een vraag of query wordt, net als de documenten, vertaald naar embeddings. Op basis van deze embeddings worden de top-k dichtstbijzijnde buren uit de vector database opgehaald. Dit betekent dat de meest relevante delen van de opgeslagen documenten worden opgehaald. Deze relevante “chunks” worden vervolgens gebruikt om context te bieden aan het LLM-model dat wordt ondervraagd \autocite{Wu2024}.
    
            
    \begin{figure}[H]
        \centering
        \includegraphics[width=\textwidth]{QueryRetrieverWu2024.png}
        \caption{Vraagafhandeling en ophalen van relevante chunks \cite{Wu2024}}
        \label{fig:RAG bevragen vector database}
    \end{figure}
    
    \paragraph{Generatie}
    
    Zodra de meest relevante informatie is opgehaald, wordt deze gebruikt in het proces. Uiteindelijk blijft het de bedoeling om het LLM-model te bevragen. De opgehaalde documenten worden, samen met de vraag van de gebruiker, toegevoegd aan de context van het model. Hierdoor beschikt het LLM-model over extra informatie en kan het deze verwerken. Dit leidt tot een antwoord voor de gebruiker, gebaseerd op de geïnjecteerde kennis \autocite{Zhao2024}.
    
    \subsection{Cache-Augmented Generation}
    
    Met de opkomst van nieuwe LLM modellen die een grotere context bevatten is het niet altijd nodig om te werken met een RAG architectuur. Wanneer het aantal documenten en de lengte ervan niet dermate groot is kan je ook meteen alle documenten injecteren in de context van een LLM model. Dit zorgt voor een heel simpele en efficiënte benadering om een LLM model bedrijf- of contextspecifieke kennis te geven.
    
    \subsection{RAG vs CAG}
    
    
    \subsection{LLM benchmarks}
    Om te bepalen welke modellen het meest geschikt zijn voor de ontwikkeling van IT-support chatbot, is een objectieve meetmethode noodzakelijk. Gelukkig bestaan er verschillende platformen die LLM's vergelijken en rangschikken op basis van prestaties. In deze sectie bespreken we enkele van deze platformen en maken we een selectie van modellen die het meest geschikt lijken voor het bouwen van een RAG-model.
       
    \paragraph{LiveBench} 
    Een platform dat LLM-modellen evalueert, is LiveBench. Dit platform stelt een rangschikking op voor verschillende modellen en biedt een actuele scorebord die elke zes maanden wordt bijgewerkt. Voor deze bachelorproef zal gebruik worden gemaakt van de ranking afkomstig uit november 2024.
    
    LiveBench beoordeelt LLM-modellen op basis van zes categorieën. Binnen elke categorie worden meerdere taken uitgevoerd om een nauwkeurige beoordeling te verkrijgen. De zes categorieën zijn:
    \begin{itemize}
        \item Wiskundige vaardigheden (Math)
        \item Programmeervaardigheden (Coding)
        \item Redeneren en probleemoplossing (Reasoning)
        \item Data-analyse (Data Analysis)
        \item Volgen van instructies (Instruction Following)
        \item Begrip van natuurlijke taal (Language Comprehension)
    \end{itemize}
    
    Elke categorie wordt geëvalueerd op basis van specifieke taken. Dit resulteert uiteindelijk in een algemene rangschikking, waarin zowel de beste modellen per categorie als het beste presterende model overall worden geïdentificeerd.
    
    \begin{figure}[H]
        \centering
        \includegraphics[width=\textwidth]{LiveBenchRanking.png}
        \caption{LiveBench ranking van verschillende LLMs.}
        \label{fig:livebench}
    \end{figure}
    
    Uit de ranking van LiveBench kan geconcludeerd worden dat 3 verschillende organisaties elk een model aanbieden die vanuit globaal oogpunt tot de top 3 behoort. Deze top 3 zijn: 
    \begin{enumerate}
        \item claude-3-5-sonnet-20240620 van Anthropic
        \item Meta-llama-3.1-405b-instruct-turbo van Meta
        \item gpt-4o-2024-05-13 van OpenAI
    \end{enumerate}
    
    \subsubsection{Chatbot arena} 
    
    Een andere benchmark tool Chatbot arena, net zoals LiveBench is Chatbot arena een site die een actuele weergave biedt van de beste LLM's op basis van vooraf gedefinieerde categorieën. 
    Volgens de ranking van \textcite{LiveBench2025} zijn dit de top 3 LLM's:
    
    \begin{enumerate}
        \item GPT-4-Turbo van OpenAI
        \item GPT-4-0613 van OpenAI
        \item Mistral-Medium Mistral AI
    \end{enumerate}
    
    Hoewel het doel van beide benchmarks hetzelfde is is de manier van werken wel anders. ChatbotArena gaat gebruikers twee anonieme LLM modellen tegenover elkaar plaatsen, de gebruiker kan vervolgens een eigen vraag stellen en de gebruiker bepaalt zelf welke van de 2 het beste resultaat heeft opgeleverd.
    
    Het voordeel van deze methode is dat de LLM-modellen realistische cases moeten behandelen die door gebruikers zelf worden gesteld. Op basis van resultaten die de LLM modellen tonen kan een gebruiker zijn voorkeur meegeven. Het nadeel van deze manier van werken is dat de gebruikers die deze testen uitvoeren niet representatief zijn voor alle gebruikers van LLM-modellen. De gebruikers die deze testen uitvoeren zijn vaak mensen met een interesse in LLM-modellen of mensen die onderzoek doen in dit vakgebied. Desondanks kan op basis van deze stemmen verscheidene modellen tegenover elkaar worden geplaatst. In januari 2024 werden ruim 240.000 stemmen uitgebracht door ongeveer 90.000 gebruikers \autocite{Chiang2024}. 
    
    \paragraph{Conclusie}
    
    %TODO Navragen of ze een lijst van criteria hebben welke LLM wel en niet te gebruiken => geen echte beperkingen inzake LLM maar zou best moeten kunnen draaien op Azure opgeving dus chatGPT zou hierbij zeker de voorkeur hebben.
    Op basis van de 2 benchmarks die hier werden besproken kan niet meteen éénduidig besloten worden welke modellen het best zouden gebruikt worden voor het maken van het RAG model. Aangezien dit een zeer volatiele omgeving is met veel en snelle ontwikkelingen zijn de modellen die vandaag het beste scoren over een maand potentieel voorbij gestoken door nieuwe modellen. Desalniettemin bevatten deze benchmarks heel wat nuttige informatie en inzichten over de sterktes van bepaalde modellen tegenover andere modellen. 
    
    \subsection{LLM modellen evalueren}
    
    Het is uiteraard niet enkel noodzakelijk om een performante LLM te gaan gebruiken. Het is eveneens van groot belang om na te gaan indien de antwoorden die worden gegenereerd ook accuraat zijn. Hiertoe zijn verschillende mogelijkheden. Zo heb je de Recall-Oriented Understudy for Gisting Evaluation (ROUGE), Bilingual Evaluation Understudy (BLUE) en BERTScore, deze gaan allemaal op een bepaalde manier een score geven aan een antwoord die een LLM genereert.
    
    \paragraph{BLUE}
    
    Bron te gebruiken voor dit deel \textcite{papineni-etal-2002-bleu}
    
    \paragraph{ROUGE}
    
    bron te gebruiken voor dit deel \textcite{Ganesan2018}
    
    \paragraph{BERTScore}

\section{De AI Act en de belangrijkste richtlijnen}
Wat houdt de AI Act in, en wat zijn de belangrijkste richtlijnen die hierin moeten worden gevolgd?

\section{Best practices voor IT-supportprocessen}
Wat zijn de bestaande best practices voor de organisatie van IT-supportprocessen?

%%=============================================================================
%% Methodologie
%%=============================================================================

\chapter{Methodologie}
\label{ch:methodologie}

%% TODO: In dit hoofstuk geef je een korte toelichting over hoe je te werk bent
%% gegaan. Verdeel je onderzoek in grote fasen, en licht in elke fase toe wat
%% de doelstelling was, welke deliverables daar uit gekomen zijn, en welke
%% onderzoeksmethoden je daarbij toegepast hebt. Verantwoord waarom je
%% op deze manier te werk gegaan bent.
%% 
%% Voorbeelden van zulke fasen zijn: literatuurstudie, opstellen van een
%% requirements-analyse, opstellen long-list (bij vergelijkende studie),
%% selectie van geschikte tools (bij vergelijkende studie, "short-list"),
%% opzetten testopstelling/PoC, uitvoeren testen en verzamelen
%% van resultaten, analyse van resultaten, ...
%%
%% !!!!! LET OP !!!!!
%%
%% Het is uitdrukkelijk NIET de bedoeling dat je het grootste deel van de corpus
%% van je bachelorproef in dit hoofstuk verwerkt! Dit hoofdstuk is eerder een
%% kort overzicht van je plan van aanpak.
%%
%% Maak voor elke fase (behalve het literatuuronderzoek) een NIEUW HOOFDSTUK aan
%% en geef het een gepaste titel.

\section{Literatuurstudie}

Voor de opstart van deze bachelorproef is een grondige literatuurstudie essentieel om de bestaande mogelijkheden op het gebied van virtuele assistenten te verkennen, met een focus op finetuning, RAG en CAG. Allereerst wordt een inleiding gegeven in de onderliggende materie, zodat het onderwerp voldoende gekaderd wordt. Daarnaast worden de voor- en nadelen van de verschillende opties in kaart gebracht, zodat op basis van deze inzichten een onderbouwde keuze kan worden gemaakt voor de uitwerking van een PoC.
\\[1em]
Op basis van de literatuurstudie wordt een analyse opgesteld om te bepalen welke optie het meest geschikt is voor verdere uitwerking. De belangrijkste beslissing binnen deze use case betreft de keuze van de techniek die zal worden toegepast. Elke methode heeft zijn eigen voor- en nadelen, die zorgvuldig moeten worden afgewogen bij het opstellen van de PoC en bij de selectie van de LLM-modellen die kunnen worden ingezet.
\\[1em]
Daarnaast worden de beschikbare benchmarks besproken en wordt toegelicht hoe deze geïnterpreteerd moeten worden. Dit is van belang om op een onderbouwde manier een model te kiezen voor de PoC. Ook komen verschillende evaluatiemetrieken aan bod die nuttig zijn voor het beoordelen van de PoC.

\section{Requirements analyse}

Gekoppeld aan de literatuurstudie zal een interview worden afgenomen met één van de uiteindelijke gebruikers van de PoC. Het doel van dit interview is om een helder overzicht te verkrijgen van de verschillende vereisten voor de virtuele assistent. Op basis van het overzicht uit de literatuurstudie kan vervolgens een meer gerichte selectie worden gemaakt van de verschillende opties. Het uiteindelijke doel is om voor drie à vier modellen een PoC uit te werken die verder getest kunnen worden. Het is met andere woorden van belang om de verschillende modellen zo uitgebreid mogelijk te onderzoeken, zodat een zo breed mogelijke set modellen getoetst kunnen worden aan de gevraagde vereisten. 
\\[1em]
Eens de literatuurstudie is afgerond en de interviews zijn afgenomen, kan met behulp van een MoSCoW-analyse een overzicht gemaakt worden van de nodige functionaliteiten die nodig zijn om de PoC op te stellen. Aan de hand van deze functionaliteiten kan een rangschikking worden opgesteld van de verschillende beschikbare modellen waarmee de PoC kan verwezenlijkt worden. Deze lijst bepaalt welke modellen worden geselecteerd voor het uitwerken van de PoC.

\section{Long list}

In deze fase wordt een overzicht opgesteld van de verschillende opties voor het opzetten van de PoC. Dit omvat zowel mogelijke technieken als categorieën van LLM’s. Het doel is om een zo breed mogelijk beeld te schetsen, zodat alle relevante benaderingen kunnen worden overwogen. In deze fase worden nog geen keuzes gemaakt. De uiteindelijke selectie volgt in een latere stap op basis van de resultaten van de behoefteanalyse. Daarbij wordt rekening gehouden met de beperkingen die tijdens het onderzoek naar voren komen.
\\[1em]
Op basis van de beschikbare modellen wordt een selectie gemaakt, waarbij uitsluitend open modellen in aanmerking komen. Deze zijn niet alleen kosteloos te gebruiken, maar kunnen bovendien lokaal worden geïnstalleerd en ingezet. Doordat de verwerking volledig op de eigen hardware gebeurt, blijven alle data en query’s binnen de PoC strikt lokaal en ontstaat er geen risico op privacyproblemen.

\section{Short list}

Op basis van de beschikbare modellen uit de longlist wordt een selectie gemaakt. Deze selectie hangt voornamelijk af van de structuur van de PoC. Als het model aan specifieke vereisten moet voldoen, wordt de lijst verder ingeperkt. Hoe dit precies gebeurt, wordt toegelicht in het volgende hoofdstuk.
\\[1em]
Omdat gekozen werd voor open modellen die lokaal op de eigen hardware draaien, beperkt de keuze zich bovendien tot modellen van een bepaalde omvang. De meest krachtige open modellen vallen daardoor eveneens buiten de selectie.
\\[1em]
De uiteindelijke keuze tussen Finetuning, RAG of CAG hangt af van een voorafgaande analyse, die in het volgende hoofdstuk wordt besproken. Samen met de scope en de beperkingen van het onderzoek bepaalt deze analyse in hoofdzaak welke techniek wordt toegepast.

\section{Opbouw van de PoC en Evaluatie}

De PoC bestaat uit het ontwikkelen van een chatbot die het MyMinfin-IT team in staat stelt om supportvragen op een efficiënte manier op te lossen. Aan de basis van deze toepassing ligt een LLM. De PoC wordt zodanig ontwikkeld dat verschillende modellen afwisselend kunnen worden ingezet. Dit zal in een latere fase de kern vormen van de vergelijkende studie.
\\[1em]
Tijdens de opbouw van de PoC worden periodiek tests uitgevoerd om te controleren of de werking naar behoren verloopt. Hierdoor is het mogelijk om tijdig bij te sturen en eventuele onvoorziene problemen aan te pakken. De focus van deze evaluaties ligt op het correct functioneren van de PoC en het voorkomen van regressie bij verdere ontwikkeling.

\section{Algemene Evaluatie}

Aan het einde van de PoC wordt elk geselecteerd model getest met een identieke set vragen. Op deze manier kunnen de modellen onderling vergeleken worden. Tijdens deze algemene evaluatie worden de volgende aspecten onderzocht:

\begin{itemize}
    \item De kwaliteit van de gegenereerde antwoorden, beoordeeld volgens wetenschappelijk onderbouwde evaluatiecriteria. Hierbij wordt gebruik gemaakt van het Ragas test framework.
    \item De prestaties van het model bij het beantwoorden van triviale vragen, waarvoor geen domeinspecifieke kennis vereist is.
    \item Het al dan niet vertonen van hallucinaties, geeft het model foutieve of verzonnen informatie?
\end{itemize}

Tijdens de evaluatie wordt voor elk aspect gebruikgemaakt van vooraf gedefinieerde meetinstrumenten en testscenario’s om de modellen onder identieke omstandigheden te vergelijken.

Deze drie aspecten bieden elk inzicht in een specifiek onderdeel van het functioneren van de PoC. Hierdoor kan ieder model worden beoordeeld op zijn sterke en zwakke punten, zodat aan het einde van de testfase een weloverwogen keuze kan worden gemaakt voor het model dat het beste heeft gepresteerd.



% Voeg hier je eigen hoofdstukken toe die de ``corpus'' van je bachelorproef
% vormen. De structuur en titels hangen af van je eigen onderzoek. Je kan bv.
% elke fase in je onderzoek in een apart hoofdstuk bespreken.

%%=============================================================================
%% Proof of concept
%%=============================================================================
\chapter{Proof of concept}
\label{ch:proof-of-concept}

%Wat wil je bewijzen of aantonen met deze PoC?
De hoofddoelstelling van deze PoC is te onderzoeken hoe een LLM ondersteuning kan bieden binnen een IT-supportsysteem, met als doel technische IT-vragen sneller af te handelen. De toepassing is bedoeld voor intern gebruik door het MyMinfin IT-team en niet voor externe gebruikers, zoals burgers. In de stand van zaken werden verschillende benaderingen onderzocht. In dit hoofdstuk wordt een keuze gemaakt uit deze benaderingen en toegepast op de PoC.
\\[1em]
%Welke hypothese test je eigenlijk?
De hypothese die binnen deze PoC getest wordt, kan als volgt worden samengevat:
\textit{“Kan een LLM effectief en efficiënt ondersteuning bieden binnen een IT-supportsysteem door relevante informatie uit documentatie te halen en daarmee het supportproces te optimaliseren en te vereenvoudigen.”}

\section{Behoeften analyse}
Voor de PoC werd een interview afgelegd. Op basis van dit interview werd een MoSCoW-analyse uitgevoerd om de functionele vereisten te prioritiseren. De analyse werd onderverdeeld in \textit{Must haves}, \textit{Should haves} en \textit{Could haves}

\subsection{Must haves}
\begin{itemize}
    \item \textbf{Vragen kunnen stellen in de eigen taal (bijvoorbeeld Nederlands, Frans of Engels)}:\\ 
    Het systeem moet correct functioneren ongeacht of de gebruiker vragen stelt in het Nederlands, Frans of Engels.
    \item \textbf{Een duidelijk en bruikbaar antwoord ontvangen op basis van betrouwbare bronnen}:\\ 
    De gegenereerde antwoorden moeten informatief, relevant en praktisch toepasbaar zijn.
    \item \textbf{Eenvoudige en intuïtieve interface}:\\  
    De gebruiker mag geen technische kennis nodig hebben. De interactie moet vanzelfsprekend en gebruiksvriendelijk aanvoelen.
    \item \textbf{Consistente ervaring (snel, zonder fouten of willekeurige antwoorden)}:\\  
    De gebruiker verwacht dat de applicatie betrouwbaar functioneert en consistente resultaten levert.
\end{itemize}

\subsection{Should haves}
\begin{itemize}
    \item \textbf{Het systeem begrijpt vervolgvragen binnen een sessie}:\\  
    De gebruiker moet een gesprek kunnen voeren waarbij eerdere vragen worden meegenomen in de context, zodat interacties natuurlijker verlopen.
    \item \textbf{Transparantie over de gebruikte bronnen van het antwoord}:\\  
    De gebruiker moet kunnen zien uit welk document of welke bron het antwoord afkomstig is, zodat de informatie steeds gecontroleerd en gevalideerd kan worden.
\end{itemize}

\subsection{Could haves}
\begin{itemize}
    \item \textbf{Langetermijngeheugen over meerdere sessies heen}:\\  
    Het opzetten van een geheugen dat sessie-overstijgend werkt, brengt extra complexiteit met zich mee en is daarom minder geschikt voor een eerste versie.
    \item \textbf{Zelf documenten kunnen uploaden of bewerken}:\\  
    In deze PoC ligt de focus op het zoeken en beantwoorden van vragen, niet op het beheren van documenten.
    \item \textbf{Automatisch mails behandelen}:\\  
    Het automatisch verwerken en beantwoorden van e-mails valt qua complexiteit buiten de scope van een eerste versie.
\end{itemize}

\section{Architectuur en Ontwerp}

Vooraleer de structuur van de PoC kan worden opgezet, moet een keuze gemaakt worden voor de meest geschikte techniek. Fine-tuning is, gelet op de scope van de PoC, moeilijk uitvoerbaar. Om dit te realiseren zou een model getraind moeten worden op data die kan helpen bij IT-support. Hoewel er data beschikbaar is in de vorm van e-mails, is het onderhoud bij fine-tuning niet te onderschatten. Wanneer een procedure of bepaalde gegevens veranderen, moet het model immers opnieuw getraind worden. Deze keuze brengt met andere woorden een hoge initiële kost met zich mee en potentieel ook een hoge onderhoudskost, die niet aanwezig is bij RAG of CAG.
\\[1em]
Met CAG aan de andere kant zouden alle documenten worden toegevoegd aan de context. Hoewel dit technisch mogelijk is, zeker met de modellen van vandaag, heeft deze aanpak enkele nadelen. Ten eerste kan \textit{context rot} een probleem vormen. Zoals aangegeven in de stand van zaken, worden modellen minder performant wanneer te veel informatie in de context wordt geplaatst. Daarnaast wordt bij CAG niet alleen relevante, maar ook irrelevante informatie toegevoegd. Dit verhoogt niet alleen de kans op minder nauwkeurige antwoorden, maar doordat alle documenten in de context worden opgenomen, zullen de API-kosten per query hoger zijn dan bij RAG.
\\[1em]
Met al deze redenen in gedachten is voor deze PoC gekozen om RAG toe te passen. Enerzijds zijn er geen hoge initiële kosten zoals bij fine-tuning en anderzijds worden enkel relevante documenten toegevoegd aan de context. Dit helpt om mogelijke problemen zoals context rot te vermijden en houdt bovendien de kost per query laag. Tabel \ref{tab:techniek_vergelijking} geeft een overzicht van de voor- en nadelen van de verschillende benaderingen.

\begin{table}[H]
    \resizebox{\textwidth}{!}{
            \begin{tabular}{|p{4cm}|c|c|c|}
            \hline
            \textbf{Kenmerk} & \textbf{Fine-tuning} & \textbf{CAG} & \textbf{RAG} \\
            \hline
            Initiële kost & Hoog & Laag & Laag \\
            \hline
            Onderhoudskost & Hoog & Laag & Laag \\
            \hline
            Document selectie & n.v.t -- kennis zit in model & Alle documenten & Enkel relevante \\
            \hline
            Kost per query & Laag & Potentieel hoog  & Laag \\
            \hline
            Implementatie & Complex door training model & Relatief eenvoudig & Complex door implementatie retrieval\\
            \hline
        \end{tabular}
    }
    \caption{Vergelijking van voor- en nadelen van fine-tuning, CAG en RAG}
    \label{tab:techniek_vergelijking}
\end{table}

\subsection{Algemene structuur}

De standaard RAG-architectuur gaat een query gaan embedden en haalt vervolgens de meest relevante documenten op uit de vectordatabase. Dit gebeurt zonder verdere reflectie over de nood van het ophalen van de documenten en de werkelijke relevantie van deze documenten. Hoewel dit op het eerste gezicht een efficiënte aanpak lijkt, betekent het ook dat het systeem bij elke willekeurige gebruikersvraag opnieuw de vectordatabase bevraagt, zelfs wanneer dat niet noodzakelijk is.
\\[1em]
Om dit te vermijden, werd gebruikgemaakt van een graafstructuur die het model de mogelijkheid geeft om zelf te beoordelen wanneer het relevant is om documenten op te halen. Daarnaast wordt de LLM ook verantwoordelijk geacht om te oordelen over de opgehaalde documenten. De volledige flow van het proces kan worden bekeken in figuur~\ref{fig:Architectuur}
\\[1em]
De PoC werd uitgewerkt met behulp van Ollama, een tool waarmee open modellen op lokale hardware kunnen functioneren. Tijdens overleg met de betrokken partijen werd geen voorkeur uitgesproken voor specifieke modellen, enkel het gebruik van DeepSeek modellen werd afgeraden.

\begin{figure}[H]
    \includegraphics[width=1\textwidth]{flowchart.png}
    \caption{Architectuur van de PoC gebaseerd op een graafgebaseerde workflow. De LLM beslist binnen deze flow of de vectordatabase geraadpleegd moet worden. Bij een database-query beoordeelt de LLM vervolgens of de opgehaalde documenten relevant zijn voor de gestelde vraag. Indien de documenten niet relevant zijn, herschrijft de LLM de vraag en voert de herschreven query opnieuw uit.}
    \label{fig:Architectuur}
\end{figure}

%Belangrijke keuzes tijdens het bouwen (waarom bv. technologie X en niet Y?).

Eén van de belangrijkste beslissingen in deze PoC was de initiële keuze voor RAG ten opzichte van andere mogelijke benaderingen. Eens deze keuze gemaakt was, volgde de selectie van een geschikt framework om RAG te implementeren.
\\[1em]
Voor deze PoC werd gekozen voor LangGraph, een framework waarmee zogenaamde agents kunnen worden opgebouwd aan de hand van grafen. Dit verschilt van bijvoorbeeld LangChain, waar gewerkt wordt met ketens van opeenvolgende stappen.
\\[1em]
LangGraph biedt meer flexibiliteit om een eigen flow te implementeren. Zo kan een LLM bijvoorbeeld zelf kiezen welke paden in de graaf doorlopen worden.Het is bijvoorbeeld ook mogelijk een lus in de graafstructuur te introduceren. Beide mechanismen werden effectief toegepast in deze PoC.
\\[1em]
%Qwen3 hypoImage vraag zie je rewrite en 2de poging lukt het wel
Dit is bijzonder nuttig in deze use case omdat er voldaan moet worden aan een meertalige omgeving. Het is vooraf namelijk niet zeker in welke taal de gebruiker de toepassing zal gebruiken. Dit kan Frans, Nederlands of Engels zijn. Daarnaast is de documentatie ook in verschillende talen beschikbaar. Sommige documenten zijn voorhanden in het Frans, terwijl andere in het Engels of Nederlands beschikbaar zijn. Met andere woorden kan een vraag in het Nederlands gesteld worden terwijl de documentatie enkel in het Engels beschikbaar is. Dit kan soms tot problemen leiden bij het ophalen van de correcte documentatie. Door te werken met een graaf kan de LLM complexere paden volgen die bijvoorbeeld met LangChain niet realiseerbaar zouden zijn geweest.
\\[1em]
Daarnaast maakt deze aanpak het mogelijk om de vaardigheden van de verschillende modellen te testen. Het is immers aan de LLM zelf om op verschillende momenten in het proces keuzes en inschattingen te maken.

\subsection{Gebruikte tools en frameworks}
Tabel \ref{tab:gebruikte_technologieen} geeft een overzicht van de verschillende frameworks, tools en programmeertalen die werden gebruikt om deze PoC op te stellen.

\begin{table}[H]
    \begin{tabular}{|l|l|}
        \hline
        \textbf{Categorie}       & \textbf{Technologieën}               \\ \hline
        Frameworks/Libraries     & LangChain, LangGraph, Streamlit, Ragas \\ \hline
        Tools                   & ChromaDB, Ollama                    \\ \hline
        Programmeertalen        & Python                             \\ \hline
    \end{tabular}
    \caption{Overzicht van de gebruikte technologieën}
    \label{tab:gebruikte_technologieen}
\end{table}

\section{Implementatie}

In deze sectie wordt de concrete opbouw van de PoC toegelicht. Er wordt stap voor stap beschreven hoe de verschillende componenten zijn geïmplementeerd. De focus ligt op de praktische keuzes, zodat een volledig beeld ontstaat van de werking van de PoC.

\subsection{vectorstore}
In de eerste fase werden de originele documenten opgesplitst in kleinere tekstsegmenten en vervolgens omgezet naar embeddings via een vooraf getraind embedding model. Voor deze PoC werd hiervoor gebruikgemaakt van het \\ \verb|mxbai-embed-large model|. Dit is één van de twee beschikbare embedding modellen binnen Ollama. Dit model werd gekozen omdat het de hoogste semantische nauwkeurigheid biedt.

\begin{lstlisting}[basicstyle=\small, frame=single, breaklines=true, postbreak=\mbox{\textcolor{red}{$\hookrightarrow$}\space}, escapeinside ={\%,}, escapechar={!}, numbers=left, language=Python, caption=Ophalen van embedding model]
def get_embedding():
    return OllamaEmbeddings(model='mxbai-embed-large')
\end{lstlisting}

Aan het embedding model is ook een chatmodel gekoppeld dat doorheen het proces wordt gebruikt. De variabele \verb|response_model_name| bepaalt welk model wordt gebruikt tijdens de flow. De enige configuratie die werd aangepast, is het instellen van de temperatuur op nul. De temperatuur bepaalt hoe creatief het model zal zijn tijdens het beantwoorden van vragen. Aangezien het de bedoeling is dat het model  zich zo strikt mogelijk houdt aan de informatie uit de vectordatabase, werd ervoor gekozen om deze waarde op nul te zetten. Welke modellen zijn gekozen en waarom, wordt in een latere sectie besproken.

\begin{lstlisting}[basicstyle=\small, frame=single, breaklines=true, postbreak=\mbox{\textcolor{red}{$\hookrightarrow$}\space}, escapeinside ={\%,}, escapechar={!},
numbers=left, language=Python, caption=Initialisatie van het chat model]
def get_response_model():
    return ChatOllama(model=response_model_name, temperature=0)
\end{lstlisting}

\subsubsection{Opbouw vectordatabase}
Om de vectordatabase op te bouwen, wordt één functie gebruikt. De gehele functie is voor de volledigheid toegevoegd als bijlage en kan teruggevonden worden in bijlage \ref{functie-vectorstore}.
\\[1em]
De functie bevat verschillende parameters die nodig zijn voor het opbouwen van de vectorstore. Ten eerste is er de embeddingsfunctie die zojuist besproken is. Vervolgens de locatie waar de documenten beschikbaar zijn en de naam van de database. Tot slot worden de laatste twee parameters gebruikt om in te stellen op welke manier gezocht moet worden in de database. Dit hangt af van de gekozen ophaalmethode.

\begin{lstlisting}[basicstyle=\small, frame=single, breaklines=true, postbreak=\mbox{\textcolor{red}{$\hookrightarrow$}\space}, escapeinside ={\%,}, escapechar={!},
    numbers=left, language=Python, caption=Functie met parameters]
def build_vector_store(embeddings_function, document_path, db_name, search_type, search_kwargs):
\end{lstlisting}

Gebruik makend van onderstaande dictionary wordt een lader geselecteerd die in een latere fase wordt gebruikt om de documenten te parsen.

\begin{lstlisting}[basicstyle=\small, frame=single, breaklines=true, postbreak=\mbox{\textcolor{red}{$\hookrightarrow$}\space}, escapeinside ={\%,}, escapechar={!},
    numbers=left, language=Python, caption=Mapping van bestandsextensies naar de bijbehorende document loaders]
loader_mapping = {
    ".md": TextLoader,
    ".txt": TextLoader,
    ".pdf": PyPDFLoader,
    ".docx": UnstructuredWordDocumentLoader,
    ".doc": UnstructuredWordDocumentLoader,
}
\end{lstlisting}

In het onderstaande codefragment worden de documenten ingelezen en verzameld in een array. De inhoud wordt, afhankelijk van de extensie, met een specifieke lader ingelezen. Elk document krijgt extra metadata mee, namelijk de map en het pad naar het bestand, zodat later duidelijk is om welk document het gaat.

\begin{lstlisting}[basicstyle=\small, frame=single, breaklines=true, postbreak=\mbox{\textcolor{red}{$\hookrightarrow$}\space}, escapeinside ={\%,}, escapechar={!},
    numbers=left, language=Python, caption=Inladen en parsen van documenten per bestandstype met toevoeging van metadata]
    documents = []
    for doc_file_path in glob.glob(os.path.join(folder, f"*{ext}")):
        try:
            document = base_loader_type(doc_file_path, encoding="utf-8")
            doc = document.load()
            for d in doc:
                d.metadata = {"source": doc_file_path, "folder": folder}
                documents.append(d)
        except Exception as e:
            print(f"Failed to load file {doc_file_path}: {e}")
\end{lstlisting}

Hierna worden de documenten opgesplitst in chunks. Elke chunk heeft een grootte van 2000 karakters, met een overlap van 500 karakters tussen opeenvolgende chunks. Na het testen van verschillende chunk groottes bleek dit de kleinst mogelijke waarde te zijn waarbij elke chunk nog voldoende relevante informatie bevatte.

\begin{lstlisting}[basicstyle=\small, frame=single, breaklines=true, postbreak=\mbox{\textcolor{red}{$\hookrightarrow$}\space}, escapeinside ={\%,}, escapechar={!}, numbers=left, language=Python, caption=Definiëren van tekstsplitter] 
# Split the document into chunks
text_splitter = RecursiveCharacterTextSplitter(chunk_size=2000, chunk_overlap=500)
docs = text_splitter.split_documents(documents)
\end{lstlisting}

De gegenereerde embeddings worden vervolgens opgeslagen in een vectorstore. Voor deze PoC is gekozen voor ChromaDB, een performante en gebruiksvriendelijke oplossing. Hoewel alternatieven zoals FAISS eveneens mogelijk zijn, werd ChromaDB gekozen vanwege de uitgebreide documentatie en de brede ondersteuning binnen het gebruikte framework LangGraph. Hierdoor kon de implementatie op een efficiënte worden uitgevoerd.

\begin{lstlisting}[basicstyle=\small, frame=single, breaklines=true, postbreak=\mbox{\textcolor{red}{$\hookrightarrow$}\space}, escapeinside ={\%,}, escapechar={!}, numbers=left, language=Python, caption=Aanmaken ChromaDB instantie]
# Return a ChromaDB instance
return (Chroma.from_documents(
        docs, embeddings_function, persist_directory=persistent_directory)
        .as_retriever(search_type=search_type, search_kwargs=search_kwargs))
\end{lstlisting}

Deze initiële set-up maakt het mogelijk om relevante documenten snel op te halen op basis van semantische gelijkenis met de vraag van de gebruiker. Dit is met andere woorden een kerncomponent van de RAG-oplossing.

\subsection{Graaf structuur en nodes}

Eens de vectordatabase met de nodige documenten beschikbaar is kunnen vragen gesteld worden aan de LLM. Die kan aan de hand van een graaf keuzes maken naargelang de inhoud van deze vraag. De volledige workflow werd gemodelleerd als een LangGraph-graaf. Deze structuur is weergegeven in figuur~\ref{fig:langgraph}.

\begin{figure}[H]
    \centering
    \includegraphics[width=0.8\textwidth]{langgraph_workflow.png}
    \caption{Technische uitwerking van de workflow als LangGraph-graaf, waarbij de verschillende nodes de stappen in het proces voorstellen en de pijlen de mogelijke routes aangeven die de LLM kan volgen.}
    \label{fig:langgraph}
\end{figure}

\subsubsection{Antwoorden of documenten ophalen}

Vooraleer het RAG-proces op gang wordt gebracht, moet de LLM eerst een inschatting maken over de vraag die de gebruiker heeft gesteld. Wanneer het gaat om een triviale vraag, dient de LLM onmiddellijk een antwoord te geven. Op die manier hoeft het volledige RAG-proces niet te worden doorlopen. Om dit te realiseren, wordt gebruikgemaakt van de volgende prompt, waarin de input van de gebruiker wordt geïnjecteerd: 

\begin{lstlisting}[basicstyle=\small, frame=single, breaklines=true, postbreak=\mbox{\textcolor{red}{$\hookrightarrow$}\space}, escapeinside ={\%,}, escapechar={!}, numbers=left, language=Python, caption=Prompt voor beslissen tussen direct antwoord of documentopvraging]
RETRIEVE_DOCUMENTS_OR_RESPOND_PROMPT = """
    This method decides whether to call the retriever tool or respond directly.
    
    If the user's question is trivial, respond directly. Just respond directly. Do not show your reasoning or thinking process.
    If the question is non-trivial, use the retriever tool to generate a response.
    If in doubt, use the retriever tool to make sure.
    
    Given the user's question:  
    "{message}"
    
    Determine whether the question is trivial. 
"""
\end{lstlisting}

Het voordeel van deze aanpak is dat geen onnodige resources worden verbruikt bij triviale vragen. Enkel wanneer de vraag van de gebruiker effectief nood heeft aan specifieke contextuele informatie, wordt de retriever geactiveerd.
\\[1em]
De LLM krijgt een retriever tool ter beschikking en kan vervolgens autonoom beslissen of deze tool wordt gebruikt of dat er meteen een antwoord gegenereerd wordt. Dit wordt geïmplementeerd in de volgende functie:

\begin{lstlisting}[basicstyle=\small, frame=single, breaklines=true, postbreak=\mbox{\textcolor{red}{$\hookrightarrow$}\space}, escapeinside ={\%,}, escapechar={!}, numbers=left, language=Python, caption=Functie die beslist tussen direct antwoord en documentopvraging]
def retrieve_documents_or_respond(state: MessagesState) -> MessagesState:
    """
    This methods will call the retriever tool when given a non trivial question is asked.
    In the case of a trivial question it will simply provide a response
    Call the model to generate a response based on the current state. Given
    the question, it will decide to retrieve using the retriever tool, or simply respond to the user.
    """
    message = state["messages"][-1].content
    
    prompt = RETRIEVE_DOCUMENTS_OR_RESPOND_PROMPT.format(message=message)
    
    response_model_with_tools = response_model.bind_tools([myminfin_retriever_tool])
    response = response_model_with_tools.invoke([SystemMessage(content=prompt),
    HumanMessage(content=message)])
    return MessagesState(messages=[response])
\end{lstlisting}

In deze stap ontvangt de LLM twee afzonderlijke berichttypes, een SystemMessage en een HumanMessage. De SystemMessage bevat instructies voor het model hoe het dient te handelen. De oorspronkelijke gebruikersvraag wordt hierbij in de prompt \verb|(RETRIEVE_DOCUMENTS_OR_RESPOND_PROMPT)| geïntegreerd.
\\[1em]
Daarnaast wordt de gebruikersvraag ook afzonderlijk toegevoegd als HumanMessage. Dit berichttype geeft expliciet aan dat de inhoud afkomstig is van een menselijke gebruiker.
\\[1em]
Tijdens de implementatie is gebleken dat het opnemen van beide berichttypes noodzakelijk is om optimale resultaten te behalen. De SystemMessage verschaft duidelijke instructies en context, terwijl de HumanMessage de oorspronkelijke vraag onvervormd aan het model doorgeeft. Op deze manier beschikt het model zowel over specifieke richtlijnen en het exacte bericht van de gebruiker. Dit vergroot de kans op een correcte interpretatie en verwerking.

\subsubsection{Retriever}

Zodra de keuze wordt gemaakt om documenten op te halen, spreekt het model de retriever-tool aan. Deze tool raadpleegt ChromaDB om, op basis van een vooraf bepaalde retrieval methode, de relevante documenten op te halen. Er zijn verschillende methoden beschikbaar om chunks uit de vectordatabank op te vragen. Na het testen van meerdere opties werd gekozen voor de \verb|similarity_search| methode.
\\[1em]
De alternatieve ophaalmethoden bleken minder geschikt tijdens het testen. Het eerste alternatief, \verb|similarity_score_threshold|, vertoont overeenkomsten met de gebruikte \verb|similarity_search|, in de zin dat eveneens de meest relevante documenten worden opgehaald. Het verschil is dat bij deze aanpak een vooraf ingestelde relevantie drempelwaarde vereist is. Alleen chunks die voldoen aan deze drempel worden uit de vectordatabase opgehaald. In de praktijk leidde dit ertoe dat bij bepaalde vragen geen enkele chunk als relevant werd beschouwd, waardoor de vraag niet kon worden beantwoord. Dit ondanks het feit dat het antwoord wel aanwezig was in de vectordatabase.  
\\[1em]
Het tweede alternatief, \textit{Maximal Marginal Relevance} (MMR), kiest daarentegen bewust voor zo veel mogelijk variatie in de opgehaalde chunks. Dit resulteerde echter vaak in documenten die onvoldoende relevant waren ten opzichte van de gestelde vraag.  
\\[1em]
Om die redenen werd uiteindelijk gekozen voor de \verb|similarity_search| methode. Deze functie selecteert de documenten die het meest relevant zijn voor de vraag. In deze PoC specifiek worden bij elke bevraging de vier meest relevante chunks uit de vectordatabase opgehaald.

\subsubsection{Beoordeling documenten}

Na het ophalen van de documenten moet de LLM opnieuw evalueren of het over voldoende informatie beschikt om een antwoord te formuleren. Indien de documenten voldoende relevantie vertonen ten opzichte van de oorspronkelijke vraag, wordt overgegaan tot het genereren van een antwoord. Indien dit niet het geval is, zal de oorspronkelijke vraag geherformuleerd worden en wordt het ophaal proces opnieuw opgestart.

\begin{lstlisting}[basicstyle=\small, frame=single, breaklines=true, postbreak=\mbox{\textcolor{red}{$\hookrightarrow$}\space}, escapeinside ={\%,}, escapechar={!}, numbers=left, language=Python, caption=Functie die beslist tussen antwoord genereren of vraag herschrijven]
def grade_documents(
state: MessagesState,
) -> Literal["generate_answer", "rewrite_question"]:
    """Determine whether the retrieved documents are relevant to the question."""
    
    # Shortcut: If too many messages (multiple rewrites), stop rewriting
    if len(state["messages"]) >= 5:
    return "generate_answer"
    
    question = state["messages"][0].content
    context = state["messages"][-1].content
    
    prompt = GRADE_DOCUMENTS_PROMPT.format(question=question, context=context)
    response = (
    grader_model
        .with_structured_output(GradeDocuments)
        .invoke([HumanMessage(content=prompt)])
    )
    score = response.binary_score
    
    if score == "yes":
        return "generate_answer"
    else:
        return "rewrite_question"
\end{lstlisting}

Om te vermijden dat de LLM de initiële vraag eindeloos blijft herformuleren, wordt gecontroleerd hoe lang de array van de MessagesState is. Wanneer het aantal berichten gelijk is aan of groter is dan 5 (wat neerkomt op maximaal twee herformuleringen), wordt het model verplicht om door te gaan naar de functie die verantwoordelijk is voor het genereren van een antwoord. Op die manier wordt gegarandeerd dat de gebruiker een antwoord ontvangt en de LLM niet in een oneindige lus terecht komt.
\\[1em]
Om ervoor te zorgen dat de LLM deze vraag op een correcte manier gaat verwerken worden zowel de opgehaalde documenten als de originele vraag in een prompt toegevoegd. Hierna is het aan de LLM om een oordeel te vellen over de opgehaalde documenten en de mate waarin deze relevant en nuttig zijn om de vraag te beantwoorden.
\begin{lstlisting}[basicstyle=\small, frame=single, breaklines=true, postbreak=\mbox{\textcolor{red}{$\hookrightarrow$}\space}, escapeinside ={\%,}, escapechar={!}, numbers=left, language=Python, caption=Prompt om opgehaalde documenten te beoordelen op basis van de gestelde vraag]
GRADE_DOCUMENTS_PROMPT = (
    "You are a grader assessing relevance of a retrieved document to a user question. \n "
    "Here are the retrieved documents: \n\n {context} \n\n"
    "Here is the user question: {question} \n"
    "If the document contains keyword(s) or semantic meaning related to the user question, grade it as relevant. \n"
    "Give a binary score 'yes' or 'no' score to indicate whether the document is relevant to the question."
)
\end{lstlisting}

\subsubsection{Genereren antwoord}

Wanneer het hele proces doorlopen is moet de LLM aan de slag met de opgehaalde documenten en de vraag van de gebruiker. Het is hierbij de bedoeling dat het antwoord geformuleerd wordt in de taal waarin ze origineel werd gesteld. Ook wanneer de LLM de vraag heeft herschreven in een andere taal. Om hallucinaties te voorkomen, wordt expliciet gevraagd om alleen een antwoord te geven wanneer de informatie beschikbaar is in de context. Indien dit niet het geval is, moet de LLM dit ook aan de gebruiker melden. Daarnaast wordt gevraagd om bronnen te vermelden indien mogelijk. Hiervoor werd de volgende prompt gebruikt: 

\begin{lstlisting}[basicstyle=\small, frame=single, breaklines=true, postbreak=\mbox{\textcolor{red}{$\hookrightarrow$}\space}, escapeinside ={\%,}, escapechar={!}, numbers=left, language=Python, caption=Prompt voor genereren van antwoord op basis van de opgehaalde context]
GENERATE_ANSWER_PROMPT = (
    "You are a helpful assistant supporting users with their MyMinfin IT-related questions.\n"
    "Based on the following context, please provide a clear and complete answer.\n"
    "If the answer is not available in the context, kindly let the user know that you don't have enough information.\n"
    "If the context contains a source, always mention it in your answer as the reference.\n"
    "Always respond in the same language this question {question} is asked, even if the context is in a different language.\n"
    "Do not use Markdown or any special formatting in your answer, respond in plain text only.\n\n"
    "Question: {question}\n"
    "Context: {context}"
)
\end{lstlisting}

\subsection{LLM-model}

Nu een duidelijk beeld is geschetst van de structuur van de graaf, kan worden bekeken welke modellen in aanmerking komen voor gebruik. Hierbij gelden enkele beperkingen. Ten eerste wordt gebruikgemaakt van zogenoemde open modellen. Een tweede beperking is de hardware waarmee deze PoC wordt uitgevoerd. Omdat de modellen lokaal draaien, is het niet mogelijk modellen te gebruiken die meer RAM vereisen dan wat op de lokale hardware beschikbaar is. Concreet betekent dit dat alleen modellen met maximaal acht miljard parameters in aanmerking komen. Tot slot moeten de modellen in staat zijn tool calls uit te voeren. Op basis van deze criteria en rekening houdend met de rangschikkingen op Hugging Face, is de volgende lijst van mogelijke modellen opgesteld:

\begin{itemize}
    \item Llama-3.1-8b-instruct
    \item Llama-3.2
    \item Mistral-7b-instruct
    \item Qwen2.5-7B instruct
    \item Smollm2-1.7b-instruct
    \item Granite-3.2-8b-instruct
    \item Granite-3.3
    \item Qwen3
\end{itemize}

Tijdens het testen van de PoC werden echter enkele problemen vastgesteld met de tool calls. Dit leidde ertoe dat de onderstaande modellen niet langer gebruikt kon worden. Dit probleem wordt verder toegelicht in subsectie \ref{tool-calls}.

\begin{itemize}
    \item Mistral-7b-instruct
    \item Smollm2-1.7b-instruct
    \item Granite-3.2-8b-instruct
    \item Granite-3.3
\end{itemize}

Aangezien de bovenstaande modellen niet langer bruikbaar bleken, is ervoor gekozen om voor de PoC en de vergelijkende studie de volgende modellen te gebruiken:

\begin{itemize}
    \item Llama-3.1-8b-instruct
    \item Llama-3.2
    \item Qwen2.5-7B instruct
    \item Qwen3
\end{itemize}


\section{Problemen en oplossingen}

Tijdens de ontwikkeling van de toepassing deden zich verschillende problemen voor. Zo waren er uitdagingen bij het efficiënt parsen van documenten, het uitvoeren van tool calls en het onverwachte gedrag van de LLM bij het werken met een graafstructuur. Voor elk van deze problemen werd een passende oplossing gezocht en geïmplementeerd.

%Waar liep je tegenaan tijdens de implementatie?
\subsection{Document parsing}
Het parsen van documenten naar embeddings bleek een uitdagend proces. Initieel werd gewerkt met PDF-bestanden maar deze leverden niet altijd het gewenste resultaat op. Dezelfde problemen traden eveneens op bij Word-documenten. De moeilijkheden deden zich vooral voor bij documenten met ongestructureerde elementen. De structuur van bijvoorbeeld de tabellen ging na het parsen vaak verloren.
\\[1em]
Een goede parsing is echter cruciaal voor een goed functionerende RAG-oplossing. Wanneer documenten niet correct gestructureerd zijn, kan dit problemen veroorzaken. Worden deze documenten vervolgens als embeddings opgeslagen in de vectordatabase, dan kan de gebrekkige structuur opnieuw in de context van de LLM terechtkomen. Dit kan leiden tot problemen bij het genereren van kwalitatieve antwoorden.
\\[1em]
Om dit probleem te verhelpen, werden de documenten in een eerste fase omgezet naar docx-bestanden. Aangezien dit eveneens niet het gewenste resultaat opleverde, werd uiteindelijk gekozen om de documenten te converteren naar Markdown formaat. Deze aanpak had als voordeel dat de documentstructuur tijdens het parsen beter behouden bleef. Wat resulteerde in beter gestructureerde, relevante chunks die vervolgens als context konden worden gebruikt bij het beantwoorden van vragen. De chunks in PDF, docx en Markdown kunnen worden geconsulteerd in bijlage \ref{chunks-verschillende-formaten}

\subsection{Tool calls}
\label{tool-calls}

Omdat in de graaf gebruik wordt gemaakt van een tool, is het noodzakelijk om modellen te gebruiken die in staat zijn om tool calls uit te voeren. Hierdoor is het in deze PoC bijvoorbeeld niet mogelijk om met Google's Gemma modellen te werken. Deze modellen bieden immers geen ondersteuning voor tool calls.
\\[1em]
Hoewel sommige Ollama modellen wel tool calls ondersteunen, blijkt uit de praktijk dat deze functionaliteit niet altijd betrouwbaar is. Hoewel de modellen volgens Ollama tool calls kunnen uitvoeren, bleek in de praktijk dat een aantal modellen hier toch problemen mee had.
\\[1em]
De figuur~\ref{fig:Mistral} illustreert waar het fout loopt bij dit model. Ondanks dat het model zich bewust is van de aanwezigheid van de tool en correct inschat dat de gestelde vraag niet triviaal is, wordt de tool call toch niet effectief uitgevoerd.
\\[1em]
Opmerkelijk is dat de tool call met de juiste parameters wel aanwezig is in de content van het antwoord, maar deze wordt niet als daadwerkelijke tool call geïnterpreteerd of geactiveerd door het model.

\begin{figure}[H]
    \includegraphics[width=0.8\textwidth]{mistral.png}
    \caption{Tool call van het Mistral:7b-model. Het model probeert een tool call uit te voeren, maar faalt hierin.}
    \label{fig:tool_call_Mistral}
\end{figure}

Gelijkaardige problemen deden zich voor bij de volgende modellen:
\begin{itemize}
    \item Mistral-7b-instruct
    \item Smollm2-1.7b-instruct
    \item Granite-3.2-8b-instruct
    \item Granite-3.3
\end{itemize}

Om deze reden was het niet mogelijk deze modellen te gebruiken binnen de PoC en de bijbehorende vergelijkende studie. Een correcte uitvoering van de tool call is immers essentieel om de relevante documenten uit de vectordatabase op te halen.

\subsection{Fout door oneindige herformuleringslus}

Hoewel de LLM in staat is om zelf te bepalen wat een non triviale vraag is, betekent dit niet noodzakelijk dat het antwoord daarop terug te vinden is in de beschikbare documentatie. Zelfs wanneer de informatie wel aanwezig is, kan het alsnog voorkomen dat de LLM, zelfs na herformulering van de vraag, geen passend antwoord weet te genereren. Dit leidde aanvankelijk tot een oneindige lus waarbij uiteindelijk een foutmelding werd gegenereerd zodra de array messages in de MessageState een lengte van 25 bereikte. In dat geval werd er geen antwoord aangemaakt, maar kreeg de gebruiker enkel een foutmelding te zien met stacktrace.
\\[1em]
Om te voorkomen dat het programma in een dergelijke oneindige lus terechtkomt, werd er een extra controle ingebouwd op het moment dat de LLM moet kiezen tussen het herformuleren van de vraag of het genereren van een antwoord. In de praktijk betekent deze controle dat de LLM de vraag maximaal twee keer mag herformuleren. Hierdoor wordt de vectordatabase in totaal maximaal drie keer bevraagd. Indien na deze pogingen nog steeds geen relevante documenten worden teruggevonden, is het de bedoeling dat de LLM dit expliciet communiceert aan de gebruiker. 

\subsection{Meertaligheid documentatie}
%het gaat hier over het feit dat hij steeds de eerste vraag vertaalde naar het engels.
%Hoe heb je dat opgelost? Eventueel kort uitleggen als dat relevant is.
Bij het testen van de lus ontstond het probleem dat de LLM bij het herformuleren van de vraag steeds dezelfde formulering gaf. Dit kwam doordat de LLM alleen werd gevraagd de originele vraag te herschrijven, zonder rekening te houden met eerdere herschrijvingen. Hierdoor werd het doel van de node tenietgedaan, aangezien het juist de bedoeling is de vraag telkens opnieuw te herformuleren om relevantere documenten uit de database op te halen 
\\[1em]
Om dit probleem op te lossen, werd de prompt in de \verb|rewrite_question| node aangepast. De LLM kon aan de hand van deze wijzigingen nu ook rekening houden met de eerder gestelde en herschreven vragen. Dit voorkomt herhaling en zorgt voor meer variatie in de gegenereerde vragen.
\\[1em]
Aangezien het grootste deel van de beschikbare documentatie in het Frans of Engels is geschreven, wordt de LLM eveneens aangemoedigd om de vraag eerst naar het Frans te herschrijven en vervolgens, indien nodig, naar het Engels.

\begin{lstlisting}[basicstyle=\small, frame=single, breaklines=true, postbreak=\mbox{\textcolor{red}{$\hookrightarrow$}\space}, escapeinside ={\%,}, escapechar={!}, numbers=left, language=Python, caption=Prompt om vraag te herschrijven]
REWRITE_QUESTION_PROMPT = (
    "You are assisting with improving a user's question related to Myminfin IT support.\n"
    "Conversation history so far (most recent last):"
    "\n ------- \n"
    "{questions}"
    "\n ------- \n"
    "Original question:"
    "\n ------- \n"
    "{original_question}"
    "\n ------- \n"
    "Rewrite the last question to make it short, clear, and easy to match with relevant documents in a vector database.\n"
    "- Keep the meaning and intent exactly the same.\n"
    "- Use simple, direct wording without adding unnecessary details.\n"
    "- If the question is in Dutch, rewrite and translate it to French.\n"
    "- If the question is in French, rewrite and translate it to English.\n"
    "- Avoid repeating previous questions word-for-word.\n"
    "- If it’s similar to a previous question, add only minimal context needed for distinction.\n"
    "- Return only the rewritten question."
)
\end{lstlisting}
\section{Samenvatting}

%Korte recap: wat is gebouwd en werkt zoals verwacht?
%Eventueel: wat is nog niet geïmplementeerd en waarom niet (scope, tijdsbeperkingen)?

Het eindresultaat van deze PoC is een toepassing die kan worden gebruikt om gerichte vragen te stellen over de IT-support voor MyMinfin. De toepassing werkt zoals verwacht. Afhankelijk van de vraag worden relevante documenten opgehaald en wordt de nodige informatie aan de eindgebruiker verstrekt. Hoewel de hoeveelheid documentatie nog verder opgeschaald kan worden, vormt dit een eerste stap in het opzetten van een ondersteunende tool voor de dagelijkse supportwerking.
\\[1em]
Toch zijn er ook enkele zaken die momenteel nog ontbreken. Zo beschikt de huidige LLM-toepassing niet over enige vorm van geheugen. Bij elke nieuwe vraag begint de LLM volledig van nul, zonder kennis van eerdere interacties binnen dezelfde sessie. Er is dus geen sprake van contextopbouw of het onthouden van vorige vragen en antwoorden. Dit beperkt de gebruikservaring momenteel tot losstaande interacties.
\\[1em]
Zo zou het nuttig zijn indien de LLM-toepassing op vraag de documentatie kan aanpassen of aanvullen. Dit zou ervoor zorgen dat de beschikbare informatie steeds up-to-date blijft, zonder dat hiervoor handmatige tussenkomst van gebruikers nodig is. Deze functionaliteit viel echter buiten de scope van deze PoC en kon, omwille van tijdsbeperkingen, niet worden gerealiseerd.

%%=============================================================================
%% Resultaten
%%=============================================================================

\chapter{Resultaten}
\label{ch:resultaten}

%TODO In de caption van figuren altijd een conclusie, op te merken afwijking… opgeven


%ResponseRelevancy  Scores the relevancy of the answer according to the given question. Answers with incomplete, redundant or unnecessary information is penalized. Score can range from 0 to 1 with 1 being the best.

%Faithfulness The Faithfulness metric measures how factually consistent a response is with the retrieved context. It ranges from 0 to 1, with higher scores indicating better consistency.

% AnswerCorrectness  Measures answer correctness compared to ground truth as a combination of factuality and semantic similarity.

%LLMContextRecall checkt in hoeverre de juiste info voor het antwoord ook effectief aanwezig was in de opgehaalde context. Hoe meer overlap tussen juiste antwoord en context, hoe hoger de score.  aka hoe goed het model zijn retrievalproces heeft aangestuurd.

Om de verschillende modellen te testen, worden meerdere scenario’s uitgevoerd.
Ten eerste wordt gebruikgemaakt van een testset met tien vragen waarvan de informatie beschikbaar is in de documentatie. Van elk model wordt verwacht dat het op iedere vraag een concreet antwoord geeft. Deze antwoorden worden vervolgens geëvalueerd met behulp van het test framework \textit{Ragas}, waarbij vier verschillende meetcriteria worden toegepast.
\\[1em]
Het tweede testscenario bestaat uit het stellen van niet-triviale vragen waarvoor de informatie niet beschikbaar is in de documentatie. Dit scenario is vooral bedoeld om te onderzoeken of bepaalde modellen tekenen van hallucinaties vertonen.
\\[1em]
Als laatste worden enkele triviale vragen gesteld. Hierbij wordt getest of het model in staat is een juiste inschatting te maken door direct te antwoorden, zonder onnodige zoekopdrachten in de vectordatabase.

\section{Evaluatie 1: Documentatiegebaseerde vragen}

Elk model kreeg éénzelfde vragenlijst te verwerken, waarbij de verwachte antwoorden zijn opgenomen in bijlage \ref{vragenlijst}.
De antwoorden die door de modellen zijn gegenereerd, zijn weergegeven in bijlage \ref{antwoordenlijst}.
In beide bijlagen zijn uit privacy overwegingen alle contactgegevens geanonimiseerd.
\\[1em]
Voor deze evaluatie wordt gebruikgemaakt van het testframework Ragas. Tijdens de testen wordt voor iedere vraag de volgende informatie verzameld: de gebruikersinput, de opgehaalde context, het antwoord van de LLM en het verwachte antwoord. Op basis van deze gegevens gebruikt Ragas zelf een LLM om de dataset te beoordelen. In dit geval werd als testmodel \textit{gemma3:4b} gekozen. Dit model biedt een goede balans tussen rekensnelheid, geheugengebruik en beoordelingskwaliteit. Hierdoor kon de evaluatie efficiënt worden uitgevoerd. Dit resulteert in een rapport waarin iedere vraag wordt beoordeeld op verschillende criteria.
\\[1em]
Iedere vraag werd beoordeeld op vier verschillende categorieën. De score per categorie variëren tussen 0 en 1, waarbij een score van 1 een volledig correct antwoord representeert.
\\[1em]
Het meetcriterium Faithfulness beoordeelt hoe consistent het antwoord is met de opgehaalde context. \textit{Response Relevancy} kijkt naar hoe relevant het antwoord is ten opzichte van de vraag. Onvolledige of overbodige informatie leidt hierbij tot een lagere score. \textit{Answer Correctness} beoordeelt hoe correct het antwoord is in vergelijking met het verwachte antwoord, waarbij zowel feitelijke als semantische overeenkomsten worden meegenomen. Context Recall controleert in welke mate de benodigde informatie voor het antwoord aanwezig is in de opgehaalde context. Hoe groter de overlap tussen het juiste antwoord en de context, hoe hoger de score. Dit meetcriterium kijkt dus vooral naar hoe goed het model het ophaalproces aanstuurt.
\\[1em]
Bij sommige modellen en vragen traden tijdens het testen errors op. Dit hield in dat de documentatie ofwel niet werd opgehaald, ofwel foutief uit de vectordatabase kwam. Voor deze vragen werd vastgesteld dat de meetcriteria Faithfulness en Response Relevancy een score van 0 kregen, terwijl Context Recall een perfecte 1 score werd toegekend. Om een realistischer beeld van de prestaties van de modellen te geven, werd voor deze vragen de Context Recall aangepast naar 0. Deze correcties werden toegepast voor Llama3.1 vraag 4 en voor Qwen2.5 vragen 4, 7 en 9.
\\[1em]
%Vragen waar het model geen antwoord op kon bieden waren 1, 2, 7 en 10 Voor vragen 3, gaf het model wel een antwoord maar dit was niet in het geval van vraag 3 niet waarheidsgetrouw in het geval van vragen 5, lag het probleem eerder aan de documentatie die niet éénduidig was. Bij vraag 4 was er een error tijdens het RAG proces wat resulteerde in de slechte resultaten.
In tabel \ref{tab:resultaten_vragen_llama3.1} worden de resultaten van Llama3.1 weergegeven. Opvallend is dat de Response Relevancy in veel gevallen nul scoort, wat aangeeft dat de antwoorden van het model vaak niet relevant zijn voor de gestelde gebruikersvragen. Daarnaast blijft de score voor Answer Correctness over het algemeen laag. Dit valt deels te verklaren doordat het model voor een aantal vragen de nodige informatie niet kon vinden in de vectordatabase.
\\[1em]
Verder valt op dat het model vaak wel de juiste informatie in de context heeft, maar deze niet correct weet te gebruiken om tot het juiste antwoord te komen. Een duidelijk voorbeeld hiervan is vraag 10. Hier was de Context Recall een 1, maar de Answer Correctness slechts 0.205. Het model gaf aan de vraag niet te kunnen beantwoorden, ondanks dat de benodigde informatie wel beschikbaar was in de context.

\begin{table}[H]
    \begin{tabular}{|l|c|c|c|c|}
        \hline
        \textbf{Vraag} & \textbf{Faithfulness} & \textbf{Response Relevancy} & \textbf{Answer Correctness} & \textbf{Context Recall} \\
        \hline
        Vraag 1 & 1.000 & 0.000 & 0.225 & 0.500 \\
        Vraag 2 & 0.333 & 0.000 & 0.221 & 0.000 \\
        Vraag 3 & 0.385 & 0.000 & 0.223 & 0.500 \\
        Vraag 4 & 0.000 & 0.000 & 0.174 & 0.000 \\
        Vraag 5 & 1.000 & 0.863 & 0.212 & 0.125 \\
        Vraag 6 & 0.667 & 0.920 & 0.477 & 1.000 \\
        Vraag 7 & 0.500 & 0.000 & 0.205 & 1.000 \\
        Vraag 8 & 0.714 & 0.000 & 0.408 & 1.000 \\
        Vraag 9 & 0.500 & 0.932 & 0.391 & 1.000 \\
        Vraag 10 & 0.333 & 0.000 & 0.205 & 1.000 \\
        \hline
    \end{tabular}
    \caption{Resultaten per vraag op de vier meetcriteria van het Llama3.1 model. De antwoorden blijken vaak niet relevant voor de vraag (Response Relevancy = 0), en de Answer Correctness blijft over het algemeen laag.}
    \label{tab:resultaten_vragen_llama3.1}
\end{table}

In tabel \ref{tab:resultaten_vragen_llama3.2} worden de resultaten van Llama3.2 weergegeven. Het meest opvallende resultaat is vraag 7, waar het model zeer hoge scores behaalt op alle criteria en bijna een perfecte score bereikt voor Answer Correctness (0.991). Daarnaast laat het model, ondanks de kleinere modelgrootte, toch degelijke prestaties zien.
\\[1em]
Ook de resultaten voor vraag 8 zijn opmerkelijk. Hier behaalt het model op bijna alle criteria een hoge score, met uitzondering van Answer Correctness. Dit verschil wordt veroorzaakt door de ambiguïteit in de opgehaalde documentatie. De bronnen boden geen eenduidig correct antwoord. Het door het model gegeven antwoord komt daardoor niet overeen met de ground truth, maar kan wel worden verantwoord op basis van de beschikbare documentatie.
\begin{table}[H]
    \centering
    \begin{tabular}{|l|c|c|c|c|}
        \hline
        \textbf{Vraag} & \textbf{Faithfulness} & \textbf{Response relevancy} & \textbf{Answer Correctness} & \textbf{Context Recall} \\
        \hline
        Vraag 1  & 0.500 & 0.894 & 0.485 & 0.500 \\
        Vraag 2  & 0.333 & 0.000 & 0.220 & 0.000 \\
        Vraag 3  & 0.800 & 0.000 & 0.219 & 0.500 \\
        Vraag 4  & 0.571 & 0.863 & 0.196 & 0.833 \\
        Vraag 5  & 0.625 & 0.000 & 0.205 & 0.125 \\
        Vraag 6  & 0.857 & 0.000 & 0.194 & 1.000 \\
        Vraag 7  & 0.917 & 0.860 & 0.991 & 1.000 \\
        Vraag 8  & 0.833 & 0.953 & 0.212 & 1.000 \\
        Vraag 9  & 0.429 & 0.883 & 0.631 & 0.375 \\
        Vraag 10 & 0.333 & 0.000 & 0.207 & 1.000 \\
        \hline
    \end{tabular}
    \caption{Resultaten per vraag op de vier meetcriteria van het Llama3.2 model. Opvallend is vraag 7, waar het model bijna perfecte scores behaalt op alle criteria. Bij vraag 8 werd vastgesteld dat ambiguïteit in de documentatie kan leiden tot een lage Answer Correctness, ondanks hoge scores op de andere criteria.}
    \label{tab:resultaten_vragen_llama3.2}
\end{table}

In tabel \ref{tab:resultaten_vragen_qwen2.5} worden de resultaten van Qwen2.5 weergegeven. Opvallend is dat de Response Relevancy vrijwel overal nul scoort en dat de Context Recall zeer laag is. De enige uitzondering is vraag 6, waar het model zowel de context correct ophaalt als een relevante respons genereert. Deze lage scores zijn deels het gevolg van errors tijdens het uitvoeren van het RAG-proces. Een andere reden is dat het model in sommige gevallen direct een antwoord formuleerde zonder eerst de vectordatabase te raadplegen. Ook blijft de score voor Answer Correctness voor de andere vragen relatief laag waardoor de prestaties van dit model niet overtuigend zijn.

\begin{table}[H]
    \begin{tabular}{|l|c|c|c|c|}
        \hline
        \textbf{Vraag} & \textbf{Faithfulness} & \textbf{Response relevancy} & \textbf{Answer Correctness} & \textbf{Context Recall} \\
        \hline
        Vraag 1  & 0.750 & 0.000 & 0.485 & 0.500 \\
        Vraag 2  & 0.500 & 0.000 & 0.226 & 0.000 \\
        Vraag 3  & 0.929 & 0.000 & 0.229 & 0.500 \\
        Vraag 4  & 0.000 & 0.000 & 0.174 & 0.000 \\
        Vraag 5  & 0.000 & 0.000 & 0.180 & 0.000 \\
        Vraag 6  & 1.000 & 0.871 & 0.469 & 1.000 \\
        Vraag 7  & 0.000 & 0.000 & 0.176 & 0.000 \\
        Vraag 8  & 0.000 & 0.000 & 0.179 & 0.000 \\
        Vraag 9  & 0.000 & 0.000 & 0.176 & 0.000 \\
        Vraag 10 & 0.667 & 0.000 & 0.216 & 0.000 \\
        \hline
    \end{tabular}
    \caption{Resultaten per vraag op de vier meetcriteria van het Qwen2.5 model. De Response Relevancy en Context Recall zijn vrijwel overal nul, met uitzondering van vraag 6 waar het model zowel de context correct ophaalt als een relevante respons geeft. Over alle vragen blijft de Answer Correctness laag, wat duidt op matige prestaties van dit model.}
    \label{tab:resultaten_vragen_qwen2.5}
\end{table}


In tabel \ref{tab:resultaten_vragen_qwen3} worden de resultaten van Qwen3 weergegeven. Voor dit model valt op hoe hoog de Faithfulness is wat betekent dat het model zich goed houdt aan de informatie die het heeft opgehaald en bijgevolg minder geneigd is om hallucinaties te vertonen. Het model scoort voor vraag 7 zeer hoog op alle categorieën. De Answer Correctness blijft bij enkele vragen echter laag. Voor vragen 1 tot en met 4 gaf het model aan geen antwoord te kunnen vinden in de context. Bij vraag 5 werd gedateerde informatie uit de vectordatabase opgehaald waardoor het antwoord niet overeenstemde met de ground truth. Voor vraag 9 werd slechts een deel van het correcte antwoord uit de database gehaald. Op basis van de opgehaalde context besloot het model een andere chunk te gebruikten om een antwoord te genereren. Dit resulteert in een lage Answer Correctness met op hetzelfde moment een hoge Faithfulness en een vrij hoge Context Recall.

\begin{table}[H]
    \begin{tabular}{|l|c|c|c|c|}
        \hline
        \textbf{Vraag} & \textbf{Faithfulness} & \textbf{Response Relevancy} & \textbf{Answer Correctness} & \textbf{Context Recall} \\
        \hline
        Vraag 1  & 1.000 & 0.000 & 0.216 & 0.500 \\
        Vraag 2  & 1.000 & 0.000 & 0.224 & 0.000 \\
        Vraag 3  & 1.000 & 0.000 & 0.220 & 0.250 \\
        Vraag 4  & 0.857 & 0.000 & 0.203 & 0.000 \\
        Vraag 5  & 0.636 & 0.867 & 0.211 & 0.125 \\
        Vraag 6  & 0.750 & 0.874 & 0.571 & 1.000 \\
        Vraag 7  & 1.000 & 0.906 & 0.989 & 1.000 \\
        Vraag 8  & 0.556 & 0.000 & 0.411 & 1.000 \\
        Vraag 9  & 1.000 & 0.000 & 0.340 & 0.625 \\
        Vraag 10 & 1.000 & 0.935 & 0.636 & 1.000 \\
        \hline
    \end{tabular}
    \caption{Resultaten per vraag voor het Qwen3 model waarbij opvalt dat Faithfulness hoog is en het model bij vraag 7 op alle meetcriteria hoge scores behaalt terwijl de Answer Correctness bij enkele vragen lager blijft door ontbrekende of gedateerde informatie.}
    \label{tab:resultaten_vragen_qwen3}
\end{table}


\subsection{Samenvatting}

Om de prestaties van de modellen per meetcriterium te vergelijken, worden in figuur~\ref{fig:vergelijking_metrics} vier staafdiagrammen weergegeven. Elk diagram toont de gemiddelde score van de verschillende modellen voor één specifiek meetcriterium.

\begin{figure}[H]
    \centering
    \begin{subfigure}{0.48\textwidth}
        \centering
        \begin{tikzpicture}
            \begin{axis}[
                ybar,
                ymin=0,
                ymax=1,
                bar width=8pt,
                width=\textwidth,
                height=0.6\textwidth,
                enlarge x limits=0.25,
                ylabel={Score},
                symbolic x coords={L3.1,L3.2,Q2.5,Q3},
                xtick=data,
                nodes near coords,
                nodes near coords align={vertical},
                every node near coord/.append style={font=\small},
                title={Faithfulness}
                ]
                \addplot coordinates {(L3.1,0.543) (L3.2,0.620) (Q2.5, 0.385) (Q3,0.880)};
            \end{axis}
        \end{tikzpicture}
    \end{subfigure}
    \hfill
    \begin{subfigure}{0.48\textwidth}
        \centering
        \begin{tikzpicture}
            \begin{axis}[
                ybar,
                ymin=0,
                ymax=1,
                bar width=8pt,
                width=\textwidth,
                height=0.6\textwidth,
                enlarge x limits=0.25,
                ylabel={Score},
                symbolic x coords={L3.1,L3.2,Q2.5,Q3},
                xtick=data,
                nodes near coords,
                nodes near coords align={vertical},
                every node near coord/.append style={font=\small},
                title={Response Relevancy}
                ]
                \addplot coordinates {(L3.1,0.272) (L3.2,0.445) (Q2.5,0.087) (Q3,0.358)};
            \end{axis}
        \end{tikzpicture}
    \end{subfigure}
    
    \vspace{1em}
    
    \begin{subfigure}{0.48\textwidth}
        \centering
        \begin{tikzpicture}
            \begin{axis}[
                ybar,
                ymin=0,
                ymax=1,
                bar width=8pt,
                width=\textwidth,
                height=0.6\textwidth,
                enlarge x limits=0.25,
                ylabel={Score},
                symbolic x coords={L3.1,L3.2,Q2.5,Q3},
                xtick=data,
                nodes near coords,
                nodes near coords align={vertical},
                every node near coord/.append style={font=\small},
                title={Answer Correctness}
                ]
                \addplot coordinates {(L3.1,0.274) (L3.2,0.356) (Q2.5,0.251) (Q3,0.402)};
            \end{axis}
        \end{tikzpicture}
    \end{subfigure}
    \hfill
    \begin{subfigure}{0.48\textwidth}
        \centering
        \begin{tikzpicture}
            \begin{axis}[
                ybar,
                ymin=0,
                ymax=1,
                bar width=8pt,
                width=\textwidth,
                height=0.6\textwidth,
                enlarge x limits=0.25,
                ylabel={Score},
                symbolic x coords={L3.1,L3.2,Q2.5,Q3},
                xtick=data,
                nodes near coords,
                nodes near coords align={vertical},
                every node near coord/.append style={font=\small},
                title={Context Recall}
                ]
                \addplot coordinates {(L3.1,0.613) (L3.2,0.633) (Q2.5,0.2) (Q3, 0.550)};
            \end{axis}
        \end{tikzpicture}
    \end{subfigure}
    
    \caption{Vergelijking van de gemiddelde scores per meetcriterium voor de modellen Llama3.1 (L3.1), Llama3.2 (L3.2), Qwen2.5 (Q2.5) en Qwen3 (Q3). De figuur toont dat Qwen3 de hoogste Faithfulness en Answer Correctness behaalt, terwijl Llama3.2 de hoogste score behaalt voor Response Relevancy en Context Recall.}
    \label{fig:vergelijking_metrics}
\end{figure}

Uit tabel \ref{tab:modelvergelijking} blijkt dat het model Qwen3 de hoogste scores behaalde voor Faithfulness (0.880) en Answer Correctness (0.402), terwijl Llama3.2 het beste presteerde op Response Relevancy (0.445) en Context Recall (0.633). 
\\[1em]
Opvallend is dat niet één model op alle gemeten criteria het beste scoorde. Zo behaalde het Qwen3 model de hoogste score op zowel Faithfulness als Answer Correctness. Dit suggereert dat dit model zich het meest consistent hield aan de geleverde context. Over de set van tien vragen heeft Qwen3 ook de meest accurate antwoorden gegenereerd.
\\[1em]
Voor Response Relevancy was het Llama3.2 model de best presterende. De laagste score op dit meetcriterium werd behaald door het Qwen2.5 model met een opvallend lage waarde van 0.087. Dit wijst erop dat de antwoorden van dit model vaak irrelevante of overbodige informatie bevatten of dat cruciale elementen uit het antwoord ontbraken.
\\[1em]
Op het vlak van Context Recall scoorde het Llama3.2 model het hoogst, op korte afstand gevolgd door Llama3.1. Dit betekent dat Llama3.2 er het best in slaagde om relevante documentatie op te halen. Toch bleek dit niet automatisch te leiden tot een hoge Answer Correctness. Ondanks de hoge Context Recall, presteerde Llama3.2 niet het sterkste op vlak van Answer Correctness. In tegenstelling daarmee behaalde het Qwen3 model wel een hoge Answer correctness met een iets lagere Context Recall, wat impliceert dat het model de opgehaalde informatie beter wist te verwerken en toepassen.

\begin{table}[H]
    \begin{tabular}{|l|c|c|c|c|c|}
        \hline
        \textbf{Model} & \textbf{Faithfulness} & \textbf{Response Relevancy} & \textbf{Answer Correctness} & \textbf{Context Recall} \\
        \hline
        \textbf{Llama3.1} & 0.543 & 0.272 & 0.274 & 0.613 \\
        \textbf{Llama3.2} & 0.620 & \textbf{0.445} & 0.356 & \textbf{0.633} \\
        \textbf{Qwen2.5}  & 0.385 & 0.087 & 0.251 & 0.200 \\
        \textbf{Qwen3}    & \textbf{0.880} & 0.358 & \textbf{0.402} & 0.550 \\
        \hline
    \end{tabular}
    \caption{Vergelijking van modelprestaties op vier evaluatiecriteria. Qwen3 behaalt de hoogste gemiddelde scores, terwijl Llama3.2 de op één na hoogste score behaalt en bovendien relatief consistent presteert over de verschillende categorieën. Dit ondanks het kleinere aantal parameters.}
    \label{tab:modelvergelijking}
\end{table}

Figuur \ref{fig:gemiddelde_score_per_model_met_sd} geeft een overzicht van de gemiddelde score van elk model over de vier categorieën, waarbij elke categorie even zwaar meeweegt in de eindberekening. Op deze manier kunnen de verschillende modellen snel met elkaar worden vergeleken. Hoewel dit niet het hele verhaal vertelt, sluit dit gemiddelde wel aan bij de individuele prestaties van de modellen.
\\[1em]
Gemiddeld scoorde Qwen3 het best met Llama3.2 op de tweede plaats. Dit is opvallend aangezien Llama3.2 slechts drie miljard parameters heeft, tegenover de andere modellen die zeven à acht miljard parameters hebben.
\\[1em]
Figuur \ref{fig:gemiddelde_score_per_model_met_sd} illustreert tevens dat Llama3.2 de meest consistente prestaties vertoont over de vier categorieën. De foutbalken representeren de standaarddeviatie van de scores. Waarbij een kleinere waarde wijst op minder grote spreiding en daarmee op consistente prestaties over de verschillende meetcriteria. Het model combineert met andere woorden een relatief hoge gemiddelde score met een hoge mate van consistentie.

\begin{figure}[H]
    \centering
    \begin{tikzpicture}
        \begin{axis}[
            ybar,
            ymin=0,
            ymax=1,
            bar width=20pt,
            width=0.8\textwidth,
            height=0.5\textwidth,
            enlarge x limits=0.25,
            ylabel={Gemiddelde score},
            symbolic x coords={Llama3.1,Llama3.2,Qwen2.5,Qwen3},
            xtick=data,
            nodes near coords,
            nodes near coords align={vertical},
            every node near coord/.append style={font=\small}
            ]
            \addplot+[
            error bars/.cd,
            y dir=both,
            y explicit
            ] coordinates {
                (Llama3.1,0.426) +- (0,0.178)
                (Llama3.2,0.514) +- (0,0.136)
                (Qwen2.5,0.231) +- (0,0.124)
                (Qwen3,0.548)   +- (0,0.236)
            };
        \end{axis}
    \end{tikzpicture}
    \caption{Gemiddelde score per model over de vier categorieën, met foutbalken die de standaarddeviatie weergeven. Llama3.2 vertoont de meest consistente prestaties, terwijl Qwen3 het hoogste gemiddelde behaalt maar met grotere spreiding.}
    \label{fig:gemiddelde_score_per_model_met_sd}
\end{figure}

\section{Evaluatie 2: Hallucinaties}

Bij deze test kreeg ieder model een set van 5 vragen voorgeschoteld. Geen van deze vragen komt voor in de documentatie. De antwoorden van de verschillende modellen zijn te raadplegen in bijlage \ref{hallucinatie-resultaten}. Dit heeft als doel om na te gaan welke modellen tekenen vertonen van hallucinaties en welke niet. 
\\[1em]
De vragenlijst is als volgt:

\begin{enumerate}
    \item Voor welke dienst is Jan Cabooter verantwoordelijk?
    \item Wat is de maximale bestandsgrootte (in MB) die via de repush voor RV kan worden verstuurd?
    \item Wat is de kost voor een Nationality change?
    \item Wat is de technical resource voor DQSS?
    \item Wat is het maximum aantal documenten die kunnen geraadpleegd worden bij myminfin
\end{enumerate}

\begin{table}[H]
    \begin{tabular}{|l|c|c|c|c|}
        \hline
        \textbf{Vraag} & \textbf{Llama3.1} & \textbf{Llama3.2} & \textbf{Qwen2.5} & \textbf{Qwen3} \\
        \hline
       Vraag 1 & Nee & Ja & Nee & Nee \\
       Vraag 2 & Nee & Ja & Ja & Ja \\
       Vraag 3 & Ja & Nee & Nee & Nee \\
       Vraag 4 & Nee & Ja & Nee & Nee \\
       Vraag 5 & Nee & Ja & Nee & Nee \\
        \hline
    \end{tabular}
    \caption{Resultaten hallucinatie test per model bij vragen die niet in de documentatie voorkomen (Ja = hallucinatie, Nee = geen hallucinatie). Uit de resultaten blijkt dat voornamelijk Llama3.2 frequent hallucinaties vertoont, terwijl de andere modellen dit nauwelijks doen.}
\end{table}

\noindent\textbf{Samenvatting:}  
\begin{table}[H]
    \begin{tabular}{|l|c|}
        \hline
        \textbf{Model} & \textbf{Hallucinatie (\%)} \\
        \hline
        Llama3.1 & 1/5 (20\%) \\
        Llama3.2 & 4/5 (80\%) \\
        Qwen2.5  & 1/5 (20\%) \\
        Qwen3    & 1/5 (20\%) \\
        \hline
    \end{tabular}
    \caption{Samenvatting hallucinatie test. Het percentage geeft aan hoe vaak een model hallucinaties vertoonde. Opvallend is dat Llama3.2 met 80\% veruit het vaakst hallucinaties genereerde, terwijl de andere modellen dit slechts in 20\% van de gevallen deden.}
\end{table}

Uit de resultaten blijkt dat Llama3.2 in 80\% van de gevallen hallucinaties vertoonde, terwijl de overige modellen dit slechts in 20\% van de gevallen deden.

\subsection{Bespreking van de resultaten}

De modellen Qwen3, Qwen2.5 en Llama3.1 vertoonden geen tekenen van hallucinaties. Ze hielden zich goed aan de instructie om geen extra informatie te verzinnen en zich uitsluitend te baseren op de gegevens die in de opgehaalde documenten stonden. Wel ondervond elk van deze modellen problemen bij het beantwoorden van één specifieke vraag. Na nader onderzoek bleek dat, wanneer het model de vraag in het Frans herschreef, de tool call niet altijd correct functioneerde. Het is niet meteen duidelijk waarom dit gebeurde, maar bij de andere vragen traden er bij deze modellen geen hallucinaties op.
\\[1em]
Het Llama3.2-model daarentegen maakte meerdere keren aannames en verzon informatie die niet in de documentatie stond. Dit is uiteraard onwenselijk en kan verstrekkende gevolgen hebben voor de medewerkers die de support uitvoerer. Zeker wanneer zij deze informatie als juist beschouwen.

\subsubsection{Relevantie voor echte toepassingen}
%Hoe cruciaal is dit type vraag in de praktijk?

Aangezien de opgeleverde informatie wordt gebruikt voor het behandelen van specifieke IT-vragen van burgers over MyMinfin, is het van groot belang dat de modellen geen hallucinaties vertonen. Zelfs wanneer het model uitsluitend informatie levert aan de supportmedewerker, is het essentieel dat deze informatie waarheidsgetrouw is. Op zijn minst moet het model bij twijfel expliciet aangeven dat de benodigde informatie niet beschikbaar is. Op basis van deze test kan het Llama3.2-model niet realistisch worden ingezet.

\section{Evaluatie 3: Triviale vraagstelling}

Tijdens deze test kreeg ieder model een reeks van vijf triviale vragen voorgelegd. De resultaten werden bepaald door de stacktrace van ieder model te analyseren. Op die manier werd vastgesteld of het model de vraag direct beantwoordde of eerst onnodig RAG uitvoerde. Voor meer details kan bijlage \ref{triviale-vragen-resultaten} geraadpleegd worden.
\\[1em]
De vragenlijst is als volgt:

\begin{enumerate}
    \item Hallo hoe gaat het?
    \item Hoeveel dagen zitten er in een week?
    \item Wat is 2 + 2?
    \item In welk land ligt Parijs?
    \item Hoeveel minuten zitten er in een uur?
\end{enumerate}

\begin{table}[H]
    \centering
    \begin{tabular}{|l|c|c|c|c|}
        \hline
        \textbf{Vraag} & \textbf{llama3.1} & \textbf{llama3.2} & \textbf{Qwen2.5} & \textbf{Qwen3} \\
        \hline
        Hallo hoe gaat het? & Nee & Nee & Ja & Ja \\
        Hoeveel dagen zitten er in een week? & Nee & Nee & Ja & Ja \\
        Wat is 2 + 2? & Nee & Nee & Ja & Ja \\
        In welk land ligt Parijs? & Nee & Nee & Ja & Ja \\
        Hoeveel minuten zitten er in een uur? & Nee & Nee & Ja & Ja \\
        \hline
    \end{tabular}
    \caption{Resultaten van het correcte gebruik van de graaf per model bij triviale vragen waarbij verwacht wordt dat een direct antwoord wordt gegeven zonder gebruik van RAG. Er is een duidelijk onderscheid tussen de Llama-modellen en Qwen-modellen. De Llama-modellen raadplegen bij iedere vraag de vectordatabase terwijl de Qwen-modellen dit achterwege laten zoals wordt verwacht bij triviale vragen.}
    
\end{table}

\noindent\textbf{Samenvatting:}  
\begin{table}[H]
    \begin{tabular}{|l|c|}
        \hline
        \textbf{Model} & \textbf{Correct (\%)} \\
        \hline
        Llama3.1 & 0/5 (0\%) \\
        Llama3.2 & 0/5 (0\%) \\
        Qwen2.5  & 5/5 (100\%) \\
        Qwen3    & 5/5 (100\%) \\
        \hline
    \end{tabular}
    \caption{Samenvatting van correct gebruik van de graaf per model. Qwen2.5 en Qwen3 volgden in alle gevallen het verwachte gedrag terwijl Llama3.1 en Llama3.2 dit niet deden.}
\end{table}

Uit de resultaten blijkt dat Qwen2.5 en Qwen3 alle triviale vragen direct beantwoordden, terwijl Llama3.1 en Llama3.2 in alle gevallen onnodige RAG uitvoerden.

\subsection{Bespreking van de resultaten}
%Verwachtte je een foutloos resultaat?
De resultaten waren in dit geval vrij duidelijk. De twee Llama-modellen slaagden er niet in onderscheid te maken tussen triviale en niet-triviale vragen. Dit leidde ertoe dat zij bij elke vraag het volledige ophaalproces uitvoerden. De twee Qwen-modellen daarentegen gaven direct antwoord op de triviale vragen. Met andere woorden: het contrast tussen de Qwen- en Llama-modellen was tijdens deze test aanzienlijk.
\\[1em]
Hoewel dit niet direct invloed heeft op de inhoudelijke correctheid van de antwoorden, heeft het wel gevolgen voor de performantie van de toepassing. Bij niet-triviale vragen duurt het beantwoorden aanzienlijk langer bij de Llama-modellen dan bij de Qwen-modellen. Voor een consistente en snelle werking van de PoC zijn de Qwen-modellen daardoor beter geschikt dan de Llama-modellen.
%Kunnen deze fouten vermeden worden met betere prompting?


\section{Samenvatting}
%Doel: Alles samenbrengen en overkoepelend reflecteren.
In deze vergelijkende studie werden vier modellen getest op drie manieren. Met name het beantwoorden van vragen op basis van opgehaalde documentatie, het behandelen van vragen die niet in de documentatie voorkwamen en het omgaan met triviale vragen. Elk model werd bij iedere deeltest beoordeeld op zijn prestaties.
\\[1em]
Op basis van deze testen kan worden geconcludeerd dat Qwen3 het beste model is in deze vergelijkende studie. In de eerste test behaalde het model twee keer de hoogste score en één keer de op één na hoogste score. Daarnaast vertoonde het model geen hallucinaties en ging het op een correcte manier om met triviale vragen. Met andere woorden is dit het enige model dat in alle drie de testen aan de verwachtingen voldeed. Op basis van deze resultaten kan worden geconcludeerd dat Qwen3 binnen deze PoC de meest betrouwbare prestaties leverde.
\\[1em]
De PoC met het Qwen3-model dekt de meeste behoeften die in de requirement analyse naar boven kwamen. Ten eerste gaat het model goed om met meertaligheid. Er werden hieromtrent geen problemen vastgesteld. Het model slaagde er consistent in om te antwoorden in de taal van de gebruiker. 
\\[1em]
Daarnaast gaf het model aan wanneer het onvoldoende kennis had om een vraag te beantwoorden. Dit is van groot belang, aangezien het anders tot foutieve antwoorden kan leiden. Dit resultaat sluit aan bij de hoge score voor Faithfulness die Qwen3 behaalde. Wanneer het model een antwoord genereerde, vermeldde het bovendien vaak de bron van de informatie, wat positief is omdat de gebruiker de informatie op die manier kan verifiëren.  
\\[1em] 
Een aantal van de behoeften zijn niet vervuld. Een concreet voorbeeld hiervan is de afwezigheid van een volwaardige chatfunctionaliteit binnen de PoC.
%%=============================================================================
%% Discussie
%%=============================================================================

\chapter{Discussie}
\label{ch:discussie}

\section{Documentatiegebaseerde vragen}


\section{Vragen buiten documentatie}

\section{Triviale vragen}

\section{Vergelijking en implicaties}
%%=============================================================================
%% Aanbevelingen
%%=============================================================================

\chapter{Aanbevelingen}
\label{ch:aanbevelingen}
%%=============================================================================
%% Conclusie
%%=============================================================================

\chapter{Conclusie}%
\label{ch:conclusie}

% TODO: Trek een duidelijke conclusie, in de vorm van een antwoord op de
% onderzoeksvra(a)g(en). Wat was jouw bijdrage aan het onderzoeksdomein en
% hoe biedt dit meerwaarde aan het vakgebied/doelgroep? 
% Reflecteer kritisch over het resultaat. In Engelse teksten wordt deze sectie
% ``Discussion'' genoemd. Had je deze uitkomst verwacht? Zijn er zaken die nog
% niet duidelijk zijn?
% Heeft het onderzoek geleid tot nieuwe vragen die uitnodigen tot verder 
%onderzoek?


VRAGEN DIE TOT VERDER ONDERZOEK LEIDEN:

%TODO laten corrigeren op taal door ChatGPT

Heel wat zaken binnen dit onderzoek werden proefondervindelijk getest maar de keuzes die binnen deze PoC werden gemaakt zijn niet per definitie steeds de beste keuzes voor iedere RAG usecase. 

Het zou dus met andere woorden interessant zijn om naast een vergelijking van verschillende modellen andere aspecten van een RAG architectuur onder de loep te nemen.
Bijvoorbeeld wat de impact is van verschillende parsers op ongestructureerde bestanden. Wat zijn de verschillende opties en welke kan het best omgaan met PDF bestanden die niet louter uit tekst bestaan. 

Een tweede mogelijk onderzoek is nagaan wat de impact is van de verschillende beschikbare retrieval methodes. Tot op heden zijn er vier methodes beschikbaar en elk hebben ze een andere manier om documenten uit de vector database te halen.

Een andere mogelijkheid is om na te gaan wat de impact is van modellen met een groter of kleiner aantal parameters. Voor dit onderzoek werden verschillende modellen onderzocht met gelijkaardige aantal parameters maar het zou interessant zijn om na te gaan hoe verschillende versies van een model presteren tegenover elkaar. 

Een laatste mogelijk onderzoek is kiezen voor een andere methode. Binnen dit onderzoek werd gekozen voor RAG maar met de evolutie van de modellen en de grotere context die deze bieden is het zeker mogelijk om na te gaan hoe een CAG systeem functioneert en kan werken.



%---------- Bijlagen -----------------------------------------------------------

\appendix

\chapter{Onderzoeksvoorstel}

Het onderwerp van deze bachelorproef is gebaseerd op een onderzoeksvoorstel dat vooraf werd beoordeeld door de promotor. Dat voorstel is opgenomen in deze bijlage.
 
\section*{Samenvatting}

 Binnen de IT-sector is het verlenen van support aan klanten een essentieel onderdeel van de dienstverlening. Dit proces kan echter veel tijd en middelen vergen, vooral binnen grote organisaties. Het is vaak een uitdaging om snel en adequaat antwoorden te bieden op vragen van klanten, wat resulteert in een aanzienlijke investering van resources. Tegelijkertijd verwachten klanten een snelle oplossing voor hun problemen. Het is daarom in het belang van zowel de organisatie als de klant om vragen efficiënt te beantwoorden.

Deze bachelorproef onderzoekt de mogelijkheden voor het ontwikkelen van een virtuele assistent die supportmedewerkers kan ondersteunen bij het vinden van relevante antwoorden. Door middel van interviews met betrokkenen wordt een analyse gemaakt van het huidige proces, met als doel de pijnpunten in het verwerken van supporttickets in kaart te brengen. Daarnaast worden via een grondige literatuurstudie de verschillende opties voor het inzetten van een virtuele supportassistent onderzocht. Het einddoel is het ontwikkelen van een proof of concept dat bijdraagt aan een efficiëntere verwerking van klantvragen en de ondersteuning van medewerkers bij het oplossen van deze problemen.

Het verwachte resultaat van dit onderzoek omvat enerzijds een overzicht van de mogelijkheden van een virtuele supportassistent, met aandacht voor wat praktisch haalbaar is en welke factoren daarbij een rol spelen. Anderzijds wordt op basis van een concrete casus een toepassing ontwikkeld in de vorm van een proof of concept (PoC). Er wordt verwacht dat dergelijke virtuele assistenten in veel gevallen een meerwaarde kunnen bieden, maar niet voor alle bedrijven. Afhankelijk van de specifieke behoeften en de manier waarop een organisatie een virtuele assistent wil inzetten, moet eerst grondig worden geanalyseerd of een dergelijke implementatie daadwerkelijk waarde toevoegt. Pas na een dergelijke analyse kan overwogen worden om een virtuele supportassistent te implementeren.

%---------- Inleiding ---------------------------------------------------------

\section{Inleiding}%
\label{sec:inleiding}

Een ontwikkelingsteam heeft uiteenlopende verantwoordelijkheden. Naast hun kernactiviteit, het ontwikkelen van software, is het ook belangrijk dat zij ondersteuning bieden bij vragen en problemen van klanten. Deze vragen hebben vaak betrekking op fouten of problemen binnen de toepassing. In grote organisaties is het echter vaak niet meteen duidelijk welk team verantwoordelijk is voor een specifieke fout. Het zoeken naar het juiste team kan daardoor tijdrovend en inefficiënt zijn, wat leidt tot onnodig hoge kosten en verspilling van middelen.

Dit onderzoek richt zich op grote organisaties waar efficiënte support essentieel is om klantvragen snel en accuraat af te handelen. Een virtuele assistent kan hierbij een ondersteunende rol spelen door medewerkers, die verantwoordelijk zijn voor het afhandelen van support, te voorzien van een chatbot. Deze chatbot kan antwoorden geven op veelvoorkomende vragen, of de medewerkers doorverwijzen naar relevante documentatie en contactpersonen. Zo fungeert de virtuele assistent als hulpmiddel voor supportmedewerkers, waardoor zij sneller en efficiënter kunnen werken. Dit leidt tot een hogere klanttevredenheid, een snellere afhandeling van supporttickets en mogelijke kostenbesparingen binnen de organisatie.

In dit onderzoek richten we ons specifiek op de supporttickets die IT-teams binnen FOD Financiën moeten afhandelen. Om het proces te verbeteren, is het van belang om een helder beeld te krijgen van de huidige manier waarop supporttickets worden afgehandeld. Dit begint met een grondige analyse van het bestaande proces, waarbij we in kaart brengen waar de knelpunten zich bevinden en waarom deze problemen optreden.

Op basis van deze analyse wordt een duidelijk overzicht gevormd van de zwakke punten in het proces, waarna mogelijke oplossingen kunnen worden onderzocht. Concreet wordt gekeken hoe een Large Language Model (LLM) het proces kan versnellen of efficiënter kan maken. Hiervoor worden verschillende opties voor LLM’s geëvalueerd, waarbij we rekening houden met de functionaliteiten die het model moet bieden en met de financiële en praktische beperkingen.

Als onderdeel van dit onderzoek wordt een Proof of Concept (PoC) ontwikkeld. Deze PoC kan vervolgens worden getest door de betrokken medewerkers, zodat we kunnen beoordelen of het voorgestelde model daadwerkelijk bijdraagt aan een efficiëntere afhandeling van supporttickets.

%---------- Stand van zaken ---------------------------------------------------

\section{Stand van zaken}%
\label{sec:stand van zaken}

In een sterk veranderende omgeving wordt het steeds uitdagender om snel en adequaat incidenten te beheren en klantvragen te beantwoorden. Dit wordt veroorzaakt door factoren zoals de snel evoluerende IT-omgeving en de toenemende complexiteit van IT-infrastructuren. Deze ontwikkelingen leiden tot een groei van diverse en complexe incidenten, die vaak moeilijk binnen het incidentmanagementproces kunnen worden opgelost. Hierdoor worden vragen van de eerstelijnsmedewerkers regelmatig doorgestuurd naar meer gespecialiseerde IT-teams. Hoewel dit begrijpelijk is, resulteert het in hogere kosten en een toename van klantontevredenheid \autocite{Schmidt2024}.

Om hier een antwoord op te bieden kan gekeken worden naar de opkomst van ChatGPT en andere Large Language Models (LLM). Deze hebben de mogelijkheid om generatieve informatieverzameling te vergroten. Dit roept de vraag op of dergelijke systemen binnen een organisatie effectief kunnen worden ingezet en welke factoren daarbij moeten worden overwogen.

Naast het bekende GPT-4 zijn er ook diverse open-source modellen, zoals Llama 2, StableBeluga2 en Mixtral 8x7B. Hoewel deze modellen over het algemeen iets minder krachtig zijn dan GPT-4, bieden ze belangrijke voordelen op het gebied van dataveiligheid en privacy. Bovendien kunnen open-source modellen eenvoudiger worden aangepast en gepersonaliseerd om beter aan te sluiten op de specifieke behoeften van de gebruiker \autocite{KernanFreire2024}.

Hoewel het ontwikkelen van een eigen gepersonaliseerde LLM wellicht als een luxeprobleem kan lijken, kan dit in sommige organisaties cruciaal zijn. Het is dan ook belangrijk om niet alleen te kijken naar de prestaties van een model, maar ook naar de mate waarin het kan worden aangepast aan de specifieke vereisten van de organisatie.

Uit het onderzoek van \textcite{Topsakal2023} blijkt dat er veel mogelijkheden zijn om zelf LLM-gebaseerde modellen op te zetten. In dit onderzoek wordt gebruikgemaakt van LangChain, een open-source framework dat specifiek is ontwikkeld om het bouwen van LLM-toepassingen te vergemakkelijken.

\textcite{Topsakal2023} onderzochten de diverse functionaliteiten van LangChain, waaronder de optie om vragen te beantwoorden op basis van documenten. In dit proces wordt de LLM getraind met de verstrekte documenten, zodat een gebruiker vragen kan stellen over deze documenten en zo snel extra kennis kan opdoen. LangChain ondersteunt een breed scala aan documenttypen, zoals CSV, PDF, HTML, JSON, Excel, GitHub, Google Drive, OneDrive en XML.

Hoewel LangChain als open-source framework veel mogelijkheden biedt, is het verstandig om verder onderzoek te doen naar alternatieve frameworks en hun specifieke voor- en nadelen. Dit maakt het mogelijk om een weloverwogen keuze te maken voor het meest geschikte framework, afgestemd op de specifieke use case. Frameworks zoals Hugging Face en Haystack kunnen hierbij als waardevolle alternatieven worden overwogen.

% Voor literatuurverwijzingen zijn er twee belangrijke commando's:
% \autocite{KEY} => (Auteur, jaartal) Gebruik dit als de naam van de auteur
%   geen onderdeel is van de zin.
% \textcite{KEY} => Auteur (jaartal)  Gebruik dit als de auteursnaam wel een
%   functie heeft in de zin (bv. ``Uit onderzoek door Doll & Hill (1954) bleek
%   ...'')

%---------- Methodologie ------------------------------------------------------
\section{Methodologie}%
\label{sec:methodologie}

Voor de opstart van deze bachelorproef is een uitgebreide literatuurstudie essentieel om de bestaande mogelijkheden op het gebied van virtuele assistenten te verkennen. Deze studie richt zich op een vergelijkende analyse die inzicht biedt in de verschillende beschikbare opties voor het ontwikkelen van een virtuele supportassistent. Hierbij worden de voor- en nadelen van elke optie in kaart gebracht, zodat op basis van deze informatie een onderbouwde keuze kan worden gemaakt voor de uitwerking van een proof of concept. De literatuurstudie moet ook duidelijkheid verschaffen over de benodigde hardware- en softwarevereisten voor de ontwikkeling.

Een eerste verkenning binnen de literatuurstudie richt zich op open-source softwareoplossingen voor de ontwikkeling van een LLM-gebaseerde applicatie. LangChain is hierbij een interessante optie, maar het is van cruciaal belang om ook andere alternatieven te onderzoeken. Voor versiebeheer en samenwerking tijdens de ontwikkeling is het gebruik van een GitHub-repository noodzakelijk.

Het uiteindelijke framework dat gekozen wordt, zal afhankelijk zijn van de specifieke noden en eisen. Deze worden in kaart gebracht door middel van een interview met de betrokken partijen. Tijdens dit proces kunnen de partijen duidelijk aangeven welke functionaliteiten essentieel zijn en welke minder prioriteit hebben. De keuze voor een framework en de selectie van de benodigde tools zullen gebaseerd zijn op de inzichten die door de promotor worden verschaft. Deze inzichten worden eveneens verzameld via een interview.

Binnen de bachelorproef zijn zowel de vergelijkende studie als de proof of concept van cruciaal belang. Om beide onderdelen voldoende aandacht te geven, wordt de beschikbare tijd gelijkmatig verdeeld. De vergelijkende studie krijgt twee maanden om tot een gedegen resultaat te komen, terwijl ook twee maanden worden gereserveerd voor de ontwikkeling van de proof of concept. Het is mogelijk dat ideeën voor de proof of concept al tijdens de vergelijkende studie ontstaan, of dat de vergelijkende studie nog wordt aangepast tijdens de ontwikkeling. Desondanks is het doel om eerst een solide basis te leggen met de vergelijkende studie, die dient als fundament voor de verdere uitwerking van de proof of concept.

Het uiteindelijke resultaat is een virtuele assistent die medewerkers ondersteunt bij het afhandelen van supporttickets. In eerste instantie zal deze tool als proefproject worden getest door een IT-team. De resultaten van deze test zullen worden geanalyseerd en opgenomen in de bachelorproef. Bij een succesvolle test kan een bredere implementatie worden overwogen, al valt dit waarschijnlijk buiten de scope van deze bachelorproef.


%---------- Verwachte resultaten ----------------------------------------------
\section{Verwacht resultaat, conclusie}%
\label{sec:verwachte_resultaten}

Het verwachte resultaat van deze bachelorproef is tweeledig. Enerzijds wordt er een vergelijkende studie opgesteld, die voor bedrijven kan dienen als leidraad bij de implementatie van virtuele support assistenten en hen inzicht biedt in de verschillende mogelijkheden. Het is daarbij essentieel dat de vergelijkende studie niet alleen de technische aspecten behandelt, maar ook de financiële en juridische factoren in kaart brengt.

Anderzijds zal een proof of concept worden ontwikkeld als illustratie van één van de mogelijke opties binnen dit domein. Hiermee wordt de vergelijkende studie praktisch toegepast en vormt het een voorbeeld voor personen of organisaties die een soortgelijk project willen realiseren.

Bedrijven die te maken hebben met uitdagingen op het gebied van support kunnen deze vergelijkende studie benutten om te bepalen welke oplossing het beste bij hun behoeften past. Bovendien biedt de proof of concept, indien relevant en toepasbaar, de mogelijkheid om als basis te dienen voor een eigen implementatie, waarbij aanpassingen kunnen worden gemaakt om deze optimaal af te stemmen op de specifieke bedrijfscontext. Indien de proof of concept voldoende functioneel blijkt, kan deze een meerwaarde vormen door een snellere afhandeling van vragen die klanten hebben. Dit leidt op zijn beurt naar een hogeren klantentevredenheid. 

Het uiteindelijke onderzoeksresultaat zal een op ChatGPT lijkende interface opleveren, waarmee gebruikers gerichte antwoorden kunnen krijgen op vragen die specifiek betrekking hebben op support vragen die hun organisatie krijgt van klanten. Deze oplossing richt zich dus op een bedrijfsspecifieke virtuele support assistent, die getraind is op basis van de interne documentatie van een organisatie en binnen die context wordt ingezet.




%%---------- Andere bijlagen --------------------------------------------------
% TODO: Voeg hier eventuele andere bijlagen toe. Bv. als je deze BP voor de
% tweede keer indient, een overzicht van de verbeteringen t.o.v. het origineel.
\chapter{Bijlagen}
\section{Vragenlijst en verwacht antwoord}
\label{vragenlijst}

\begin{enumerate}
    \item \textbf{Wie is de product owner van AGDP en kan je de contactgegevens geven?} \\
    De product owner van AGDP is XXXXXXX en je kan haar contacteren via\\
    \texttt{XXXXXXX@XXXXXXX.be}.
    
    \item \textbf{Wie is de product owner van AGPR en kan je de contactgegevens geven?} \\
    De product owner van AGPR is XXXXXXX en je kan hem contacteren via\\
    \texttt{XXXXXXX@XXXXXXX.be}.
    
    \item \textbf{Kan je me alle contactgegevens geven voor het fragment UBO-register, ook de SNOW groep?} \\
    Fragment: UBO-register – Contactgegevens
    \begin{itemize}
        \item ICT:
        \item Group SNOW: TRES\_GRP\_CIFI
        \item Service manager: XXXXXXX (back-up: XXXXXXX)
        \item Business:
        \item Business analyst: XXXXXXX
    \end{itemize}
    
    \item \textbf{Hoe of waar moet ik mijn tijd loggen voor de support?} \\
    Where to log support time (from most preferred to least preferred):
    \begin{enumerate}
        \item Related ticket
        \begin{itemize}
            \item Project: MYMINFIN
            \item Log time to an existing issue whenever possible.
        \end{itemize}
        \item New bug
        \begin{itemize}
            \item Project: MYMINFIN
            \item If the reported problem requires creating a bug issue, log your time in an “analysis” subtask.
        \end{itemize}
        \item Business Support
        \begin{itemize}
            \item Fisc: [MYMINFIN-9958] [support] MYMINFIN 3rd line AGFISC/AAFISC - JIRA Cloud – ACTIVE
            \item Rec: MYMINFIN-12102: Recouvrement Support – ACTIVE
            \item Patdoc: MYMINFIN-12104: PatDoc Support – ACTIVE
            \item Douane: MYMINFIN-12106: Douane Support – ACTIVE
            \item Trésorerie: MYMINFIN-12108: Trésorerie Support – ACTIVE
            \item (Until end of August, except FISC)
        \end{itemize}
        \item Transversal Support (last resort if none of the above applies)
        \begin{itemize}
            \item Transversal: MYMINFIN-12182: Transversal Support – ACTIVE
            \item ICT: MYMINFIN-12110: Support Day – ACTIVE
            \item (Until end of August)
        \end{itemize}
    \end{enumerate}
    
    \item \textbf{Een gebruiker heeft geen toegang tot een functionaliteit wat moet ik hiermee doen?} \\
    Vragen met betrekking tot toegangsrechten moeten voortaan worden doorgestuurd naar de algemene mailboxen van de IAM-cellen.\\
    \textbf{Let op:} deze adressen MOGEN NIET aan klanten worden meegedeeld.\\
    Fiscaliteit: XXXXXXX@XXXXXXX.be\\
    Invordering \& Innen: XXXXXXX@XXXXXXX.be \\
    Patrimoniumdocumentatie: XXXXXXX@XXXXXXX.be\\
    Douane \& Accijnzen: XXXXXXX@XXXXXXX.be\\
    Thesaurie: XXXXXXX@XXXXXXX.be\\
    Algemeen: XXXXXXX@XXXXXXX.be
    
    \item \textbf{Er is een probleem met attest service, wie kan ik hiervoor contacteren?} \\
    Je kan contact opnemen met het development team via \texttt{XXXXXXX@XXXXXXX.be}.
    
    \item \textbf{Hoe doe ik een repush voor RV?} \\
    Hier zijn de stappen die gevolgd moeten worden wanneer het RV-team vraagt om een “repush” uit te voeren:
    \begin{enumerate}
        \item Open MyMinfin Civil Servant in een incognitovenster.
        \item Open een tweede tabblad en zet er alvast: \texttt{http://www.finbel/}
        \item Zorg ervoor dat u over de volgende informatie beschikt:
        \begin{itemize}
            \item Application Type
            \item Status
            \item ID
        \end{itemize}
        \item In het tweede tabblad, plak een link met de volgende structuur achter:\\
      \begin{verbatim}
{http://www.finbel/myminfin-rest/myminfin-files-stirint
  /put-in-queue/\{applicationType\}/\{status\}/\{id\}}
      \end{verbatim}
        Voorbeeld:
        \begin{verbatim}
{myminfin-rest/myminfin-files-stirint/put-in-queue-by-id
    /WITHHOLDING\_TAX/VALIDATED/204832}
        \end{verbatim}
        \textbf{Opmerking:} Als u het volledige adres van de aanvrager krijgt, let er dan op dat er niet te veel spaties in staan.
        \item Druk op ENTER.
        \item Als alles goed verlopen is, geeft het systeem een succesbericht terug dat gecombineerd moet zijn met de melding 'queued files: 1'.\\
        Voorbeeld: 
        \begin{verbatim}
{"restMessages":[{"type":"SUCCESS",
        "bundleKey":"queued files : 1",...}]}
        \end{verbatim}
      \end{enumerate}
        
        \item \textbf{Iemand wil een document verwijderen dat hij heeft opgeladen via MyMinfin, wat kan ik hierop antwoorden?} \\
        Je kan het volgende antwoorden: MYMINFIN slaat geen gegevens op. Als het proces niet toelaat om te annuleren, moet de gebruiker contact opnemen met de afdeling die het document heeft ontvangen.\\
        Ik vermoed dat dit document is verstuurd via Antwoord op een brief - in dit geval moet de klant de gebruikte driecijferige code meedelen en moet je het ticket doorsturen naar het SOH-team. Die kunnen je vertellen met welke bevoegde service contact moet worden opgenomen.\\
        Wij kunnen u niet helpen.
        
        \item \textbf{Een burger kan werkt bij een onderneming, maar kan niet inloggen in de naam van de onderneming, wat moet ik hierop antwoorden?} \\
        Indien u niet de wettelijke vertegenwoordiger van de onderneming bent, hebt u een rol nodig om in te loggen in MYMINFIN in naam van uw eigen onderneming (inclusief filialen).\\
        U wilt in MYMINFIN toegang tot:
        \begin{itemize}
            \item de berichten van de onderneming, dan hebt u de rol FODFIN Berichten van de onderneming nodig;
            \item de documenten en formaliteiten van de onderneming, dan hebt u meestal genoeg aan de rol FOD Fin Aanstelling Eigen Onderneming met één of meerdere parameters in functie van uw verantwoordelijkheden (bv. Om de BTW formaliteiten te beheren, hebt u de parameter ‘Aanstelling BTW' nodig).
            \end{itemize}
Indien uw onderneming meerdere filialen telt, hebt u enkel de rol nodig die volstaat voor elk van die ondernemingen. Vergeet het begrip ‘mandaten'. Die zijn niet nodig als het om uw eigen onderneming (inclusief filialen) gaat.\\
Voor meer informatie kan u de volgende websites raadplegen:
\begin{itemize}
\item Welke rol ik moet toekennen : Rollen
\item Hoe toekennen of aanvragen van een rol : Beheer van de toegangsaanvragen in de rollenadministratie- RMA - Loginhulp.be
\end{itemize}
Eens dit in orde is, logt u aan op MyMinfin en identificeert zich “in naam van een onderneming”.\\
Selecteer uw eigen onderneming (of een filiaal).

\item \textbf{Wat zijn de contactgegevens voor de hypoimage service?} \\
service : AGDP - HYPOIMAGE \\
contacts : \texttt{XXXXXXX@XXXXXXX.be} ou PATDOC-RZSJ-Hypo (1e lijn) \\
- XXXXXXX@XXXXXXX.be (ICT - servicemanager ad interim) \\
- XXXXXXX@XXXXXXX.be (ICT - développeur)
\end{enumerate}

\section{Antwoordenlijst}
\label{antwoordenlijst}

\subsection{Qwen3}
\begin{enumerate}
    \item \textbf{Wie is de product owner van AGDP en kan je de contactgegevens geven?} \\
    De context bevat geen informatie over de product owner van AGDP of contactgegevens van deze persoon. De beschikbare informatie vermeldt contactgegevens voor services zoals AGDP - MYRENT en AGDP - HYPOIMAGE, maar deze zijn gericht op ICT-managers en andere functies, niet op de product owner. Zonder specifieke vermelding van de product owner in de context kan deze vraag niet volledig beantwoord worden.
    
    \item \textbf{Wie is de product owner van AGPR en kan je de contactgegevens geven?} \\
    De product owner van AGPR wordt niet expliciet genoemd in de beschikbare context. In de context wordt slechts vermelding gedaan van product owners voor andere projecten onder MYMINFIN (zoals AGFISC, AGDA, AGDP en AGTRES), maar er is geen informatie over AGPR. Daarnaast zijn er geen contactgegevens verstrekt voor AGPR in de opgegeven bronnen. U moet contact opnemen met de relevante afdeling of product owner voor verdere informatie.
    
    \item \textbf{Kan je me alle contactgegevens geven voor het fragment UBO-register, ook de SNOW groep?} \\
    De context bevat geen informatie over contactgegevens voor het fragment UBO-register of specifieke contactgegevens voor de SNOW groep gerelateerd aan het UBO-register. Er zijn echter enkele algemene SNOW groepen vermeld in de context, zoals **SERVICEDESK** en **TRANS-IAM**, maar deze zijn niet direct gekoppeld aan het UBO-register. Voor verdere hulp wordt aangeraden contact op te nemen met de relevante diensten via de beschikbare kanalen zoals SNOW of e-mailadressen die in de context worden vermeld.
    
    \item \textbf{Hoe of waar moet ik mijn tijd loggen voor de support?} \\
    Het loggen van tijd voor ondersteuning wordt in de gegeven context niet expliciet vermeld. De beschikbare informatie bevat vooral instructies over technische implementaties, contactgegevens voor ondersteuning, en procedures voor het openen van tickets (zoals via ServiceNow). Voor specifieke richtlijnen over tijdregistratie voor ondersteuning zou je contact moeten opnemen met je interne service desk of technische teams, zoals vermeld in de context. Geen directe referentie naar tijdlogboekingsprocedures.
    
    \item \textbf{Een gebruiker heeft geen toegang tot een functionaliteit wat moet ik hiermee doen?}
\begin{verbatim}
Als een gebruiker geen toegang heeft tot een functionaliteit, 
volg dan de volgende stappen:  
1. Controleer bij de gebruiker welke **egov-rollen** 
hij heeft voor zijn bedrijf. 
Als hij mandataris is, check dan welke mandaten
hij heeft ondertekend met zijn klant.  
2. Controleer of deze rol of mandaat toegang geeft tot de functionaliteit 
(gebruik de tool voor toegangsrechten).  
3. Als de gebruiker de juiste rol en machtiging heeft,
 maar geen toegang, stuur het ticket door naar de **IAM dienst** 
 (zie ook de contactgegevens in de context).  

Voor specifieke functionaliteiten zoals bankgegevens, 
zie de contactinformatie in de **bank-account**-sectie. 
Als het probleem niet duidelijk is, 
meld het aan uw communicatieteam of servicedesk.  

Bron: Treatment incidents (toegangsrechten) en Contacts (IAM ICT).
\end{verbatim}
    \item \textbf{Er is een probleem met attest service, wie kan ik hiervoor contacteren?} \\
    Als er een probleem is met de attest-service, kunt u contact opnemen met de volgende personen:  
    - **XXXXXXX** (ICT - dev-team, prioriteit voor technische problemen)  
    - **XXXXXXX** (ICT - Service Manager)  
    - **XXXXXXX** (ICT - dev)  
    - **XXXXXXX** / **XXXXXXX** (business - chefs de projet)  
    
    Deze contactgegevens zijn te vinden in de context van **Treatment incidents**. Voor technische problemen wordt aanbevolen om eerst contact op te nemen met het dev-team.
    
    
    \item \textbf{Hoe doe ik een repush voor RV?} \\
\begin{verbatim}
1. Open **MyMinfin Civil Servant** in een incognito- of privé-venster.  
2. Open een tweede tab en bezoek de URL: [http://www.finbel/]
(http://www.finbel/).  
3. Verzamel de benodigde informatie:  
- **Application Type** (bijv. "WITHHOLDING TAX")  
- **Status** (bijv. "VALIDATED")  
- **ID** (bijv. "204832")  
4. Plak in het tweede tab het volgende linkformaat achter de URL:  
`myminfin-rest/myminfin-files-stirint/put-in-queue
/{applicationType}/{status}/{id}`  
Voorbeeld: `myminfin-rest/myminfin-files-stirint
/put-in-queue-by-id/WITHHOLDING TAX/VALIDATED/***204832***`  
Let op: Zorg dat er geen extra spaties zijn in de URL.  
5. Druk op **ENTER**.  
6. Controleer of het systeem een succesbericht 
retourneert met de tekst *"queued files: 1"*.  
Voorbeeld van een succesreactie:  
{
    "restMessages": [
    {
        "type": "SUCCESS",
        "bundleKey": "queued files: 1",
        "sticky": false,
        "date": "2023-07-03T13:41:29.848+0000"
    }
    ],
    "mustRedirectToHome": false,
    "sessionExpired": false,
    "error": false
}
\end{verbatim}

        
    Bron: RV
    \item \textbf{Iemand wil een document verwijderen dat hij heeft opgeladen via MyMinfin, wat kan ik hierop antwoorden?} \\
    Als iemand probeert om een document te verwijderen dat jij hebt opgeladen via MyMinfin, kun je antwoorden dat jezelf geen toegang hebt tot de functie voor verwijderen en dat de service die het document heeft ontvangen moet worden benaderd. Dit is gebaseerd op de informatie uit de context, waarin staat dat MYMINFIN geen gegevens opslaat en dat gebruikers contact moeten opnemen met de afdeling die het document heeft ontvangen als het niet mogelijk is om een aanvraag te annuleren (bron: Rôles Rollen). Bovendien wordt aangeraden om contact op te nemen met je IAM-responsabiel als je toegang hebt tot intern beheer en verificatie nodig hebt.
    
    \item \textbf{Een burger kan werkt bij een onderneming, maar kan niet inloggen in de naam van de onderneming, wat moet ik hierop antwoorden?} \\
    De situatie waarin een burger werkt bij een onderneming maar niet kan inloggen in de naam van de onderneming is gerelateerd aan een beperking in het systeem. Volgens de context (bron: *Naam onderneming niet automatisch ingevuld*) wordt de naam van een onderneming niet automatisch ingevuld in MyMinfin, omdat het systeem enkel namen in het Frans, Nederlands en Duits ontvangt. Wanneer een onderneming haar naam in het Engels registreert bij de Kruispuntbank, wordt deze niet overgenomen in Sitran, en dus niet beschikbaar in MyMinfin. Dit leidt tot problemen bij het invullen van bepaalde formulieren, zoals het E705-formulier, waar de ondernemingsnaam automatisch moet worden ingevuld.
    
    **Oplossing**: Er is momenteel geen directe oplossing voor dit probleem, aangezien het gaat om automatisch geleverde data. Indien het probleem zich herhaalt, kan er later een businessregel worden toegepast. Tot dan toe wordt aangeraden om contact op te nemen met de **servicedesk** van het SPF Economie (telefoonnummer: 02 576 36 36) voor verdere ondersteuning bij authenticatieproblemen (zie bron: *Treatment incidents*). 
    
    Let op: De context bevat geen specifieke instructies voor het inloggen onder de ondernemingsnaam, maar benadrukt dat de naam niet automatisch beschikbaar is.
    
    
    \item \textbf{Wat zijn de contactgegevens voor de hypoimage service?} \\
    \begin{verbatim}
De contactgegevens voor de hypoimage service zijn als volgt:  
- **Eerste lijn ondersteuning**: [XXXXXXX@XXXXXXX.be]
(mailto:XXXXXXX@XXXXXXX.be) 
(te gebruiken voor contact en tickettoewijzing aan het groep PATDOC-RZSJ-Hypo).  
- **ICT-service manager ad interim**: [XXXXXXX]
(mailto:XXXXXXX@XXXXXXX.be).  
- **ICT-ontwikkelaar**: [XXXXXXX]
(mailto:XXXXXXX@XXXXXXX.be).  

Bron: Treatment incidents.
    \end{verbatim}
\end{enumerate}

\subsection{Qwen2.5:7b-instruct}
\begin{enumerate}
    \item \textbf{Wie is de product owner van AGDP en kan je de contactgegevens geven?} \\
    De product owner van AGDP is XXXXXXX.
    Helaas heb ik geen contactgegevens voor XXXXXXX in de gegeven context. Voor verdere informatie over contactgegevens kunt u proberen via een van de genoemde e-mailadressen of telefoonnummer 02 576 36 36 om verder te vragen.
    
    \item \textbf{Wie is de product owner van AGPR en kan je de contactgegevens geven?} \\
    \begin{verbatim}
De product owner van AGPR is XXXXXXX. Hier zijn de contactgegevens:

- Naam:XXXXXXX
- Email: Geen specifieke email vermeld in het gegeven context,
 maar je kunt proberen via het service ABA te communiceren.
- Telefoonnummer: Geen specifiek nummer vermeld,
 maar je kunt contact opnemen met de ABA-dienst.
Als je verdere hulp nodig hebt of meer informatie wilt, 
raadpleeg dan aub het service ABA.
    \end{verbatim}
    
    \item \textbf{Kan je me alle contactgegevens geven voor het fragment UBO-register, ook de SNOW groep?} \\
    \begin{verbatim}
Based on the provided context, I can provide some contact information 
for the UBO-register fragment and SNOW group, but not all details are available.

For the UBO-register fragment:
- No specific contact information is listed in the given context.

For the SNOW group related to the UBO-register:
- The relevant SNOW group mentioned is **SERVICEDESK**. 
- Contact: 02 576 36 36

Please note that the context does not provide additional 
contacts or detailed information for the UBO-register fragment specifically.

For other fragments and groups, some contact details are available:
- For obligation de retenue (Obligation to Withhold):
- Service manager:XXXXXXX (backup:XXXXXXX)
- Business group SNOW: IR-DAD-Bus
- Business analyst:XXXXXXX

For ICT-related groups:
- Group SNOW: IR-ICT-Team3, TRANS-Doc&eForm, PATDOC-MEOW-Maps-IT
- Service managers and experts are listed for each group.

If you need more detailed or specific information, 
please provide additional context or clarify your request.
    \end{verbatim}
    
    \item \textbf{Hoe of waar moet ik mijn tijd loggen voor de support?} \\
    \textit{Voor dit antwoord werden geen documenten opgehaald.}
    
    \item \textbf{Een gebruiker heeft geen toegang tot een functionaliteit wat moet ik hiermee doen?} \\
    \textit{Voor dit antwoord werden geen documenten opgehaald.}
   
    \item \textbf{Er is een probleem met attest service, wie kan ik hiervoor contacteren?} \\
    \begin{verbatim}
Voor problemen met de attest service kunt u contact opnemen met:

- De ontwikkelteam van ICT via [XXXXXXX@XXXXXXX.be]
(mailto:XXXXXXX@XXXXXXX.be)
- XXXXXXX (ICT - Service Manager) via [XXXXXXX@XXXXXXX.be]
(mailto:XXXXXXX@XXXXXXX.be)
- XXXXXXX (ICT - ontwikkelaar) via [XXXXXXX]
(mailto:XXXXXXX)
-XXXXXXX of XXXXXXX (business - projectleiders) via 
[XXXXXXX@XXXXXXX.be](mailto:XXXXXXX@XXXXXXX.be) / [XXXXXXX]
(mailto:XXXXXXX@XXXXXXX.be)

Referentie: Treatment_incidents
    \end{verbatim}
    
    \item \textbf{Hoe doe ik een repush voor RV?} \\
    \textit{Voor dit antwoord werden geen documenten opgehaald.}
    
    \item \textbf{Iemand wil een document verwijderen dat hij heeft opgeladen via MyMinfin, wat kan ik hierop antwoorden?} \\
    \textit{Voor dit antwoord werden geen documenten opgehaald.}
    
    \item \textbf{Een burger werkt bij een onderneming, maar kan niet inloggen in de naam van de onderneming, wat moet ik hierop antwoorden?} \\
    \textit{Voor dit antwoord werden geen documenten opgehaald.}
    
    \item \textbf{Wat zijn de contactgegevens voor de hypoimage service?} \\
    Helaas hebben we geen specifieke contactgegevens voor de hypoimage service gevonden in het gegeven context. Het lijkt erop dat de informatie over contactpersonen en diensten gerelateerd is aan andere services zoals reply-letter, service desk, VTOM, en myID.be. Voor verdere hulp of vragen over de hypoimage service zou u mogelijk direct moeten contacteren met het bedrijf zelf of een relevante afdeling binnen MyMinfin.
    
\end{enumerate}

\subsection{Llama3.1:8b-instruct-q4 K M}
\begin{enumerate}
    \item \textbf{Wie is de product owner van AGDP en kan je de contactgegevens geven?} \\
    De product owner van AGDP is niet direct vermeld in de gegeven bronnen. Echter, in het gedeelte "Rôles\_Rollen" wordt er verwezen naar de contactgegevens voor verschillende diensten en afdelingen. Voor de service AGDP - MYRENT zijn de contactgegevens [XXXXXXX@XXXXXXX.be](mailto:XXXXXXX@XXXXXXX.be) of PATDOC-MyRent (1e lijn), [XXXXXXX@XXXXXXX.be](mailto:XXXXXXX@XXXXXXX.be) (ICT - service manager) en [XXXXXXX@XXXXXXX.be](mailto:XXXXXXX@XXXXXXX.be) (business-responsable service).
    
    \item \textbf{Wie is de product owner van AGPR en kan je de contactgegevens geven?} \\
    De product owner van AGPR is niet te vinden in de gegeven bronnen. Wel vind je contactinformatie voor andere diensten en teams, maar die van AGPR zelf is niet beschikbaar.
 
    \item \textbf{Kan je me alle contactgegevens geven voor het fragment UBO-register, ook de SNOW groep?} \\
    Ik kan je de contactgegevens geven voor het fragment UBO-register, inclusief de SNOW groep. Volgens de informatie in de bron "Contacts" (bron 1) zijn de contacten voor het UBO-register:
    
    * XXXXXXX
    * XXXXXXX
    * L'utilisateur moet d'office worden gericht naar\\
     XXXXXXX@XXXXXXX.be
    
    De SNOW groep voor dit fragment is niet expliciet vermeld in deze bron, maar we kunnen de informatie uit andere bronnen gebruiken om een antwoord te geven. In de bron "Treatment\_incidents" (bron 3) staat dat voor databreach-incidenten de IAM PRIVACY Business moet worden bereikt, en dat de SNOW groep hiervoor **IAM-VP** is.
    
    Dus, de contactgegevens voor het fragment UBO-register zijn:
    
    * XXXXXXX
    * XXXXXXX
    * L'utilisateur moet d'office worden gericht naar\\
     XXXXXXX@XXXXXXX.be
    * SNOW groep: **UBO-REGISTER** (niet expliciet vermeld, maar we kunnen de informatie uit andere bronnen gebruiken om een antwoord te geven. In dit geval is de SNOW groep voor databreach-incidenten **IAM-VP**, dus het is waarschijnlijk dat deze ook geldt voor het UBO-register)
    
    Ik hoop dat dit helpt! Laat me weten als je nog meer informatie nodig hebt.
    
    
    
    \item \textbf{Hoe of waar moet ik mijn tijd loggen voor de support?} \\
    \textit{Voor dit antwoord werden geen documenten opgehaald.}
    
    \item \textbf{Een gebruiker heeft geen toegang tot een functionaliteit wat moet ik hiermee doen?} \\
    Als een gebruiker geen toegang heeft tot een functionaliteit, moet je eerst controleren of hij de juiste rol en machtiging heeft voor zijn bedrijf of klant. Dit kan gedaan worden door te kijken naar zijn eGov-rollen en mandaten die hij heeft ondertekend met zijn klant. Als deze informatie niet beschikbaar is op de site van de FOD Financiën, moet je dit melden aan het communicatieteam.
    
    Als de gebruiker de juiste rol en machtiging heeft, maar nog steeds geen toegang heeft, moet je een ticket openen bij de IAM dienst. Als de gebruiker echter mandataris is, moet je controleren welke mandaten hij heeft ondertekend met zijn klant en of deze hem toegang geeft tot het document of functionaliteit.
    
    Als er sprake is van een datalek ("databreach"), moet de gebruiker een klacht indienen via de website van de FOD Financiën, die automatisch doorgestuurd worden naar de bevoegde afdeling IAM Security.
    
    Bron: Treatment\_incidents 
    
    \item \textbf{Er is een probleem met attest service, wie kan ik hiervoor contacteren?} \\
    Als je een probleem hebt met de attest service, kun je contact opnemen met het ICT-dev team via [ XXXXXXX@XXXXXXX.be](mailto: XXXXXXX@XXXXXXX.be). Als je prioriteit hebt voor technische problemen, is dit de beste optie.
    
    \item \textbf{Hoe doe ik een repush voor RV?} \\
    Ik kan je niet helpen met het antwoord op deze vraag omdat de context over RV gaat en niet over MyMinfin. De context geeft aan hoe je een repush voor RV moet doen, maar niet hoe je dit doet in MyMinfin.
    
    
    \item \textbf{Iemand wil een document verwijderen dat hij heeft opgeladen via MyMinfin, wat kan ik hierop antwoorden?} \\
    U kunt antwoorden: "Als u wilt dat ik het document verwijder, moet u contact opnemen met de afdeling die het document heeft ontvangen. U kunt dit doen door naar [www.myminfin.be](http://www.myminfin.be) te gaan en inloggen als mandataris voor uw onderneming. Vervolgens selecteert u uw klant en zoekt u het dossier op. Van daaruit kunt u contact opnemen met de afdeling die het document heeft ontvangen om hen te vragen het document te verwijderen."
    
    
    \item \textbf{Een burger kan werkt bij een onderneming, maar kan niet inloggen in de naam van de onderneming, wat moet ik hierop antwoorden?} \\
    Als een burger werkt bij een onderneming, maar niet kan inloggen in de naam van die onderneming, moet je hem adviseren om contact op te nemen met de servicedesk op 02 576 36 36. Als zij geen oplossing kunnen vinden, kan hij een ticket openen bij de dienst IAM ICT.
    
    \item \textbf{Wat zijn de contactgegevens voor de hypoimage service?} \\
    Ik kan de contactgegevens voor de hypoimage service niet vinden in het gegeven context. Het enige wat ik vind is dat er een tabel staat met contactgegevens voor verschillende diensten, maar die van hypoimage zijn niet vermeld.
\end{enumerate}

\subsection{Llama3.2}
\begin{enumerate}
    \item \textbf{Wie is de product owner van AGDP en kan je de contactgegevens geven?} \\
    De product owner van AGDP is XXXXXXX. U kunt haar contactgegevens vinden in de bron 'Rôles\_Rollen' onder het punt (7) BUCA, nouveau projet.
    
    Reference: Rôles\_Rollen
    
    
    \item \textbf{Wie is de product owner van AGPR en kan je de contactgegevens geven?} \\
    Het product owner van AGPR is XXXXXXX. De contactgegevens voor XXXXXXX zijn niet direct beschikbaar in de geleverde context, maar volgens het document "Treatment\_incidents" kan je haar contact opnemen via het service ABA.
    
    
    \item \textbf{Kan je me alle contactgegevens geven voor het fragment UBO-register, ook de SNOW groep?} \\
    Ik heb genoeg informatie om je te helpen bij je vraag. Het UBO-register is een register van onroerende boekhouding en het SNOW-groep is een groep in de IT-afdeling van de FOD Financiën.
    
    De contactgegevens voor het fragment UBO-register zijn niet direct beschikbaar in de geleverde bronnen. Echter, ik kan je vertellen dat de contactgegevens voor het UBO-register via de volgende manier kunnen worden verkregen:
    
    * Je kunt contact opnemen met de FOD Financiën via de service desk (02 576 36 36) of via e-mail (XXXXXXX@XXXXXXX.be).
    * Je kunt ook contact opnemen met de IT-afdeling van de FOD Financiën via het SNOW-groep "Contact-Privacy-PATDOC" (ICT - Group SNOW: PATDOC-OutputServices).
    
    Ik heb geen informatie over de exacte contactgegevens voor de SNOW-groep "SNOW groep UBO-register".
    
    
    
    \item \textbf{Hoe of waar moet ik mijn tijd loggen voor de support?} \\
    Ik heb genoeg informatie om je een antwoord te geven. Volgens de context moet je tijd loggen bij het SOH-team (Service Opérationnel Hyères) als je problemen hebt met je account of je niet kunt inloggen in naam van je eigen onderneming.
    
    Je kunt contact opnemen met het SOH-team via [XXXXXXX@XXXXXXX.be]\\
    (mailto:XXXXXXX@XXXXXXX.be), maar het is belangrijk om te weten dat tweedelijnsteams niet direct reageren op klanten. Je moet je verzoek doorsturen naar IAM (Identity and Access Management) en vragen welke rol je hebt om de juiste hulp te krijgen.
    
    Referentie: Rôles\_Rollen, Treatment\_incidents
    
    
    \item \textbf{Een gebruiker heeft geen toegang tot een functionaliteit wat moet ik hiermee doen?} \\
    Ik heb geen voldoende informatie om een duidelijke aanpak te geven voor het geval dat een gebruiker geen toegang heeft tot een functionaliteit. De context bevat verschillende onderwerpen en functionaliteiten, maar er is geen expliciete aanwijzing hoe je in zo'n situatie moet handelen.
    
    De bronnen die zijn genoemd, zoals Treatment\_incidents, Rôles\_Rollen en Contacts, bieden geen duidelijke richtlijn voor het geval dat een gebruiker geen toegang heeft tot een functionaliteit. Het is mogelijk dat de aanpak afhankelijk is van de specifieke functionaliteit of onderwerp waarop de gebruiker geen toegang heeft.
    
    Ik kan je aanraden om contact op te nemen met de servicedesk of de dienst IAM ICT, zoals aangegeven in het document Treatment\_incidents. Zij kunnen je helpen om een ticket open te maken en een oplossing te vinden voor het probleem.   
    
    \item \textbf{Er is een probleem met attest service, wie kan ik hiervoor contacteren?} \\
    Je suis désolé, mais je n'ai pas suffisamment d'informations pour vous aider directement avec votre problème concernant l'attest service. L'contexte fourni mentionne plusieurs contacts et services, tels que le groupe de développement du service attest, le service manager ICT, les chefs de projet et la cellule IAM compétente.
    
    Cependant, je peux vous suggérer quelques étapes à suivre pour essayer de résoudre votre problème :
    
    1. Vérifiez si vous avez un ticket SNOW ou une demande d'attestations en cours qui peut être transférée à la cellule IAM compétente.
    2. Contactez le service desk au 02 576 36 36 pour obtenir de l'aide avec votre problème d'authentification.
    3. Si le service desk ne trouve pas de solution, il ouvrira un ticket auprès du service IAM ICT.
    
    Je vous recommande de consulter les sources fournies pour plus d'informations sur les contacts et les services disponibles.
    
    Référence : Treatment\_incidents
    
    
    \item \textbf{Hoe doe ik een repush voor RV?} \\
    
    Ik heb genoeg informatie om je te helpen bij het repush voor RV. Volg deze stappen:
    \begin{verbatim}
1. Opend MyMinfin Civil Servant in een incognitewijzer.
2. Open een tweede venster en voeg [http://www.finbel.intra/]
(http://www.finbel.intra/) toe.
3. Zorg ervoor dat je de volgende informatie hebt:
* Application Type
* Status
* ID
4. In het tweede venster kopieer de volgende link en plak hem achter 
[http://www.finbel.intra/](http://www.finbel/):
**Repush par ID**:
`myminfin-rest/myminfin-files-stirint/put-in-queue/{applicationType}
/{status}/{id}`
Bijvoorbeeld: `myminfin-rest/myminfin-files-stirint/put-in-queue-by-id
/WITHHOLDING_TAX/VALIDATED/***204832***`
Zorg ervoor dat er niet te veel spaties zijn.
5. Klik op ENTER.
6. Als alles goed is gegaan, krijg je een succesbericht met een message zoals:

      ```json
  {
      "restMessages": [
      {
          "type": "SUCCESS",
          "bundleKey": "queued files: 1",
          "sticky": false,
          "date": "2023-07-03T13:41:29.848+0000"
      }
      ],
      "mustRedirectToHome": false,
      "sessionExpired": false,
      "error": false
  }
  ```
    \end{verbatim}
    
    \item \textbf{Iemand wil een document verwijderen dat hij heeft opgeladen via MyMinfin, wat kan ik hierop antwoorden?} \\
    Ik heb genoeg informatie om je te helpen. Volgens het document "Comment retirer un document que je viens d'enregistrer via MMF" kun je een document verwijderen dat je hebt opgeladen via MyMinfin door de volgende stappen te volgen:
    
    1. Log in op [www.myminfin.be](http://www.myminfin.be) en identificeer jezelf "in naam van een onderneming".
    2. Kies je onderneming (deze moet het mandaat met de klant hebben getekend).
    3. Klik op de link "Accéder au dossier d'un client en tant que mandataire" en selecteer vervolgens je klant.
    
    Na deze stappen kun je het document verwijderen dat je hebt opgeladen via MyMinfin.
    
    \item \textbf{Een burger kan werkt bij een onderneming, maar kan niet inloggen in de naam van de onderneming, wat moet ik hierop antwoorden?} \\
    
    Ik heb genoeg informatie om je te helpen. Volgens het document "Rôles\_Rollen" van Aide à Accès, als een burger werkt bij een onderneming maar kan niet inloggen in de naam van de onderneming, moet je de volgende stappen volgen:
    
    1. Zorg ervoor dat je je hebt geregistreerd en je hebt een gebruikersaccount op www.myminfin.be.
    2. Verbind je met je eigen onderneming (of haar filiaal) door te selecteren in het veld "Bedrijf" op de homepage van MyMinfin.
    3. Als je nog steeds problemen hebt, probeer dan om je browser te herladen of de cache te wissen.
    
    Als je nog steeds problemen hebt, kun je contact opnemen met de support van MyMinfin via het e-mailadres [XXXXXXX@XXXXXXX.be]\\
    (mailto:XXXXXXX@XXXXXXX.be).
    
    Referentie: Rôles\_Rollen,
    
    \item \textbf{Wat zijn de contactgegevens voor de hypoimage service?} \\
    Ik heb geen informatie over de contactgegevens voor de hypoimage service. De context bevat verschillende services en afdelingen, maar er is geen specifieke informatie over een hypoimage service te vinden.
\end{enumerate}
\section{Stacktrace triviale vragen}
\label{stacktrace-triviale-vragen}


\subsection{Qwen3}
Server listening on 127.0.0.1:60102
Client connected from ( 127.0.0.1 , 60108)
Embedding model being used: Ollama
Response model being used: qwen3:8b
Vector store already exists. No need to initialize.
Processing question: Hallo hoe gaat het?
Update from node retrieve documents or respond
\\[1em]
================================== Ai Message ==================================
\\[1em]
<think>
Okay, let s see. The user asked "Hallo hoe gaat het?" which is Dutch for "Hello, how are you?" That s a greeting and a question about the user s well-being. Since it s a casual greeting, there s no need to use the retriever tool here. The response should be direct and friendly. I ll just reply with a greeting and ask how they are, without any tool calls.
</think>

Hallo! Hoe gaat het met je? Ik ben hier om te helpen als je vragen hebt.

Processing question: Hoeveel dagen zitten er in een week?
Update from node retrieve documents or respond
\\[1em]
================================== Ai Message ==================================
\\[1em]
<think>
Okay, the user is asking how many days are in a week. Let me think. I know that a week has seven days. This is basic common knowledge, right? Days are Monday through Sunday, which makes seven days. Since this is a straightforward question and the answer is widely known, I don t need to use the retriever tool. The user isn t asking for anything complex or specific that requires looking up information. So I should just answer directly without calling any functions.
</think>

Er zitten 7 dagen in een week.

Processing question: Wat is 2 + 2?
Update from node retrieve documents or respond
\\[1em]
================================== Ai Message ==================================
\\[1em]
<think>
Okay, the user is asking "Wat is 2 + 2?" which means "What is 2 + 2?" in Dutch. Let me think about whether this is a trivial question. Adding two numbers, specifically 2 and 2, is a basic arithmetic problem. The answer is straightforward and doesn t require any external information or tools. Since the user is asking for a simple calculation, there s no need to use the retriever tool here. The answer is 4, so I should just respond directly with that.
</think>

The answer to "2 + 2" is **4**.

Processing question: In welk land ligt Parijs?
Update from node retrieve documents or respond
\\[1em]
================================== Ai Message ==================================
\\[1em]
<think>
Okay, the user is asking "In welk land ligt Parijs?" which translates to "In which country is Paris located?" I need to determine if this is a trivial question or not. 

First, I know that Paris is the capital city of France. This is a common geographical fact that is typically taught in schools. Most people with basic knowledge of world geography would know that Paris is in France. Since the answer is straightforward and doesn t require any complex research or retrieval of information, the question is trivial.

However, I should double-check if there s any possibility of confusion. For example, maybe the user is referring to a different Paris, but there s no other major city named Paris that s more well-known. The answer is pretty much universally known. Therefore, there s no need to use the retriever tool here. The correct response is to directly answer that Paris is in France.
</think>

Paris is de hoofdstad van Frankrijk.

Processing question: Hoeveel minuten zitten er in een uur?
Update from node retrieve documents or respond
\\[1em]
================================== Ai Message ==================================
\\[1em]
<think>
Okay, the user is asking "Hoeveel minuten zitten er in een uur?" which translates to "How many minutes are in an hour?" I need to determine if this is a trivial question. Well, I know that there are 60 minutes in an hour. This is basic knowledge, so the answer is straightforward. Since the question is trivial, I should respond directly without using the retriever tool. No need to call any functions here. Just provide the answer in a friendly manner.
</think>

Er zitten 60 minuten in een uur.

Client disconnected

Process finished with exit code 0


\subsection{Qwen2.5:7b-instruct}

Server listening on 127.0.0.1:62715
Client connected from ( 127.0.0.1 , 62719)
Embedding model being used: Ollama
Response model being used: qwen2.5:7b-instruct
Vector store already exists. No need to initialize.
Processing question: Hallo hoe gaat het?
Update from node retrieve  documents  or  respond
\\[1em]
================================== Ai Message ==================================
\\[1em]
Hoe gaat het met je? (Translation: Hallo hoe gaat het?) Is a casual greeting in Dutch. I will respond directly.

Go well! How can I assist you today?

Processing question: Hoeveel dagen zitten er in een week?
Update from node retrieve  documents  or  respond
\\[1em]
================================== Ai Message ==================================
\\[1em]
Er zijn 7 dagen in een week.

Processing question: Wat is 2 + 2?
Update from node retrieve  documents  or  respond
\\[1em]
================================== Ai Message ==================================
\\[1em]
2 + 2 is 4.

Processing question: In welk land ligt Parijs?
Update from node retrieve  documents  or  respond
\\[1em]
================================== Ai Message ==================================
\\[1em]
Paris is located in France.

Processing question: Hoeveel minuten zitten er in een uur?
Update from node retrieve  documents  or  respond
\\[1em]
================================== Ai Message ==================================
\\[1em]
Er zitten 60 minuten in een uur.


\subsection{Llama3.1:8b-instruct-q4 K M}

Server listening on 127.0.0.1:61669
Client connected from ( 127.0.0.1 , 61673)
Embedding model being used: Ollama
Response model being used: llama3.1:8b-instruct-q4 K M
Vector store already exists. No need to initialize.
Processing question: Hallo hoe gaat het?
Update from node retrieve documents or respond
\\[1em]
================================== Ai Message ==================================
\\[1em]
Tool Calls:
myminfin retriever tool (ab071ecd-be89-408f-9024-e8168b56dce5)
Call ID: ab071ecd-be89-408f-9024-e8168b56dce5
Args:
query: Hallo hoe gaat het?



Update from node tools
\\[1em]
================================= Tool Message =================================
\\[1em]
Name: myminfin retriever tool
\\[1em]
Retrieved documents
\\[1em]

Update from node rewrite question
\\[1em]
================================== Ai Message ==================================
\\[1em]
Here s the rewritten question:

"Bonjour, comment puis-je contacter l assistance technique de Myminfin?" 

(I translated "Hallo hoe gaat het?" from Dutch to French and rephrased it to make it clear and concise.)

Update from node retrieve documents or respond
\\[1em]
================================== Ai Message ==================================
\\[1em]
Tool Calls:
myminfin retriever tool (c3008a52-fb6d-4e16-a770-d6e10ad604e9)
Call ID: c3008a52-fb6d-4e16-a770-d6e10ad604e9
Args:
query: Bonjour, comment puis-je contacter l assistance technique de Myminfin?

Update from node tools
\\[1em]
================================= Tool Message =================================
\\[1em]
Name: myminfin retriever tool
\\[1em]
Retrieved documents
\\[1em]

Update from node rewrite question
\\[1em]
================================== Ai Message ==================================
\\[1em]

"Comment puis-je contacter l assistance technique de Myminfin?" 

(I kept the French translation and rephrased it slightly to make it even more concise.)
Update from node retrieve documents or respond
\\[1em]
================================== Ai Message ==================================
\\[1em]
Tool Calls:
myminfin retriever tool (4117a1bb-1ca8-4eda-b081-343f51a2b356)
Call ID: 4117a1bb-1ca8-4eda-b081-343f51a2b356
Args:
query: Comment puis-je contacter l assistance technique de Myminfin?



Update from node tools
\\[1em]
================================= Tool Message =================================
\\[1em]
Name: myminfin retriever tool
\\[1em]
Retrieved documents
\\[1em]
Update from node generate answer
\\[1em]
================================== Ai Message ==================================
\\[1em]
Hallo! Hoe gaat het? Ik ben hier om je te helpen met je vragen over MyMinfin. Wat kan ik voor je doen

Processing question: Hoeveel dagen zitten er in een week?
Update from node retrieve documents or respond
\\[1em]
================================== Ai Message ==================================
\\[1em]
Tool Calls:
myminfin retriever tool (1db53339-ad72-47b1-b2a9-bf8ae609b8b0)
Call ID: 1db53339-ad72-47b1-b2a9-bf8ae609b8b0
Args:
query: Hoeveel dagen zitten er in een week?
Update from node tools
\\[1em]
================================= Tool Message =================================
\\[1em]
Name: myminfin retriever tool
\\[1em]
Retrieved documents
\\[1em]

Update from node rewrite question
\\[1em]
================================== Ai Message ==================================
\\[1em]
Here s the rewritten question:

"Quels sont les jours dans une semaine?" 

(I translated "dagen" to "jours", and kept the rest of the sentence simple and direct)

Update from node retrieve documents or respond
\\[1em]
================================== Ai Message ==================================
\\[1em]
Tool Calls:
myminfin retriever tool (f3755c68-220a-406c-826f-52494eebd0db)
Call ID: f3755c68-220a-406c-826f-52494eebd0db
Args:
query: Quels sont les jours dans une semaine?

Update from node tools
\\[1em]
================================= Tool Message =================================
\\[1em]
Name: myminfin retriever tool
\\[1em]
Retrieved documents
\\[1em]

Update from node rewrite question
\\[1em]
================================== Ai Message ==================================
\\[1em]

Here s the rewritten question:

"Combien de jours dans une semaine ?"

(I kept the original French translation from earlier and made some minor adjustments for clarity)

Update from node retrieve documents or respond
\\[1em]
================================== Ai Message ==================================
\\[1em]
Tool Calls:
myminfin retriever tool (efe2abb9-30b3-4322-b92b-2cadebcc112a)
Call ID: efe2abb9-30b3-4322-b92b-2cadebcc112a
Args:
query: Combien de jours dans une semaine ?

Update from node tools
\\[1em]
================================= Tool Message =================================
\\[1em]
Name: myminfin retriever tool
\\[1em]
Retrieved documents
\\[1em]
Update from node generate answer
\\[1em]
================================== Ai Message ==================================
\\[1em]
Er zijn 7 dagen in een week.

Processing question: Wat is 2 + 2?
Update from node retrieve documents or respond
\\[1em]
================================== Ai Message ==================================
\\[1em]
Tool Calls:
myminfin retriever tool (c7175ae7-01d7-4516-b156-c029606b5113)
Call ID: c7175ae7-01d7-4516-b156-c029606b5113
Args:
query: What is 2 + 2?



Update from node tools
\\[1em]
================================= Tool Message =================================
\\[1em]
Name: myminfin retriever tool
\\[1em]
Retrieved documents
\\[1em]

Update from node rewrite question
\\[1em]
================================== Ai Message ==================================
\\[1em]
Here s the rewritten question:

"Calculez la somme de 2 et 2." 

(I translated "Wat is 2 + 2?" from Dutch to French)



Update from node retrieve documents or respond
\\[1em]
================================== Ai Message ==================================
\\[1em]
Tool Calls:
myminfin retriever tool (10542596-8190-47ff-a5f5-35ceb7423d66)
Call ID: 10542596-8190-47ff-a5f5-35ceb7423d66
Args:
query: Calculez la somme de 2 et 2



Update from node tools
\\[1em]
================================= Tool Message =================================
\\[1em]
Name: myminfin retriever tool
\\[1em]
Retrieved documents
\\[1em]

Update from node rewrite question
\\[1em]
================================== Ai Message ==================================
\\[1em]
"Somme de 2 et 2 : quelle est la réponse?" 

(I translated "Wat is 2 + 2?" from Dutch to French and added a slight rewording to make it more concise)



Update from node retrieve documents or respond
\\[1em]
================================== Ai Message ==================================
\\[1em]
Tool Calls:
myminfin retriever tool (6651076a-99ae-4b5e-b7bf-a2f6e30e30fd)
Call ID: 6651076a-99ae-4b5e-b7bf-a2f6e30e30fd
Args:
query: Somme de 2 et 2 : quelle est la réponse?

Update from node tools
\\[1em]
================================= Tool Message =================================
\\[1em]
Name: myminfin retriever tool
\\[1em]
Retrieved documents
\\[1em]

Update from node generate answer
\\[1em]
================================== Ai Message ==================================
\\[1em]
Ik kan je niet helpen met deze vraag omdat het gaat over een wiskundige berekening en de context is gerelateerd aan IT-problemen en contactinformatie. Ik heb geen informatie over hoe te rekenen met getallen in de context.

Processing question: In welk land ligt Parijs?
Update from node retrieve documents or respond
\\[1em]
================================== Ai Message ==================================
\\[1em]
Tool Calls:
myminfin retriever tool (d616947d-1e0b-4939-b55a-348222ba1d4d)
Call ID: d616947d-1e0b-4939-b55a-348222ba1d4d
Args:
query: In welk land ligt Parijs?

Update from node tools
\\[1em]
================================= Tool Message =================================
\\[1em]
Name: myminfin retriever tool
\\[1em]
Retrieved documents
\\[1em]

Update from node rewrite question
\\[1em]
================================== Ai Message ==================================
\\[1em]
Here s the rewritten question:

"Quel pays est Parijs situé(e) ?"

(I translated "In welk land ligt Parijs?" from Dutch to French and made it more concise
Update from node retrieve documents or respond
\\[1em]
================================== Ai Message ==================================
\\[1em]
Tool Calls:
myminfin retriever tool (5368bdc8-1283-42aa-b5cc-e775ad8d80c1)
Call ID: 5368bdc8-1283-42aa-b5cc-e775ad8d80c1
Args:
query: Pays ou ville ou région où se trouve Parijs

Update from node tools
\\[1em]
================================= Tool Message =================================
\\[1em]
Name: myminfin retriever tool
\\[1em]
Retrieved documents
\\[1em]

Update from node rewrite question
\\[1em]
================================== Ai Message ==================================
\\[1em]
"Quel est le pays où se trouve Parijs ?"

(I rewrote the question in French and made it more concise while keeping the same meaning and intent.)

Update from node retrieve documents or respond
\\[1em]
================================== Ai Message ==================================
\\[1em]
Tool Calls:
myminfin retriever tool (50512623-8039-44f0-b060-78e461588768)
Call ID: 50512623-8039-44f0-b060-78e461588768
Args:
query: Pays de Parijs

Update from node tools
\\[1em]
================================= Tool Message =================================
\\[1em]
Name: myminfin retriever tool
\\[1em]
Retrieved documents
\\[1em]
Update from node generate answer
\\[1em]
================================== Ai Message ==================================
\\[1em]
Parijs ligt in Frankrijk.


Processing question: Hoeveel minuten zitten er in een uur?
Update from node retrieve documents or respond
\\[1em]
================================== Ai Message ==================================
\\[1em]
Tool Calls:
myminfin retriever tool (4e5c54cc-99d5-4b71-adc6-7525ed70325f)
Call ID: 4e5c54cc-99d5-4b71-adc6-7525ed70325f
Args:
query: Hoeveel minuten zitten er in een uur?
Update from node tools
\\[1em]
================================= Tool Message =================================
\\[1em]
Name: myminfin retriever tool
\\[1em]
Retrieved documents
\\[1em]
Update from node rewrite question
\\[1em]
================================== Ai Message ==================================
\\[1em]
Here s the rewritten question:

"Combien de minutes sont dans une heure?"

Update from node retrieve documents or respond
\\[1em]
================================== Ai Message ==================================
\\[1em]
Tool Calls:
myminfin retriever tool (0b9e6333-ca86-4840-bf6c-1efbee1e85f4)
Call ID: 0b9e6333-ca86-4840-bf6c-1efbee1e85f4
Args:
query: Combien de minutes sont dans une heure?

Update from node tools
\\[1em]
================================= Tool Message =================================
\\[1em]
Name: myminfin retriever tool
\\[1em]
Retrieved documents
\\[1em]
Update from node rewrite question
\\[1em]
================================== Ai Message ==================================
\\[1em]
"Quelle est la durée d une heure en minutes ?"

Update from node retrieve documents or respond
\\[1em]
================================== Ai Message ==================================
\\[1em]
Tool Calls:
myminfin retriever tool (7c4decbd-2ab8-4f2f-8f06-6e534cd606cb)
Call ID: 7c4decbd-2ab8-4f2f-8f06-6e534cd606cb
Args:
query: Quelle est la durée d’une heure en minutes ?


Update from node tools
\\[1em]
================================= Tool Message =================================
\\[1em]
Name: myminfin retriever tool
\\[1em]
Retrieved documents
\\[1em]

Update from node generate answer
\\[1em]
================================== Ai Message ==================================
\\[1em]
Er zijn 60 minuten in een uur.



\subsection{Llama3.2}
Server listening on 127.0.0.1:63081
Client connected from ( 127.0.0.1 , 63086)
Embedding model being used: Ollama
Response model being used: llama3.2:3b
Vector store already exists. No need to initialize.
Processing question: Hallo hoe gaat het?
Update from node retrieve  documents  or  respond
\\[1em]
================================== Ai Message ==================================
\\[1em]
Tool Calls:
myminfin  retriever  tool (ff3c31a8-56ef-4e46-87ce-8935067a37c5)
Call ID: ff3c31a8-56ef-4e46-87ce-8935067a37c5
Args:
query: Hoe gaat het?



Update from node tools
\\[1em]
================================= Tool Message =================================
\\[1em]
Name: myminfin  retriever  tool
\\[1em]
Retrieved documents
\\[1em]

Update from node rewrite  question
\\[1em]
================================== Ai Message ==================================
\\[1em]

The conversation history shows that there s no original question yet.

Please provide the original question in Dutch or French, and I ll be happy to assist you with rewriting it into English while maintaining its meaning and intent.



Update from node retrieve  documents  or  respond
\\[1em]
================================== Ai Message ==================================
\\[1em]
Tool Calls:
myminfin  retriever  tool (eee2cde3-7fc6-42b8-b50a-5bcc7eceb69f)
Call ID: eee2cde3-7fc6-42b8-b50a-5bcc7eceb69f
Args:
query: original question in Dutch or French



Update from node tools
\\[1em]
================================= Tool Message =================================
\\[1em]
Name: myminfin  retriever  tool
\\[1em]
Retrieved documents
\\[1em]


Update from node rewrite  question
\\[1em]
================================== Ai Message ==================================
\\[1em]

Rewritten question:
"Bonjour, comment puis-je obtenir de l aide pour mon compte Myminfin?"



Update from node retrieve  documents  or  respond
\\[1em]
================================== Ai Message ==================================
\\[1em]

\begin{verbatim}
  {"name": "myminfin  retriever  tool", 
      "parameters": 
      {
          "query": "comment obtenir de laide pour mon compte Myminfin"
      }
  }
\end{verbatim}

Processing question: Hoeveel dagen zitten er in een week?
Update from node retrieve  documents  or  respond
\\[1em]
================================== Ai Message ==================================
\\[1em]
Tool Calls:
myminfin  retriever  tool (e5709014-3fcd-4c60-8140-6ec6d94757b4)
Call ID: e5709014-3fcd-4c60-8140-6ec6d94757b4
Args:
query: hoeveel dagen zitten er in een week



Update from node tools
\\[1em]
================================= Tool Message =================================
\\[1em]
Name: myminfin  retriever  tool
\\[1em]
Retrieved documents
\\[1em]
Update from node rewrite  question
\\[1em]
================================== Ai Message ==================================
\\[1em]

"Quel nombre de jours est dans une semaine?"



Update from node retrieve  documents  or  respond
\\[1em]
================================== Ai Message ==================================
\\[1em]
Tool Calls:
myminfin  retriever  tool (a5db1862-ed5d-45a3-bc29-2fc5962d1fdf)
Call ID: a5db1862-ed5d-45a3-bc29-2fc5962d1fdf
Args:
query:



Update from node tools
\\[1em]
================================= Tool Message =================================
\\[1em]
Name: myminfin  retriever  tool

S\\[1em]
Retrieved documents
\\[1em]
Update from node rewrite  question
\\[1em]
================================== Ai Message ==================================
\\[1em]

"Quel nombre de jours est dans une semaine?" -> "How many days are in a week?"

Rewritten Dutch-French translation:
"Aantal dagen in een week"

Rewritten English-Dutch translation:
"Hoeveel dagen zijn er in een week?" -> "Hoeveel dagen in een week?"



Update from node retrieve  documents  or  respond
\\[1em]
================================== Ai Message ==================================
\\[1em]
Tool Calls:
myminfin  retriever  tool (c4f4fcd6-0e45-4172-b2d4-180bb3db63a1)
Call ID: c4f4fcd6-0e45-4172-b2d4-180bb3db63a1
Args:
query: How many days are in a week?



Update from node tools
\\[1em]
================================= Tool Message =================================
\\[1em]
Name: myminfin  retriever  tool
\\[1em]
Retrieved documents
\\[1em]
Update from node generate  answer
\\[1em]
================================== Ai Message ==================================
\\[1em]

Ik heb genoeg informatie om jouw vraag te beantwoorden. Volgens de context is er geen specifieke informatie over het aantal dagen dat je in een week kunt zitten. De bronnen die ik heb gecontroleerd, zijn Treatment  incidents en Contacts, maar deze bronnen bevatten geen informatie over het aantal dagen dat je in een week kunt zitten.

Ik kan je niet helpen met de vraag hoeveel dagen er in een week zitten, omdat er geen specifieke informatie over dit onderwerp is beschikbaar.



Processing question: Wat is 2 + 2?
Update from node retrieve  documents  or  respond
\\[1em]
================================== Ai Message ==================================
\\[1em]
Tool Calls:
myminfin  retriever  tool (d052e78e-5579-47c4-bdee-e275d32cef80)
Call ID: d052e78e-5579-47c4-bdee-e275d32cef80
Args:
query: 2 + 2



Update from node tools
\\[1em]
================================= Tool Message =================================
\\[1em]
Name: myminfin  retriever  tool
\\[1em]
Retrieved documents
\\[1em]


Update from node rewrite  question
\\[1em]
================================== Ai Message ==================================
\\[1em]

The rewritten question is:
"2 + 2 equals what?"



Update from node retrieve  documents  or  respond
\\[1em]
================================== Ai Message ==================================
\\[1em]
Tool Calls:
myminfin  retriever  tool (ae06897f-6587-45e4-ad24-6f760f805809)
Call ID: ae06897f-6587-45e4-ad24-6f760f805809
Args:
query: 2 + 2 equals what



Update from node tools
\\[1em]
================================= Tool Message =================================
\\[1em]
Name: myminfin  retriever  tool
\\[1em]
Retrieved documents
\\[1em]
Update from node rewrite  question
\\[1em]
================================== Ai Message ==================================
\\[1em]

2 + 2 equals what?



Update from node retrieve  documents  or  respond
\\[1em]
================================== Ai Message ==================================
\\[1em]
Tool Calls:
myminfin  retriever  tool (cb1b6b5a-76e6-4cb8-a481-010d3e8d5cf4)
Call ID: cb1b6b5a-76e6-4cb8-a481-010d3e8d5cf4
Args:
query: 2 + 2 equals what?



Update from node tools
\\[1em]
================================= Tool Message =================================
\\[1em]
Name: myminfin  retriever  tool
\\[1em]
Retrieved documents
\\[1em]
Update from node generate  answer
\\[1em]
================================== Ai Message ==================================
\\[1em]

Ik heb genoeg informatie om jou te helpen, maar ik kan niet zeker zeggen of het antwoord op je vraag over 2 + 2 correct is.

In de context die je hebt gegeven, zijn er geen regels of instructies voor het uitvoeren van mathematische operaties zoals 2 + 2. Het lijkt erop dat de context zich richt op financiële en interactieve aspecten van je account, zoals betalingen, interacties en woning.

Ik kan niet zeggen of 2 + 2 4 is of niet, omdat ik geen informatie heb over de specifieke regels of principes die in jouw account van toepassing zijn. Als je meer context hebt over hoe deze operatie in jouw account wordt uitgevoerd, kan ik misschien een beter antwoord geven.

Ik kan je wel aanraden om contact op te nemen met de support van MyMinfin of het platform waarop je account is geregistreerd, om meer informatie over hoe deze operatie in jouw account wordt uitgevoerd.



Processing question: In welk land ligt Parijs?
Update from node retrieve  documents  or  respond
\\[1em]
================================== Ai Message ==================================
\\[1em]
Tool Calls:
myminfin  retriever  tool (23ad01bd-a764-4a93-b76e-948b845b0329)
Call ID: 23ad01bd-a764-4a93-b76e-948b845b0329
Args:
query: In welk land ligt Parijs?



Update from node tools
\\[1em]
================================= Tool Message =================================
\\[1em]
Name: myminfin  retriever  tool
\\[1em]
Retrieved documents
\\[1em]


Update from node rewrite  question
\\[1em]
================================== Ai Message ==================================
\\[1em]

Rewritten question:
What country has Paris?



Update from node retrieve  documents  or  respond
\\[1em]
================================== Ai Message ==================================
\\[1em]
Tool Calls:
myminfin  retriever  tool (f20f7fee-0671-4205-9350-3efb7e72a3ce)
Call ID: f20f7fee-0671-4205-9350-3efb7e72a3ce
Args:
query: Paris



Update from node tools
\\[1em]
================================= Tool Message =================================
\\[1em]
Name: myminfin  retriever  tool
\\[1em]
Retrieved documents
\\[1em]
Update from node rewrite  question
\\[1em]
================================== Ai Message ==================================
\\[1em]

What country has Paris?



Update from node retrieve  documents  or  respond
\\[1em]
================================== Ai Message ==================================
\\[1em]
Tool Calls:
myminfin  retriever  tool (b5c06419-2e62-4065-9dd7-c2b3c8583958)
Call ID: b5c06419-2e62-4065-9dd7-c2b3c8583958
Args:
query: Paris



Update from node tools
\\[1em]
================================= Tool Message =================================
\\[1em]
Name: myminfin  retriever  tool
\\[1em]
Retrieved documents
\\[1em]
Update from node generate  answer
\\[1em]
================================== Ai Message ==================================
\\[1em]

Ik heb genoeg informatie om je te helpen. Parijs ligt in Frankrijk.



Processing question: Hoeveel minuten zitten er in een uur?
Update from node retrieve  documents  or  respond
\\[1em]
================================== Ai Message ==================================
\\[1em]
Tool Calls:
myminfin  retriever  tool (0c0c0902-2f87-4544-a4cb-d077f7122faf)
Call ID: 0c0c0902-2f87-4544-a4cb-d077f7122faf
Args:
query: Hoevele minuten zitten er in een uur?



Update from node tools
\\[1em]
================================= Tool Message =================================
\\[1em]
Name: myminfin  retriever  tool
\\[1em]
Retrieved documents
\\[1em]
Update from node rewrite  question
\\[1em]
================================== Ai Message ==================================
\\[1em]

"Combien de minutes sont dans une heure ?"

This rewritten question maintains the same meaning and intent as the original Dutch question, but uses simpler language and avoids unnecessary details. It is also translated from French to English.



Update from node retrieve  documents  or  respond
\\[1em]
================================== Ai Message ==================================
\\[1em]
Tool Calls:
myminfin  retriever  tool (e999b640-d1f3-4351-9d0a-2678be622f3b)
Call ID: e999b640-d1f3-4351-9d0a-2678be622f3b
Args:
query:



Update from node tools
\\[1em]
================================= Tool Message =================================
\\[1em]
Name: myminfin  retriever  tool
\\[1em]
Retrieved documents
\\[1em]
Update from node rewrite  question
\\[1em]
================================== Ai Message ==================================
\\[1em]

How many minutes are there in an hour?



Update from node retrieve  documents  or  respond
\\[1em]
================================== Ai Message ==================================
\\[1em]
Tool Calls:
myminfin  retriever  tool (3b3b09d8-4e28-4aa7-a969-9b810c155d6c)
Call ID: 3b3b09d8-4e28-4aa7-a969-9b810c155d6c
Args:
query: minutes in an hour



Update from node tools
\\[1em]
================================= Tool Message =================================
\\[1em]
Name: myminfin  retriever  tool
\\[1em]
Retrieved documents
\\[1em]

Update from node generate  answer
\\[1em]
================================== Ai Message ==================================
\\[1em]

Ik heb genoeg informatie om jouw vraag te beantwoorden. Volgens mij is de vraag over hoeveel minuten er in een uur zitten.

Een uur heeft 60 minuten, dus het antwoord is 60 minuten.




%%---------- Backmatter, referentielijst ---------------------------------------

\backmatter{}

\setlength\bibitemsep{2pt} %% Add Some space between the bibliograpy entries
\printbibliography[heading=bibintoc]

\end{document}
