%%=============================================================================
%% Samenvatting
%%=============================================================================

% TODO: De "abstract" of samenvatting is een kernachtige (~ 1 blz. voor een
% thesis) synthese van het document.
%
% Een goede abstract biedt een kernachtig antwoord op volgende vragen:
%
% 1. Waarover gaat de bachelorproef?
% 2. Waarom heb je er over geschreven?
% 3. Hoe heb je het onderzoek uitgevoerd?
% 4. Wat waren de resultaten? Wat blijkt uit je onderzoek?
% 5. Wat betekenen je resultaten? Wat is de relevantie voor het werkveld?
%
% Daarom bestaat een abstract uit volgende componenten:
%
% - inleiding + kaderen thema
% - probleemstelling
% - (centrale) onderzoeksvraag
% - onderzoeksdoelstelling
% - methodologie
% - resultaten (beperk tot de belangrijkste, relevant voor de onderzoeksvraag)
% - conclusies, aanbevelingen, beperkingen
%
% LET OP! Een samenvatting is GEEN voorwoord!

%%---------- Samenvatting -----------------------------------------------------
% De samenvatting in de hoofdtaal van het document

\chapter*{\IfLanguageName{dutch}{Samenvatting}{Abstract}}

% TODO dit is de samenvatting van het voorstel het moet verder aangevuld worden wanneer het onderzoek verder staat
Dit proces kan echter veel tijd en middelen vergen, vooral binnen grote organisaties. Het is vaak een uitdaging
om snel en adequaat antwoorden te bieden op vragen van klanten, wat resulteert in een aanzienlijke investering
van resources. Tegelijkertijd verwachten klanten een snelle oplossing voor hun problemen. Het is daarom in het
belang van zowel de organisatie als de klant om vragen efficiënt te beantwoorden.
Deze bachelorproef onderzoekt de mogelijkheden voor het ontwikkelen van een virtuele assistent die supportmedewerkers
kan ondersteunen bij het vinden van relevante antwoorden. Door middel van interviews met betrokkenen
wordt een analyse gemaakt van het huidige proces, met als doel de pijnpunten in het verwerken van
supporttickets in kaart te brengen. Daarnaast worden via een grondige literatuurstudie de verschillende opties
voor het inzetten van een virtuele supportassistent onderzocht. Het einddoel is het ontwikkelen van een proof of
concept dat bijdraagt aan een efficiëntere verwerking van klantvragen en de ondersteuning van medewerkers
bij het oplossen van deze problemen.
Het verwachte resultaat van dit onderzoek omvat enerzijds een overzicht van de mogelijkheden van een virtuele
supportassistent, met aandacht voor wat praktisch haalbaar is en welke factoren daarbij een rol spelen. Anderzijds
wordt op basis van een concrete casus een toepassing ontwikkeld in de vorm van een proof of concept (PoC).
Er wordt verwacht dat dergelijke virtuele assistenten in veel gevallen een meerwaarde kunnen bieden, maar niet
voor alle bedrijven. Afhankelijk van de specifieke behoeften en de manier waarop een organisatie een virtuele
assistent wil inzetten, moet eerst grondig worden geanalyseerd of een dergelijke implementatie daadwerkelijk
waarde toevoegt. Pas na een dergelijke analyse kan overwogen worden om een virtuele supportassistent te
implementeren.
