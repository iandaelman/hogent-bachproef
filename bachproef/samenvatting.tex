%%=============================================================================
%% Samenvatting
%%=============================================================================

% TODO: De "abstract" of samenvatting is een kernachtige (~ 1 blz. voor een
% thesis) synthese van het document.
%
% Een goede abstract biedt een kernachtig antwoord op volgende vragen:
%
% 1. Waarover gaat de bachelorproef?
% 2. Waarom heb je er over geschreven?
% 3. Hoe heb je het onderzoek uitgevoerd?
% 4. Wat waren de resultaten? Wat blijkt uit je onderzoek?
% 5. Wat betekenen je resultaten? Wat is de relevantie voor het werkveld?
%
% Daarom bestaat een abstract uit volgende componenten:
%
% - inleiding + kaderen thema
% - probleemstelling
% - (centrale) onderzoeksvraag
% - onderzoeksdoelstelling
% - methodologie
% - resultaten (beperk tot de belangrijkste, relevant voor de onderzoeksvraag)
% - conclusies, aanbevelingen, beperkingen
%
% LET OP! Een samenvatting is GEEN voorwoord!

%%---------- Samenvatting -----------------------------------------------------
% De samenvatting in de hoofdtaal van het document

\chapter {Samenvatting}
\label{ch:samenvatting}

Deze bachelorproef onderzoekt de mogelijkheden om een LLM-gebaseerde chatbot te ontwikkelen die IT-supportmedewerkers kan bijstaan bij het behandelen van technische IT vragen. Gezien het belang van een goed functionerend supportproces, zowel voor de eindgebruiker als voor het team dat de IT-vragen afhandelt, is het noodzakelijk om dit proces zo efficiënt mogelijk te organiseren. In deze specifieke use case betekent dit dat ontwikkelaars van het MyMinfin-team potentieel minder tijd hoeven te besteden aan support en meer tijd kunnen vrijmaken voor de ontwikkeling van de website.
\\[1em]
De centrale onderzoeksvraag luidt: “Welke methode, Retrieval Augmented Generation (RAG), Finetuning of Cache-Augmented Generation (CAG), is het meest geschikt voor een LLM-gebaseerde IT-support chatbot binnen de context van de FOD Financiën?” Om deze vraag te beantwoorden werd eerst een literatuurstudie uitgevoerd waarin de voor- en nadelen van de drie benaderingen werden geanalyseerd. Uit deze analyse bleek RAG de meest geschikte methode voor deze specifieke use case.
\\[1em]
Op basis van deze keuze werd een Proof of Concept (PoC) ontwikkeld die RAG in de praktijk implementeert. Aangezien de PoC gebruik maakt van een LLM-model werd aan deze PoC een vergelijkende studie gekoppeld. Op die manier kon worden bepaald welk LLM-model het best presteert voor deze specifieke toepassing. Gezien de scope van dit onderzoek en de functionele vereisten waaraan de PoC moest voldoen, konden niet alle modellen worden opgenomen in de vergelijkende studie. Rekening houdend met deze beperking zijn in de vergelijkende studie de modellen Llama3.1, Llama3.2, Qwen2.5 en Qwen3 getest.
\\[1em]
De resultaten van het onderzoek tonen aan dat het Qwen3-model het beste heeft gepresteerd. In alle drie de uitgevoerde testfases behoorde dit model telkens bij de best presterende modellen. De ontwikkelde PoC biedt het MyMinfin-team een concrete basis om LLM-gebaseerde ondersteuning in het IT-supportproces te verkennen en verder te optimaliseren. Hiermee kan het team ook andere, mogelijk meer performante modellen evalueren en bepalen welke modellen het meest geschikt zijn voor het bieden van deze ondersteuning. Op deze manier draagt deze bachelorproef bij aan een efficiënter en meer betrouwbaar IT-supportproces binnen het MyMinfin-team.