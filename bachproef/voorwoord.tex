%%=============================================================================
%% Voorwoord
%%=============================================================================

\chapter{Woord vooraf}
\label{ch:voorwoord}
%% TODO:
%% Het voorwoord is het enige deel van de bachelorproef waar je vanuit je
%% eigen standpunt (``ik-vorm'') mag schrijven. Je kan hier bv. motiveren
%% waarom jij het onderwerp wil bespreken.
%% Vergeet ook niet te bedanken wie je geholpen/gesteund/... heeft

Deze bachelorproef is het slotstuk van een opleiding die ik bijna vijf jaar geleden begon als werkstudent. Het markeert niet alleen het einde van mijn studies, maar ook van een periode die veel heeft veranderd in mijn leven. Zowel op academisch als professioneel vlak heeft deze opleiding mij enorm veel bijgebracht en nieuwe inzichten gegeven. Het onderwerp van dit eindwerk sluit daar mooi op aan, omdat het rechtstreeks voortkomt uit mijn eigen werkervaring.
\\[1em]
Tijdens mijn dagelijkse werkzaamheden merkte ik samen met mijn collega’s dat het supportproces efficiënter kon worden georganiseerd. Deze behoefte vormde het startpunt voor dit onderzoek. Met de snelle opkomst van LLM-toepassingen zag ik een mooie kans om deze technologie te verkennen en te kijken welke mogelijkheden er zijn om een IT-supportbot te ontwikkelen die hierin kan ondersteunen.
\\[1em]
Deze bachelorproef was er nooit gekomen zonder de steun van mijn promotor, Thomas Aelbrecht, en mijn co-promotor, Koen Mekers. Hun hulp, advies en bereidheid om mee te denken hebben voor mij een wereld van verschil gemaakt. Daarom wil ik hen op deze plek nog eens heel oprecht bedanken.
\\[1em]
Met dit werk sluit ik een bijzonder hoofdstuk af en kijk ik met veel enthousiasme uit naar wat er nog komt.




