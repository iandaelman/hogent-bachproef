%%=============================================================================
%% Voorwoord
%%=============================================================================

\chapter*{\IfLanguageName{dutch}{Woord vooraf}{Preface}}%
\label{ch:voorwoord}

%% TODO:
%% Het voorwoord is het enige deel van de bachelorproef waar je vanuit je
%% eigen standpunt (``ik-vorm'') mag schrijven. Je kan hier bv. motiveren
%% waarom jij het onderwerp wil bespreken.
%% Vergeet ook niet te bedanken wie je geholpen/gesteund/... heeft

% TODO verder uitwerken van het voorwoord dit is louter een kladversie
De wereld verandert in hoog tempo, ook op het gebied van kunstmatige intelligentie. Dit biedt talloze mogelijkheden om bestaande processen kritisch te evalueren en te onderzoeken waar efficiëntiewinsten te behalen zijn. Om die reden heb ik besloten nader te onderzoeken hoe generatieve AI kan worden ingezet om de processen binnen mijn huidige werkomgeving te verbeteren. In het bijzonder wilde ik nagaan hoe we de supporttaken, die momenteel door het developmentteam van MyMinfin worden uitgevoerd, kunnen vergemakkelijken met behulp van generatieve AI.

Tevens maak ik van deze gelegenheid gebruik om mijn promotor, meneer Aelbrecht, en mijn co-promotor, Koen Meker, hartelijk te bedanken voor hun vele steun en waardevolle hulp tijdens het opstellen van deze bachelorproef.