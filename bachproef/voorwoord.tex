%%=============================================================================
%% Voorwoord
%%=============================================================================

\chapter{Woord vooraf}
\label{ch:voorwoord}
%TODO taalcheck
Deze bachelorproef vormt het finale eindstuk van een opleiding die ik nu bijna 5 jaar geleden ben begonnen als werkstudent. Naast een eindwerk is dit dus ook het einde van een opleiding die heel wat zaken heeft verandert in mijn leven zowel academisch als professioneel heeft deze opleiding me enorm veel bijgebracht.
\\[1em]
De keuze om te schrijven rond het maken van een IT-supportbot volgde uit de vraag en nood van de mensen met wie ik dagdagelijks werk om dit op een meer efficiënte manier aan te pakken. Met de sterke opkomst van LLM-toepassingen leek me dit de uitgelezen kans om hier mee op de kar te springen en te bekijken welke mogelijkheden er waren om hier iets rond te maken.
\\[1em]
Uiteraard was dit echter niet mogelijk geweest zonder de ondersteuning van mijn promotor Thomas Aelbrecht en mijn co-promotor Koen Mekers die mij hebben bijgestaan met zeer veel hulp en advies zonder wie het niet mogelijk was geweest om dit tot een goed einde te brengen. Ik wil hen op deze manier dus nogmaals uitdrukkelijk bedanken.


%% TODO:
%% Het voorwoord is het enige deel van de bachelorproef waar je vanuit je
%% eigen standpunt (``ik-vorm'') mag schrijven. Je kan hier bv. motiveren
%% waarom jij het onderwerp wil bespreken.
%% Vergeet ook niet te bedanken wie je geholpen/gesteund/... heeft
