%%=============================================================================
%% Methodologie
%%=============================================================================

\chapter{\IfLanguageName{dutch}{Methodologie}{Methodology}}
\label{ch:methodologie}

%% TODO: In dit hoofstuk geef je een korte toelichting over hoe je te werk bent
%% gegaan. Verdeel je onderzoek in grote fasen, en licht in elke fase toe wat
%% de doelstelling was, welke deliverables daar uit gekomen zijn, en welke
%% onderzoeksmethoden je daarbij toegepast hebt. Verantwoord waarom je
%% op deze manier te werk gegaan bent.
%% 
%% Voorbeelden van zulke fasen zijn: literatuurstudie, opstellen van een
%% requirements-analyse, opstellen long-list (bij vergelijkende studie),
%% selectie van geschikte tools (bij vergelijkende studie, "short-list"),
%% opzetten testopstelling/PoC, uitvoeren testen en verzamelen
%% van resultaten, analyse van resultaten, ...
%%
%% !!!!! LET OP !!!!!
%%
%% Het is uitdrukkelijk NIET de bedoeling dat je het grootste deel van de corpus
%% van je bachelorproef in dit hoofstuk verwerkt! Dit hoofdstuk is eerder een
%% kort overzicht van je plan van aanpak.
%%
%% Maak voor elke fase (behalve het literatuuronderzoek) een NIEUW HOOFDSTUK aan
%% en geef het een gepaste titel.


\section{Literatuurstudie}

Voor de opstart van deze bachelorproef is een grondige literatuurstudie essentieel om de bestaande mogelijkheden op het gebied van virtuele assistenten te verkennen, met een focus op LLM, RAG en CAG. Eerst wordt er een inleiding gegeven in de onderliggende materie, zodat het onderwerp voldoende gekaderd kan worden. Daarnaast richt deze studie zich op een vergelijkende analyse die inzicht biedt in de verschillende beschikbare opties voor het ontwikkelen van een virtuele supportassistent. Hierbij worden de voor- en nadelen van elke optie in kaart gebracht, zodat op basis van deze informatie een onderbouwde keuze kan worden gemaakt voor de uitwerking van een PoC. De literatuurstudie moet ook duidelijkheid verschaffen over de benodigde hardware- en softwarevereisten voor de ontwikkeling.

Op basis van de literatuurstudie moet een analyse worden opgesteld om na te gaan welke optie die werden besproken binnen de literatuurstudie het best in overweging worden genomen. De belangrijkste keuze die zal moeten gemaakt worden binnen deze use case is de keuze tussen een RAG implementatie of een CAG implementatie. Aangezien beide voor- en nadelen hebben moet dit weloverwogen worden voor het opstellen van de PoC en het LLM modellen die kunnen worden gebruikt voor de PoC.


\section{Requirement analyse}


Gekoppeld aan de literatuurstudie zullen interviews worden afgenomen. Het doel van deze interviews is om een helder overzicht te verkrijgen van de verschillende vereisten voor de virtuele assistent. Op basis van het overzicht uit de literatuurstudie kan vervolgens een meer gerichte selectie worden gemaakt van de verschillende opties. Het uiteindelijke doel is om voor twee à drie modellen een PoC uit te werken die verder getest kunnen worden. Het is met andere woorden van belang om de verschillende modellen zo uitgebreid mogelijk te onderzoeken, zodat zoveel mogelijk modellen getoetst kunnen worden aan de gevraagde vereisten.Eens de literatuurstudie is afgerond en de interviews zijn afgenomen, kan met behulp van een MoSCoW-analyse een rangschikking worden opgesteld van de verschillende beschikbare modellen. Deze rangschikking bepaalt welke modellen worden geselecteerd voor het uitwerken van een PoC.

\section{Long list}

\section{Short list}

\section{Opstellen PoC}

\section{Uitvoeren testen en analyse resultaten}

Elk van de geselecteerde modellen wordt vervolgens uitgewerkt in een PoC. Zodra de verschillende PoC’s beschikbaar zijn, wordt een vergelijking gemaakt tussen de modellen. Hierbij worden de volgende aspecten getest:

\begin{itemize} 
    \item Wat is de kwaliteit van de antwoorden volgens wetenschappelijke meetstaven? 
    \item Welk model verkiest de gebruiker. 
\end{itemize}

De kwaliteit van de gegenereerde antwoorden zal worden geëvalueerd aan de hand van ROUGE, BLEU en BERTScore. Aangezien deze metrics elk een ander aspect van tekstkwaliteit belichten, zoals tekstuele overeenkomst en semantische gelijkenis, is het waardevol om ze alle drie in beschouwing te nemen en geen enkele vooraf uit te sluiten. Daarnaast zullen de antwoorden worden voorgelegd aan enkele gebruikers, zodat zij op basis van realistische cases hun voorkeur kunnen uitspreken. Dit laat toe om, naast de automatische evaluatie, ook subjectieve aspecten zoals natuurlijkheid en relevantie mee te nemen.


