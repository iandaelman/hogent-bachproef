%%=============================================================================
%% Methodologie
%%=============================================================================

\chapter{\IfLanguageName{dutch}{Methodologie}{Methodology}}
\label{ch:methodologie}

%% TODO: In dit hoofstuk geef je een korte toelichting over hoe je te werk bent
%% gegaan. Verdeel je onderzoek in grote fasen, en licht in elke fase toe wat
%% de doelstelling was, welke deliverables daar uit gekomen zijn, en welke
%% onderzoeksmethoden je daarbij toegepast hebt. Verantwoord waarom je
%% op deze manier te werk gegaan bent.
%% 
%% Voorbeelden van zulke fasen zijn: literatuurstudie, opstellen van een
%% requirements-analyse, opstellen long-list (bij vergelijkende studie),
%% selectie van geschikte tools (bij vergelijkende studie, "short-list"),
%% opzetten testopstelling/PoC, uitvoeren testen en verzamelen
%% van resultaten, analyse van resultaten, ...
%%
%% !!!!! LET OP !!!!!
%%
%% Het is uitdrukkelijk NIET de bedoeling dat je het grootste deel van de corpus
%% van je bachelorproef in dit hoofstuk verwerkt! Dit hoofdstuk is eerder een
%% kort overzicht van je plan van aanpak.
%%
%% Maak voor elke fase (behalve het literatuuronderzoek) een NIEUW HOOFDSTUK aan
%% en geef het een gepaste titel.


\section{Literatuurstudie}

Voor de opstart van deze bachelorproef is een grondige literatuurstudie essentieel om de bestaande mogelijkheden op het gebied van virtuele assistenten te verkennen, met een focus op LLM, RAG en CAG. Eerst wordt er een inleiding gegeven in de onderliggende materie, zodat het onderwerp voldoende gekaderd kan worden. Daarnaast richt deze studie zich op een vergelijkende analyse die inzicht biedt in de verschillende beschikbare opties voor het ontwikkelen van een virtuele supportassistent. Hierbij worden de voor- en nadelen van elke optie in kaart gebracht, zodat op basis van deze informatie een onderbouwde keuze kan worden gemaakt voor de uitwerking van een PoC. De literatuurstudie moet ook duidelijkheid verschaffen over de benodigde hardware- en softwarevereisten voor de ontwikkeling.

Op basis van de literatuurstudie moet een analyse worden opgesteld om na te gaan welke optie die werden besproken binnen de literatuurstudie het best in overweging worden genomen. De belangrijkste keuze die zal moeten gemaakt worden binnen deze use case is de keuze welke techniek zal worden toegepast. Iedere methode heeft zijn voor- en nadelen hebben moet dit weloverwogen worden voor het opstellen van de PoC en het LLM modellen die kunnen worden gebruikt voor de PoC.


\section{Requirement analyse}


Gekoppeld aan de literatuurstudie zullen interviews worden afgenomen. Het doel van deze interviews is om een helder overzicht te verkrijgen van de verschillende vereisten voor de virtuele assistent. Op basis van het overzicht uit de literatuurstudie kan vervolgens een meer gerichte selectie worden gemaakt van de verschillende opties. Het uiteindelijke doel is om voor twee à drie modellen een PoC uit te werken die verder getest kunnen worden. Het is met andere woorden van belang om de verschillende modellen zo uitgebreid mogelijk te onderzoeken, zodat zoveel mogelijk modellen getoetst kunnen worden aan de gevraagde vereisten.Eens de literatuurstudie is afgerond en de interviews zijn afgenomen, kan met behulp van een MoSCoW-analyse een rangschikking worden opgesteld van de verschillende beschikbare modellen. Deze rangschikking bepaalt welke modellen worden geselecteerd voor het uitwerken van een PoC.

\section{Long list}

In deze sectie wordt een overzicht gegeven van de verschillende mogelijkheden die beschikbaar zijn voor het opstellen van de PoC. Dit gaat zowel over de verschillende technieken en technologieën die kunnen toegepast worden om tot het eindresultaat te komen van de chatbot. Deze sectie geeft enkel een overzicht van de verschillende mogelijkheden voor een meer gedetailleerde inhoud kan de stand van zaken geraadpleegd worden.

\subsubsection{LLM model}

Gelet op het feit dat er extreem veel modellen bestaan is het niet nuttig om alle beschikbare modellen in rekening te brengen. 
De onderstaande lijst is gebaseerd op alle modellen die in de top ranking staan van ChatbotArena en LiveBench. 
Op deze manier hebben we alle best mogelijke modellen waaruit dan verder een selectie kan gemaakt worden. 
Gelet op de scope van de PoC wordt echter wel een voorkeur gegeven aan zogenaamde open modellen.

\subsubsection{Toegepaste techniek}

\section{Short list}

\subsubsection{LLM model}

Voor het opstellen van de PoC wordt gekozen om te werken met zogenaamde open modellen. Naast het feit dat het mogelijk is om deze modellen gratis te gebruiken is het eveneens mogelijk om deze modellen lokaal te installeren en te gaan gebruiken. Aangezien de modellen lokaal worden gebruikt is er binnen het opstellen van deze PoC geen problemen met privacy schending. De gebruikte data en query's worden enkel gebruikt op de eigen lokale hardware.

\subsubsection{Toegepaste techniek}

%TODO hier verantwoorden waarom RAG

\section{Uit te voeren testen en analyse resultaten}

Elk van de geselecteerde modellen wordt vervolgens uitgewerkt in een PoC. Zodra de verschillende PoC’s beschikbaar zijn, wordt een vergelijking gemaakt tussen de modellen. Hierbij worden de volgende aspecten getest:

\begin{itemize}
    \item Wat is de kwaliteit van de antwoorden volgens wetenschappelijke meetcriteria?
    \item Hoeveel tijd kost het om een vraag te beantwoorden?
    \item Welk model geniet de voorkeur van de gebruiker?
\end{itemize}

De kwaliteit van de gegenereerde antwoorden wordt geëvalueerd met behulp van de meetinstrumenten ROUGE, BLEU en BERTScore. Elk van deze methoden richt zich op verschillende kenmerken van tekstkwaliteit, zoals overeenkomsten in formulering en inhoudelijke gelijkenis. Door deze drie evaluatiemethoden samen te gebruiken, ontstaat een gebalanceerd beeld van de prestaties van de modellen.

Daarnaast wordt de snelheid waarmee de modellen antwoorden genereren met elkaar vergeleken. Tot slot worden de antwoorden voorgelegd aan een groep gebruikers, die op basis van realistische situaties hun voorkeur kunnen aangeven. Op die manier worden naast objectieve ook subjectieve aspecten, meegenomen in de beoordeling.


