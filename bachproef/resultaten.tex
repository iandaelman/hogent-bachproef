%%=============================================================================
%% Resultaten
%%=============================================================================

\chapter{Resultaten}
\label{ch:resultaten}

Om de verschillende modellen te testen, worden meerdere scenario’s gehanteerd.
Ten eerste wordt gebruikgemaakt van een testset met tien vragen waarvan de informatie beschikbaar is in de documentatie. Er wordt verwacht dat het model op iedere vraag een concreet antwoord geeft. Deze antwoorden worden vervolgens geëvalueerd met behulp van het test framework Ragas, waarbij vier verschillende meetcriteria worden toegepast.
\\[1em]
Het tweede testscenario bestaat uit het stellen van niet-triviale vragen waarvoor de informatie niet beschikbaar is in de documentatie. Dit scenario is voornamelijk bedoeld om te onderzoeken of bepaalde modellen tekenen van hallucinaties vertonen.
\\[1em]
Ten slotte worden enkele triviale vragen gesteld. Hierbij wordt getest of het model de juiste inschatting maakt door direct te antwoorden zonder onnodige opzoekingen in de vector database.

\section{Testscenario 1}

De volledige vragenlijst met het verwachte antwoord is terug te vinden in \ref{vragenlijst}


\section{Testscenario 2}

\section{Testscenario 3}