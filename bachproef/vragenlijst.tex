\section{Vragenlijst en verwacht antwoord}
\label{vragenlijst}

\begin{enumerate}
    \item \textbf{Wie is de product owner van AGDP en kan je de contactgegevens geven?} \\
    De product owner van AGDP is XXXXXXX en je kan haar contacteren via\\
    \texttt{XXXXXXX@XXXXXXX.be}.
    
    \item \textbf{Wie is de product owner van AGPR en kan je de contactgegevens geven?} \\
    De product owner van AGPR is XXXXXXX en je kan hem contacteren via\\
    \texttt{XXXXXXX@XXXXXXX.be}.
    
    \item \textbf{Kan je me alle contactgegevens geven voor het fragment UBO-register, ook de SNOW groep?} \\
    Fragment: UBO-register – Contactgegevens
    \begin{itemize}
        \item ICT:
        \item Group SNOW: TRES\_GRP\_CIFI
        \item Service manager: XXXXXXX (back-up: XXXXXXX)
        \item Business:
        \item Business analyst: XXXXXXX
    \end{itemize}
    
    \item \textbf{Hoe of waar moet ik mijn tijd loggen voor de support?} \\
    Where to log support time (from most preferred to least preferred):
    \begin{enumerate}
        \item Related ticket
        \begin{itemize}
            \item Project: MYMINFIN
            \item Log time to an existing issue whenever possible.
        \end{itemize}
        \item New bug
        \begin{itemize}
            \item Project: MYMINFIN
            \item If the reported problem requires creating a bug issue, log your time in an “analysis” subtask.
        \end{itemize}
        \item Business Support
        \begin{itemize}
            \item Fisc: [MYMINFIN-9958] [support] MYMINFIN 3rd line AGFISC/AAFISC - JIRA Cloud – ACTIVE
            \item Rec: MYMINFIN-12102: Recouvrement Support – ACTIVE
            \item Patdoc: MYMINFIN-12104: PatDoc Support – ACTIVE
            \item Douane: MYMINFIN-12106: Douane Support – ACTIVE
            \item Trésorerie: MYMINFIN-12108: Trésorerie Support – ACTIVE
            \item (Until end of August, except FISC)
        \end{itemize}
        \item Transversal Support (last resort if none of the above applies)
        \begin{itemize}
            \item Transversal: MYMINFIN-12182: Transversal Support – ACTIVE
            \item ICT: MYMINFIN-12110: Support Day – ACTIVE
            \item (Until end of August)
        \end{itemize}
    \end{enumerate}
    
    \item \textbf{Een gebruiker heeft geen toegang tot een functionaliteit wat moet ik hiermee doen?} \\
    Vragen met betrekking tot toegangsrechten moeten voortaan worden doorgestuurd naar de algemene mailboxen van de IAM-cellen.\\
    \textbf{Let op:} deze adressen MOGEN NIET aan klanten worden meegedeeld.\\
    Fiscaliteit: XXXXXXX@XXXXXXX.be\\
    Invordering \& Innen: XXXXXXX@XXXXXXX.be \\
    Patrimoniumdocumentatie: XXXXXXX@XXXXXXX.be\\
    Douane \& Accijnzen: XXXXXXX@XXXXXXX.be\\
    Thesaurie: XXXXXXX@XXXXXXX.be\\
    Algemeen: XXXXXXX@XXXXXXX.be
    
    \item \textbf{Er is een probleem met attest service, wie kan ik hiervoor contacteren?} \\
    Je kan contact opnemen met het development team via \texttt{XXXXXXX@XXXXXXX.be}.
    
    \item \textbf{Hoe doe ik een repush voor RV?} \\
    Hier zijn de stappen die gevolgd moeten worden wanneer het RV-team vraagt om een “repush” uit te voeren:
    \begin{enumerate}
        \item Open MyMinfin Civil Servant in een incognitovenster.
        \item Open een tweede tabblad en zet er alvast: \texttt{http://www.finbel/}
        \item Zorg ervoor dat u over de volgende informatie beschikt:
        \begin{itemize}
            \item Application Type
            \item Status
            \item ID
        \end{itemize}
        \item In het tweede tabblad, plak een link met de volgende structuur achter:\\
      \begin{verbatim}
{http://www.finbel/myminfin-rest/myminfin-files-stirint
  /put-in-queue/\{applicationType\}/\{status\}/\{id\}}
      \end{verbatim}
        Voorbeeld:
        \begin{verbatim}
{myminfin-rest/myminfin-files-stirint/put-in-queue-by-id
    /WITHHOLDING\_TAX/VALIDATED/204832}
        \end{verbatim}
        \textbf{Opmerking:} Als u het volledige adres van de aanvrager krijgt, let er dan op dat er niet te veel spaties in staan.
        \item Druk op ENTER.
        \item Als alles goed verlopen is, geeft het systeem een succesbericht terug dat gecombineerd moet zijn met de melding 'queued files: 1'.\\
        Voorbeeld: 
        \begin{verbatim}
{"restMessages":[{"type":"SUCCESS",
        "bundleKey":"queued files : 1",...}]}
        \end{verbatim}
      \end{enumerate}
        
        \item \textbf{Iemand wil een document verwijderen dat hij heeft opgeladen via MyMinfin, wat kan ik hierop antwoorden?} \\
        Je kan het volgende antwoorden: MYMINFIN slaat geen gegevens op. Als het proces niet toelaat om te annuleren, moet de gebruiker contact opnemen met de afdeling die het document heeft ontvangen.\\
        Ik vermoed dat dit document is verstuurd via Antwoord op een brief - in dit geval moet de klant de gebruikte driecijferige code meedelen en moet je het ticket doorsturen naar het SOH-team. Die kunnen je vertellen met welke bevoegde service contact moet worden opgenomen.\\
        Wij kunnen u niet helpen.
        
        \item \textbf{Een burger kan werkt bij een onderneming, maar kan niet inloggen in de naam van de onderneming, wat moet ik hierop antwoorden?} \\
        Indien u niet de wettelijke vertegenwoordiger van de onderneming bent, hebt u een rol nodig om in te loggen in MYMINFIN in naam van uw eigen onderneming (inclusief filialen).\\
        U wilt in MYMINFIN toegang tot:
        \begin{itemize}
            \item de berichten van de onderneming, dan hebt u de rol FODFIN Berichten van de onderneming nodig;
            \item de documenten en formaliteiten van de onderneming, dan hebt u meestal genoeg aan de rol FOD Fin Aanstelling Eigen Onderneming met één of meerdere parameters in functie van uw verantwoordelijkheden (bv. Om de BTW formaliteiten te beheren, hebt u de parameter ‘Aanstelling BTW' nodig).
            \end{itemize}
Indien uw onderneming meerdere filialen telt, hebt u enkel de rol nodig die volstaat voor elk van die ondernemingen. Vergeet het begrip ‘mandaten'. Die zijn niet nodig als het om uw eigen onderneming (inclusief filialen) gaat.\\
Voor meer informatie kan u de volgende websites raadplegen:
\begin{itemize}
\item Welke rol ik moet toekennen : Rollen
\item Hoe toekennen of aanvragen van een rol : Beheer van de toegangsaanvragen in de rollenadministratie- RMA - Loginhulp.be
\end{itemize}
Eens dit in orde is, logt u aan op MyMinfin en identificeert zich “in naam van een onderneming”.\\
Selecteer uw eigen onderneming (of een filiaal).

\item \textbf{Wat zijn de contactgegevens voor de hypoimage service?} \\
service : AGDP - HYPOIMAGE \\
contacts : \texttt{XXXXXXX@XXXXXXX.be} ou PATDOC-RZSJ-Hypo (1e lijn) \\
- XXXXXXX@XXXXXXX.be (ICT - servicemanager ad interim) \\
- XXXXXXX@XXXXXXX.be (ICT - développeur)
\end{enumerate}
